\documentclass{parcfd2014}
\usepackage{amsmath, amssymb, amsthm, mathtools}
\usepackage{graphicx}
\usepackage{cancel}
\usepackage{color}
\usepackage{caption}
\usepackage{subcaption}
%\usepackage{subfigure}
\usepackage{pgf,tikz}
\usetikzlibrary{arrows}

\captionsetup{compatibility=false}

\newcommand{\bs}[1]{\boldsymbol{#1}}
\newcommand{\norm}[1]{\left\| #1 \right\|}
\newcommand{\snorm}[1]{\left| #1 \right|}
\newcommand{\LRp}[1]{\left( #1 \right)}
\newcommand{\LRs}[1]{\left[ #1 \right]}
\newcommand{\LRc}[1]{\left\{ #1 \right\}}
\newcommand{\LRa}[1]{\left\langle #1 \right\rangle}
\newcommand{\LRb}[1]{\left| #1 \right|}
\newcommand{\Grad}{\ensuremath{\nabla}}
\newcommand{\Gradxt}{\ensuremath{\nabla_{xt}}}
\newcommand{\Div}{\ensuremath{\nabla\cdot}}
\newcommand{\Divxt}{\ensuremath{\nabla_{xt}\cdot}}
\newcommand{\Curl}{\ensuremath{\nabla\times}}
\newcommand{\bfH}{\mbox{\boldmath $H$}}
\newcommand{\bfsigma}{\boldsymbol\sigma}
\newcommand{\bfvarsigma}{\boldsymbol\varsigma}
\newcommand{\bftau}{\boldsymbol\tau}
\newcommand{\bfbeta}{\boldsymbol\beta}
\newcommand{\bflambda}{\boldsymbol\lambda}
\newcommand{\bfpsi}{\boldsymbol\psi}
\newcommand{\bfu}{\boldsymbol u}
\newcommand{\bfv}{\boldsymbol v}
\newcommand{\bfV}{\boldsymbol V}
\newcommand{\bfZ}{\boldsymbol Z}
\newcommand{\bfz}{\boldsymbol z}
\newcommand{\bfW}{\boldsymbol W}
\newcommand{\bfw}{\boldsymbol w}
\newcommand{\bfm}{\boldsymbol m}
\newcommand{\bfM}{\boldsymbol M}
\newcommand{\bbM}{\mathbb{M}}
\newcommand{\bfq}{\boldsymbol q}
\newcommand{\bfU}{\boldsymbol U}
\newcommand{\bfS}{\boldsymbol S}
\newcommand{\bbS}{\mathbb{S}}
\newcommand{\bbD}{\mathbb{D}}
\newcommand{\bfK}{\boldsymbol K}
\newcommand{\bbK}{\mathbb{K}}
\newcommand{\bfn}{\boldsymbol n}
\newcommand{\bff}{\boldsymbol f}
\newcommand{\bfF}{\boldsymbol F}
\newcommand{\bbF}{\mathbb{F}}
\newcommand{\bfg}{\boldsymbol g}
\newcommand{\bfG}{\boldsymbol G}
\newcommand{\bfC}{\boldsymbol C}
\newcommand{\bft}{\boldsymbol t}
\newcommand{\bfT}{\boldsymbol T}
\newcommand{\bfI}{\boldsymbol I}
\newcommand{\bbI}{\mathbb{I}}
\newcommand{\bfx}{\boldsymbol x}
\newcommand{\uh}{\widehat{u}}
\newcommand{\fnh}{\widehat{f}_n}
\newcommand{\LQ}{L^2\LRp{Q}}
\newcommand{\LK}{L^2\LRp{K}}
\newcommand{\LVecK}{\mathbf{L}^2\LRp{K}}
\newcommand{\LVecQ}{\mathbf{L}^2\LRp{Q}}
\newcommand{\HdivK}{\bfH(\text{div},K)}
\newcommand{\HdivOmega}{\bfH(\text{div},\Omega)}
% \newcommand{\HdivOmegaLT}{\bfH(\text{div},\Omega)\times L^2([0,T])}
\newcommand{\HdivQ}{\bfH(\text{div}_{xt},Q)}
\newcommand{\HOneK}{H^{1}(K)}
\newcommand{\HOneVecK}{\bfH^{1}(K)}
\newcommand{\HOneQ}{H^{1}(Q)}
\newcommand{\HOneOmegah}{H^{-1}(\Omega_h)}
\newcommand{\HdivOmegah}{\bfH(\text{div},\Omega_h)}
\newcommand{\vdeltau}{v_{\delta\bs u_h}}
\newcommand{\taudeltau}{\bftau_{\delta\bs u_h}}
\newcommand{\ip}[1]{\left\langle #1 \right\rangle}
\newcommand{\pd}[2]{\frac{\partial#1}{\partial#2}}
\newcommand{\pt}[1]{\frac{\partial#1}{\partial t}}
\newcommand{\ppd}[2]{\frac{\partial^2#1}{\partial#2^2}}
\newcommand{\pdd}[3]{\frac{\partial^2#1}{\partial#2\partial#3}}
\newcommand{\der}[2]{\frac{\mathrm{d}#1}{\mathrm{d}#2}}
\newcommand{\Oh}{\Omega_h}
\newcommand{\jump}[1] {\ensuremath{\LRs{\![#1]\!}}}
\newcommand{\Gh}{\Gamma_h}
\newcommand{\mcU}{\mathcal{U}}
\newcommand{\mcUh}{\hat{\mathcal{U}}}
\newcommand{\LOmega}{L^2\LRp{\Omega_h}}

\newcommand{\eqnref}[1]{\eqref{eq:#1}}

\DeclareMathOperator*{\argmin}{arg\,min}
\DeclareMathOperator*{\trace}{tr}

\def\arrtwo#1#2#3#4{\left[
\begin{array}{cc}
#1\; & #2\\
#3\; & #4\\
\end{array}
\right]}
\def\arrthree#1#2#3#4#5#6#7#8#9{\left[
\begin{array}{ccc}
#1\; & #2\; & #3\\
#4\; & #5\; & #6\\
#7\; & #8\; & #9\\
\end{array}
\right]}
\def\arrthreeone#1#2#3{\left[
\begin{array}{ccc}
#1\; & #2\; & #3\\
\end{array}
\right]}
\def\vecttwo#1#2{\left(
\begin{array}{c}
#1\\
#2\\
\end{array}
\right)}
\def\svecttwo#1#2{\left[
\begin{array}{c}
#1\\
#2\\
\end{array}
\right]}
\def\vectthree#1#2#3{\left(
\begin{array}{c}
#1\\
#2\\
#3\\
\end{array}
\right)}
\def\svectthree#1#2#3{\left[
\begin{array}{c}
#1\\
#2\\
#3\\
\end{array}
\right]}

\renewcommand{\arraystretch}{1.2}

\def\etal{{\it et al.~}}

\graphicspath{{../Proposal/figs/}}

\title{Space-Time DPG: Designing a Method for Massively Parallel CFD}

\author{Truman E. Ellis$^{*}$, Leszek F. Demkowicz$^{*}$, Nathan V. Roberts$^{\dag}$,\\ Jesse L. Chan$^{\ddag}$ and Robert D. Moser$^{*}$}

\heading{Truman E. Ellis, Leszek F. Demkowicz, Nathan V. Roberts, Jesse L. Chan and Robert D. Moser}

\address{$^{*}$ Institute for Computational Engineering and Sciences,
University of Texas at Austin\\
201 East 24th St, Stop C0200, Austin, TX 78703
% e-mail: truman@ices.utexas.edu
\and
$^{\dag}$Argonne Leadership Computing Facility,
Argonne National Laboratory\\
9700 South Cass Avenue Building 240, Argonne, IL 60439
\and
$^{\ddag}$Computational and Applied Mathematics,
Rice University\\
6100 Main MS-134 Houston, TX 77005
}

% \keywords{CFD methodology, Parallel methodology, Parallel CFD applications}
\keywords{Discontinuous Petrov-Galerkin, Space-time finite elements}

\abstract{
We develop a space-time discontinuous Petrov-Galerkin finite element method ideal for parallel simulation of transient fluid dynamics problems.
}

\begin{document}
\section{Introduction}
\subsection{Motivation}
\subsection{Overview of DPG}
The discontinuous Petrov-Galerkin finite element method with optimal test functions 
was first proposed by Demkowicz and Gopalakrishnan in 2009\cite{DPG1, DPG2}.
The basic ideas are fairly straight-forward; DPG minimizes the residual in a user defined energy norm.
Consider a variational problem: find $u\in U$ such that
\[
b(u,v)=l(v) \quad\forall v\in V
\]
with operator $B:U\rightarrow V'\quad$ ($V'$ is the dual space to $V$) defined by $b(u,v)=\LRa{Bu,v}_{V'\times V}$.
This gives the operator equation:
\[
Bu=l\in V'\,.
\]
We wish to minimize the residual $Bu-l$ in $V'$:
\[
u_h=\argmin_{w_h\in U_h}\frac{1}{2}\norm{Bu-l}^2_{V'}\,.
\]
This is a very natural mathematical framework based soundly in functional analysis, but it is not yet a practical method as the $V'$ norm is not
especially tractable to work with.
The insight is that since we are working with Hilbert spaces, we can use the Riesz representation theorem to find a complementary object 
in $V$ rather than $V'$. Let $R_V:V\ni v\rightarrow(v,\cdot)\in V'$ be the Riesz map. 
Then the inverse Riesz map (which is an isometry) lets us represent our residual in $V$:
\[
u_h=\argmin_{w_h\in U_h}\frac{1}{2}\norm{R_V^{-1}(Bu-l)}^2_{V}\,.
\]
Taking the G\^ateaux derivative to be zero in all directions $\delta u \in
U_h$ gives,
\[
\left(R_V^{-1}(Bu_h-l),R_V^{-1}B\delta u\right)_V = 0, \quad \forall \delta u \in U,
\]
which by definition of the Riesz map is equivalent to 
\begin{equation*}
\LRa{Bu_h-l,R_V^{-1}B\delta u_h}=0\quad\forall\delta u_h\in U_h\,,
\end{equation*}
with optimal test functions $v_{\delta u_h}\coloneqq R_V^{-1}B\delta u_h$ for each trial function $\delta u_h$.
This gives a simple bilinear form
\begin{equation*}
b(u_h,v_{\delta u_h})=l(v_{\delta u_h}),
\end{equation*}
with $v_{\delta u_h}\in V$ that solves the auxiliary problem
\begin{equation*}
\LRp{v_{\delta u_h},\delta v}_V=\LRa{R_Vv_{\delta u_h},\delta v}
=\LRa{B\delta u_h,\delta v}=b(\delta u_h,\delta v)\quad\forall\delta v\in V.
\end{equation*}
We might call this an \emph{optimal Petrov-Galerkin}.
We arrive at the same method by realizing the supremum in inf-sup condition, motivating the \emph{optimal} nomenclature.
These optimal Petrov-Galerkin methods produce Hermitian, positive-definite stiffness matrices since
\[
b(u_h,\vdeltau)=(v_{u_h},\vdeltau)_V=\overline{(\vdeltau,v_{u_h})}=\overline{b(\delta u_h,v_{u_h})}\,.
\]
We can calculate the energy norm (defined by $\norm{u}_E:=\norm{Bu}_{V'}$) of the Galerkin error without knowing the exact solution by using the residual:
\[
\norm{u_h-u}_E=\norm{B(u_h-u)}_{V'}=\norm{Bu_h-l}_{V'}=\norm{R_V^{-1}(Bu_h-l)}_V\,,
\]
where we designate $R_V^{-1}(Bu_h-l)$ the \emph{error representation function}.
This has proven to be a very reliable \emph{a-posteriori} error estimator for driving adaptivity.

Babu\v{s}ka's theorem\cite{Babuska70} says that discrete stability and approximability imply convergence.
That is, if $M$ is the continuity constant for $b(u,v)$ which satisfies the discrete inf-sup condition with constant $\gamma_h$,
\[
\sup_{v_h\in V_h}\frac{|b(u,v)|}{\norm{v_h}_V}\geq\gamma_h\norm{u_h}_U\,,
\]
then the Galerkin error satisfies the bound
\[
\norm{u_h-u}_U\leq\frac{M}{\gamma_h}\inf_{w_h\in U_h}\norm{w_h-u}_U\,.
\]
Optimal test functions realize the supremum in the discrete discrete inf-sup condition such that $\gamma_h\geq\gamma$, 
the infinite-dimensional inf-sup constant.
If we then use the energy norm for $\norm{\cdot}_U$, then $M=\gamma=1$ and Babu\v{s}ka's estimate implies that
the optimal Petrov-Galerkin method is the most stable Petrov-Galerkin method possible.

There are still many features of the method that are left to be decided, for example the $U$ and $V$ spaces.
If $V$ is taken to be a continuous space, then the auxiliary problem becomes global in scope, something that we would like to avoid.
In order to ensure the auxiliary problem can be solved element-by-element, we take $V$ to be discontinuous between elements.
(Technically, $V$ should also be infinite dimensional, but we have found it to be sufficient to use an ``enriched'' space of higher
polynomial dimension than the trial space.)
The downside to using discontinuous test functions is that it introduces new interface unknowns.
When the equations are integrated by parts over each element, the jump in test functions introduces new unknowns on the mesh skeleton
that would have gone away with continuous test functions.
Moro \etal\cite{MoroNguyenPeraire11} handle the flux unknowns with a numerical flux in the hybridized DPG method, but the standard DPG method treats
these as new unknowns to be solved for.
We still haven't specified our trial space $U$, but the rule is that for ever integration by parts, a new skeleton unknown is introduced.
Most DPG considerations break a second order PDE into a system of first order PDEs which introduces a trace unknown (from the constitutive law) 
and a flux unknown (from the conservation law), but Demkowicz and Gopalakrishnan also formulated a \emph{primal DPG} method for second order equations
that does not introduce a trace unknown.
The overall number of interface unknowns in the primal DPG method is the same, however, since the solution is required to be $H^1$ conforming 
and the trace unknowns are essentially hidden here.

The final unresolved choice is what norm to apply to the $V$ space.
This is one of the most important factors in designing a robust DPG method as this norm needs to be inverted to solve for the optimal test functions.
If the norm produces unresolved boundary layers in the auxiliary problem, then many of the attractive features of DPG may fall apart.
But elimination of boundary layers in the auxiliary solve is not the only requirement at play. 
This choice also controls what norm the residual is minimized in. 
Often we want this norm to be equivalent to the $L^2$ norm.
Fortunately, we have found that it is possible to design test norms such that the implied energy norm 
is provably robust and equivalent to $L^2$ for convection-diffusion which
serves as the most relevant model problem for our research.
Norms for Navier-Stokes are derived by analogy to the convection-diffusion norm.

DPG has been successfully applied to a wide range of physical problems.
Early work on the Poisson equation was published in \cite{DPGPoisson}.
Demkowicz \etal\cite{DPGHelmholtz}, Gopalakrishnan \etal\cite{Gopalakrishnan2014}, and Zitelli \etal\cite{DPG4} 
analyzed and solved the Helmholtz equation with DPG.
DPG was applied to linear elasticity and plate problems in \cite{BramwellDPG}, \cite{NiemiBramwellDemkowicz10}, and \cite{BramwellDemkowiczQiu10}.
A 2D Maxwell cloaking problem was solved wit DPG in \cite{DPGCloaking} 
and a 3D DPG theory for Maxwell was developed by Wieners and Wohlmuth\cite{WohlmuthReport}.
DPG has been applied to various fluid problems including convection-diffusion\cite{DPG3,DemkowiczHeuer,ChanHeuerThanhDemkowicz2012,Chan2013,Ellis2013Report},
Stokes\cite{DPGStokes,Ellis2013Report}, Burgers' equation\cite{Chan2013dpg}, incompressible Navier-Stokes\cite{NateDissertation}, 
and compressible Navier-Stokes\cite{JesseDissertation}.
\subsection{DPG for High Performance Computing}
Finally a few notes on some of the implications for high performance computing.
The goal is to design a method that eliminates human intervention as much as possible.

* Exceptional stability properties prevent a solution from crashing, eliminating expensive restarts
* The method runs robustly on a wide range of Reynolds numbers
* Adaptivity allocates degrees of freedom efficiently to allow larger simulations with fewer resources
* Automaticity means that you can start a simulation and let it solve and adapt without needing to jump in and fix things part way through
* DPG is very compute intensive, with a large portion of the work being done in the embarrassingly parallel local solves and stiffness matrix assembly. Also, higher order methods tend to prevent a better compute/memory/communication profile than low order methods, and exceptional stability makes high order very easy.
* In our code, we do the local solves via QR factorization. But the local solve can be viewed as having multiple right hand sides, so that QR factorization can be recycled multiple times.
* Degrees of freedom can be separated into two categories: internal and trace. The internal degrees of freedom can be condensed out and the global solve can be conducted purely in terms of the trace variables which have limited coupling, reducing fill in of the stiffness matrix. The internal DOFs can then be solved for in an embarrassingly parallel post-processing phase. This is a similar process behind the hybridized DG method.
* As I mentioned earlier, an SPD stiffness matrix seems like a promising thing for iterative solvers, but we haven't really explored the benefits.
* Many supercomputer simulations are increasingly coupling multiphysics. The only stability requirement for a DPG method is that the continuous problem is well-posed. As such, it has successfully been applied to a wide range of problems from Helmholtz to solid mechanics to Navier-Stokes and Maxwell's equations.

\section{Space-Time DPG}
We summarize some completed work on space-time DPG. At the time of writing, Camellia does not officially support space-time computations, but we can hack it for 1D spatial problems by pretending the $y$-direction is time.
Complications arise when the PDE under consideration is parabolic (i.e. contains second derivatives in space, but only first derivatives in time). 
Mathematically, this leaves traces undefined on element edges without a spatial normal component. 
Practically, this means that I had to hack the Camellia code in order to support these ``spatial traces''.
We say that the code was ``hacked'' to indicate that we modified the code in an ``ugly'' manner in order to obtain the following results, 
but plan is to do this according to better software practices in the proposed work, since the current implementation is not very maintainable.

All of the following problems (unless otherwise noted) are run with the graph norm which is
simply defined from the adjoint of the system supplemented with $L^2$ terms to upgrade it to a full norm.


%   /$$   /$$                       /$$    
%  | $$  | $$                      | $$    
%  | $$  | $$  /$$$$$$   /$$$$$$  /$$$$$$  
%  | $$$$$$$$ /$$__  $$ |____  $$|_  $$_/  
%  | $$__  $$| $$$$$$$$  /$$$$$$$  | $$    
%  | $$  | $$| $$_____/ /$$__  $$  | $$ /$$
%  | $$  | $$|  $$$$$$$|  $$$$$$$  |  $$$$/
%  |__/  |__/ \_______/ \_______/   \___/  
%                                          
%                                          
%  
\section{Heat Equation}
The simplest space-time problem we can consider where the the spatial and temporal dimensions are treated differently is the heat equation.
We start with a general $n$-dimensional spatial derivation and later simplify to spatially 1D with a few numerical experiments.

\subsection{Derivation}
Let $\Omega(t)\subset\mathbb{R}^d$ be the spatial domain with boundary $\partial\Omega$.
The heat equation is
\begin{equation}
	\frac{\partial u}{\partial t}-\mu\Delta u=f\,,\quad\bfx\in\Omega\,,\;t\in(t_0,T)
\end{equation}
where $u$ is unknown heat, $\epsilon$ is the diffusion scale, $f$ is the source term, $t_0$ is the start time, and $T$ is the final time.
Let $Q\subset\mathbb{R}^{d+1}$ denote the full space-time domain which is then tessellated into space-time elements $K$.

The second order formulation of the heat equation is really just a composition of Fourier's law and conservation of energy:
\begin{equation}
\label{eq:heatFirstOrder}
\begin{aligned}
\bfsigma-\epsilon\Grad u&=0\\
\frac{\partial u}{\partial t}-\Div\bfsigma&=f\,,
\end{aligned}
\end{equation}
where $\bfsigma$ is the heat flux.
The key insight that we will use over and over in the following problems is that we can rewrite our conservation equation
in terms of a space-time divergence operator: $\Divxt():=\Div()+\pt{()}$.
Our new system is then
\begin{equation}
\label{eq:heatFirstOrderSpaceTime}
\begin{aligned}
\frac{1}{\epsilon}\bfsigma-\Grad u&=0\\
\Divxt\vecttwo{-\bfsigma}{u}&=f\,.
\end{aligned}
\end{equation}
We now proceed with the standard DPG practice and multiply by test functions $\bftau$ and $v$ 
and integrate by parts over each space-time element $K$:
\begin{equation}
\label{eq:heatBF}
\begin{aligned}
\LRp{\frac{1}{\epsilon}\bfsigma,\bftau}+\LRp{u,\Div\bftau}-\LRa{\hat u,\bftau\cdot\bfn_x}&=0\\
-\LRp{\vecttwo{-\bfsigma}{u},\Gradxt v}+\LRa{\hat t,v}&=f\,,
\end{aligned}
\end{equation}
where
\begin{align*}
\hat u&:=\trace(u)\\
\hat t&:=\trace(-\bfsigma)\cdot\bfn_x+\trace(u)\cdot n_t
\end{align*}
are new unknowns that live on the mesh skeleton introduced by the integration by parts.
Note that the constitutive law was only integrated by parts over spatial dimensions, which means 
that ``spatial trace'' $\hat u$ only exists on mesh boundaries with a nonzero spatial normal component.
On the other hand, flux $\hat t$ exists on all mesh boundaries, but changes nature between pure spatial and temporal edges while taking on 
a mixed nature on slanted boundaries. We illustrate the support of these skeleton variables in Figure~\ref{fig:heatMesh}.

\begin{figure}[h!]
\begin{tikzpicture}[line cap=round,line join=round,>=triangle 45,x=2.0cm,y=2.0cm]
\clip(-0.7,-1.01) rectangle (5.27,2.29);
\draw (0,2)-- (0,0);
\draw (0,0)-- (1,0);
\draw (1,0)-- (4,0);
\draw (4,0)-- (5,0);
\draw (5,0)-- (5,2);
\draw (5,2)-- (3,2);
\draw (3,2)-- (2,2);
\draw (2,2)-- (0,2);
\draw (1,0)-- (1.5,1);
\draw (1.5,1)-- (2,2);
\draw (1.5,1)-- (3.5,1);
\draw (3,2)-- (3.5,1);
\draw (3.5,1)-- (4,0);
\draw (-0.21,0.9) node[anchor=south west] {$\hat u$};
\draw (4.82,0.9) node[anchor=south west] {$\hat u$};
\draw (3.5,0.45) node[anchor=south west] {$\hat u$};
\draw (1.0,0.45) node[anchor=south west] {$\hat u$};
\draw (1.5,1.4) node[anchor=south west] {$\hat u$};
\draw (3.0,1.4) node[anchor=south west] {$\hat u$};
\draw (0.05,0.9) node[anchor=south west] {$\hat t$};
\draw (1.40,0.45) node[anchor=south west] {$\hat t$};
\draw (3.79,0.45) node[anchor=south west] {$\hat t$};
\draw (2.47,1.0) node[anchor=south west] {$\hat t$};
\draw (3.33,1.4) node[anchor=south west] {$\hat t$};
\draw (1.89,1.4) node[anchor=south west] {$\hat t$};
\draw (5.07,0.9) node[anchor=south west] {$\hat t$};
\draw (4.44,0.0) node[anchor=south west] {$\hat t$};
\draw (2.46,0.0) node[anchor=south west] {$\hat t$};
\draw (0.45,0.0) node[anchor=south west] {$\hat t$};
\draw (2.44,1.7) node[anchor=south west] {$\hat t$};
\draw (0.76,1.7) node[anchor=south west] {$\hat t$};
\draw (4.21,1.7) node[anchor=south west] {$\hat t$};
\draw [->] (-0.5,-0.5) -- (-0.5,0);
\draw [->] (-0.5,-0.5) -- (0,-0.5);
\draw (-0.54,0.29) node[anchor=north west] {$t$};
\draw (0.07,-0.35) node[anchor=north west] {$x$};
\end{tikzpicture}
\caption{Support of flux and spatial trace variables}
\label{fig:heatMesh}
\end{figure}

\subsection{Problems considered}
If we consider a domain $\Omega=[0,1]^2$ with an initial condition of $u=\cos(2\pi x)$ with zero flux conditions at the boundaries,
the exact solution is
\begin{equation*}
	u=\cos(2\pi x)e^{-4*\pi^2\epsilon t}\,.
\end{equation*}
We ran this with $\epsilon=10^{-2}$ on a sequence of uniform meshes and $p=2$ for the field representation of $u$. 
We were able to achieve the expected third order convergence as demonstrated in Figure \ref{fig:spaceTimeHeatConvergence}.

\begin{figure}[!ht]
	\centering
	\includegraphics[width=0.7\textwidth]{SpaceTimeHeat/convergence}
	\caption{$L^2$ convergence of $u$ for the space-time heat equation}
	\label{fig:spaceTimeHeatConvergence}
\end{figure}

In order to demonstrate local space-time adaptivity we consider one more problem for the heat equation. 
On the same domain, and with the same boundary conditions as the previous example, we let the initial heat distribution be zero.
Then between $t=0.25$ and $t=0.5$ we turn on a pulse source term of one on $0.375\leq x\leq 0.625$. 
Starting from an initial mesh of $4x4$, we adaptively refine four times and obtain the results in Figure \ref{fig:spaceTimeHeatPulse}.
Notice that $\hat u$ in Figure \ref{fig:spaceTimeHeatuhat} only lives on vertical edges as was discussed earlier.
Also notice that the full mesh shown in Figure \ref{fig:spaceTimeHeatfhat} automatically adapts spatially and temporally to where features are rapidly changing. 

\begin{figure}[p]
\centering
\begin{subfigure}[t]{0.45\textwidth}
\centering
\includegraphics[width=\textwidth]{SpaceTimeHeat/PulseSource/u.png}
\caption{$u$}
\label{fig:spaceTimeHeatu}
\end{subfigure}
\begin{subfigure}[t]{0.45\textwidth}
\centering
\includegraphics[width=\textwidth]{SpaceTimeHeat/PulseSource/sigma.png}
\caption{$\sigma$}
\label{fig:spaceTimeHeatsigma}
\end{subfigure}
\begin{subfigure}[t]{0.45\textwidth}
\centering
\includegraphics[width=\textwidth]{SpaceTimeHeat/PulseSource/uhat.png}
\caption{$\hat u$}
\label{fig:spaceTimeHeatuhat}
\end{subfigure}
\begin{subfigure}[t]{0.45\textwidth}
\centering
\includegraphics[width=\textwidth]{SpaceTimeHeat/PulseSource/fhat.png}
\caption{$\hat t$}
\label{fig:spaceTimeHeatfhat}
\end{subfigure}
\caption{Pulsed space-time heat problem after 4 refinements}
\label{fig:spaceTimeHeatPulse}
\end{figure}

%    /$$$$$$                       /$$$$$$                      /$$                    
%   /$$__  $$                     /$$__  $$                    |__/                    
%  | $$  \__/  /$$$$$$  /$$$$$$$ | $$  \__/ /$$   /$$  /$$$$$$$ /$$  /$$$$$$  /$$$$$$$ 
%  | $$       /$$__  $$| $$__  $$| $$$$    | $$  | $$ /$$_____/| $$ /$$__  $$| $$__  $$
%  | $$      | $$  \ $$| $$  \ $$| $$_/    | $$  | $$|  $$$$$$ | $$| $$  \ $$| $$  \ $$
%  | $$    $$| $$  | $$| $$  | $$| $$      | $$  | $$ \____  $$| $$| $$  | $$| $$  | $$
%  |  $$$$$$/|  $$$$$$/| $$  | $$| $$      |  $$$$$$/ /$$$$$$$/| $$|  $$$$$$/| $$  | $$
%   \______/  \______/ |__/  |__/|__/       \______/ |_______/ |__/ \______/ |__/  |__/
%                                                                                      
%                                                                                      
%        
% \section{Convection-Diffusion}
% Transient convection-diffusion is identical to the heat equation with the addition of a convective term:
% \begin{equation*}
% \frac{\partial u}{\partial t}+\Div(\bfbeta u)-\epsilon\Delta u=f\,.
% \end{equation*}
% The $d$-dimensional transient convection-diffusion equations could be viewed as a $d+1$ steady convection-diffusion problem with zero diffusion in the time direction.

% \subsection{Derivation}
% As a first order system in space-time, this is
% \begin{equation}
% \label{eq:confusionFirstOrder}
% \begin{aligned}
% \frac{1}{\epsilon}\bfsigma-\Grad u&=0\\
% \Divxt\vecttwo{\bfbeta u-\bfsigma}{u}&=f\,.
% \end{aligned}
% \end{equation}
% % We can view the second equation as a full space-time divergence on group variable $\bfU:=\LRc{\bfbeta u-\bfsigma,u}$ as before:
% % \begin{equation}
% % 	\Divxt\LRp{\bfU}=f\,.
% % \end{equation}
% Multiplying \eqref{eq:confusionFirstOrder} by test functions, and integrating by parts over each element, we obtain the following bilinear form:
% \begin{equation}
% \label{eq:confusionBF}
% 	\begin{aligned}
% 		\LRp{\frac{1}{\epsilon}\bfsigma,\bftau}+\LRp{u,\Div\bftau}-\LRa{\hat u,\tau_n}&=0\\
% 		-\LRp{\vecttwo{\bfbeta u-\bfsigma}{u},\Gradxt v}+\LRa{\hat t,v}&=f\,,
% 		% -\LRp{u,\frac{\partial v}{\partial t}}-\LRp{\bfbeta u,\Grad v}+\LRp{\bfsigma,\Grad v}-\LRa{\hat t,v}&=\LRp{f,v}\,,
% 	\end{aligned}
% \end{equation}
% where now $\hat t=\trace\LRp{\bfbeta u-\bfsigma}\cdot\bfn_x+\trace(u)\cdot n_t$, and $\hat u$ is as before. Our test functions, $\bftau$ and $v$ live in the same spaces as for the heat equation.

% \subsection{Problems considered}
% Since space-time convection-diffusion is identical the heat equation with the addition of a convective term, we only pursue one numerical experiment to demonstrate that everything works as expected. 
% This problem is inspired by the previous heat problem with the spatial domain extended to prevent the convected heat from impinging on the right wall.
% It might be interesting to impose a zero boundary condition on $\hat u$ and watch a boundary layer build up on the right wall, 
% but instead we enforce a zero flux condition and content ourselves with the inner layer that forms around the source pulse.
% This is an arbitrary requirement necessitated by the ``hackish'' nature of this code.
% We haven't taken the time to allow enforcement of Dirichlet boundary conditions on spatial fluxes, and since the code in this state was intended to be short-lived, it doesn't make sense to invest too heavily in adding features. 
% Thus we enforce zero flux conditions on both walls as before. 
% For this problem, the domain extends from $[0,1.5]\times[0,1]$ with the pulse occurring at $[0.25,0.5]\times[0.25,0.5]$.

% \begin{figure}[p]
% \centering
% \begin{subfigure}[t]{0.45\textwidth}
% \centering
% \includegraphics[width=\textwidth]{SpaceTimeConfusion/u.png}
% \caption{$u$}
% \label{fig:spaceTimeConfusionu}
% \end{subfigure}
% \begin{subfigure}[t]{0.45\textwidth}
% \centering
% \includegraphics[width=\textwidth]{SpaceTimeConfusion/sigma.png}
% \caption{$\sigma$}
% \label{fig:spaceTimeConfusionsigma}
% \end{subfigure}
% \begin{subfigure}[t]{0.45\textwidth}
% \centering
% \includegraphics[width=\textwidth]{SpaceTimeConfusion/uhat.png}
% \caption{$\hat u$}
% \label{fig:spaceTimeConfusionuhat}
% \end{subfigure}
% \begin{subfigure}[t]{0.45\textwidth}
% \centering
% \includegraphics[width=\textwidth]{SpaceTimeConfusion/fhat.png}
% \caption{$\hat t$}
% \label{fig:spaceTimeConfusionfhat}
% \end{subfigure}
% \caption{Space-time convection-diffusion problem after 4 refinements}
% \label{fig:spaceTimeConfusion}
% \end{figure}


%    /$$$$$$                                                                            /$$ /$$       /$$          
%   /$$__  $$                                                                          |__/| $$      | $$          
%  | $$  \__/  /$$$$$$  /$$$$$$/$$$$   /$$$$$$   /$$$$$$   /$$$$$$   /$$$$$$$  /$$$$$$$ /$$| $$$$$$$ | $$  /$$$$$$ 
%  | $$       /$$__  $$| $$_  $$_  $$ /$$__  $$ /$$__  $$ /$$__  $$ /$$_____/ /$$_____/| $$| $$__  $$| $$ /$$__  $$
%  | $$      | $$  \ $$| $$ \ $$ \ $$| $$  \ $$| $$  \__/| $$$$$$$$|  $$$$$$ |  $$$$$$ | $$| $$  \ $$| $$| $$$$$$$$
%  | $$    $$| $$  | $$| $$ | $$ | $$| $$  | $$| $$      | $$_____/ \____  $$ \____  $$| $$| $$  | $$| $$| $$_____/
%  |  $$$$$$/|  $$$$$$/| $$ | $$ | $$| $$$$$$$/| $$      |  $$$$$$$ /$$$$$$$/ /$$$$$$$/| $$| $$$$$$$/| $$|  $$$$$$$
%   \______/  \______/ |__/ |__/ |__/| $$____/ |__/       \_______/|_______/ |_______/ |__/|_______/ |__/ \_______/
%                                    | $$                                                                          
%                                    | $$                                                                          
%                                    |__/    
\section{Transient Compressible Navier-Stokes}
We make a large jump from convection-diffusion to the compressible Navier-Stokes equations.
The following discussion holds in any dimension, but the provided results are only for spatially 1D flows.
The compressible Navier-Stokes equations are
\begin{align}
\frac{\partial}{\partial t}\svectthree{\rho}{\rho\bfu}{\rho e_0}
+\Div\svectthree{\rho\bfu}{\rho\bfu\otimes\bfu+p\bfI-\mathbb{D}}{\rho\bfu e_0+\bfu p+\bfq-\bfu\cdot\mathbb{D}}
%TODO: Possible error above. cfd-online seems to have T^T
=\svectthree{f_c}{\bff_m}{f_e}\,,
\end{align}
where $\rho$ is the density, $\bfu$ is the velocity, $p$ is the pressure, $\bfI$ is the identity matrix,
$\mathbb{D}$ is the deviatoric stress tensor or viscous stress, $e_0$ is the total energy, $\bfq$ is the heat flux, 
and $f_c$, $\bff_m$, and $f_e$ are the source terms for the continuity, momentum, and energy equations, respectively.
Assuming Stokes hypothesis that $\lambda=-\frac{2}{3}\mu$, 
\begin{equation*}
	\mathbb{D}=2\mu\bfS^*=2\mu\LRs{\frac{1}{2}\LRp{\Grad\bfu+\LRp{\Grad\bfu}^T}-\frac{1}{3}\Div\bfu\bfI}\,,
\end{equation*}
where $\bfS^*$ is the trace-less viscous strain rate tensor.
The heat flux is given by Fourier's law:
\begin{equation*}
	\bfq=-C_p\frac{\mu}{Pr}\Grad T\,,
\end{equation*}
where $C_p$ is the specific heat at constant pressure and $Pr$ is the laminar Prandtl number: $Pr:=\frac{C_p\mu}{\lambda}$.
We need to close these equations with an equation of state. An ideal gas assumption gives
\begin{equation*}
	\gamma:=\frac{C_p}{C_v}\,,\quad p=\rho RT\,,\quad e=C_v T\,,\quad C_p-C_v=R\,,
\end{equation*}
where $\gamma$ is the ratio of specific heats, $C_v$ is the specific heat at constant volume, $R$ is the gas constant,
$e$ is the internal energy, $T$ is the temperature,
and $\gamma$, $C_p$, $C_v$, and $R$ are constant properties of the fluid.
The total energy is defined by
\begin{equation*}
	e_0=e+\frac{1}{2}\bfu\cdot\bfu\,.
\end{equation*}

We can write our first order system of equations in space-time as follows:
\begin{subequations}
\label{eq:compressibleNSFirstOrder}
\begin{align}
	\frac{1}{\mu}\mathbb{D}-\LRp{\Grad\bfu+\LRp{\Grad\bfu}^T}+\frac{2}{3}\Div\bfu\bfI&=0\\
	\frac{Pr}{C_p\mu}\bfq+\Grad T&=0\\
	\Divxt\vecttwo{\rho\bfu}{\rho}&=f_c\\
	\Divxt\vecttwo{\rho\bfu\otimes\bfu+\rho RT\bfI-\mathbb{D}}{\rho\bfu}&=\bff_m\\
	\Divxt\vecttwo{\rho\bfu\LRp{C_v T+\frac{1}{2}\bfu\cdot\bfu}+\bfu\rho RT+\bfq-\bfu\cdot\mathbb{D}}{\rho\LRp{C_v T+\frac{1}{2}\bfu\cdot\bfu}}&=f_e\,,
\end{align}
\end{subequations}
where our solution variables are $\rho$, $\bfu$, $T$, $\mathbb{D}$, and $\bfq$.

%   /$$$$$$$                      /$$                        /$$     /$$                    
%  | $$__  $$                    |__/                       | $$    |__/                    
%  | $$  \ $$  /$$$$$$   /$$$$$$  /$$ /$$    /$$  /$$$$$$  /$$$$$$   /$$  /$$$$$$  /$$$$$$$ 
%  | $$  | $$ /$$__  $$ /$$__  $$| $$|  $$  /$$/ |____  $$|_  $$_/  | $$ /$$__  $$| $$__  $$
%  | $$  | $$| $$$$$$$$| $$  \__/| $$ \  $$/$$/   /$$$$$$$  | $$    | $$| $$  \ $$| $$  \ $$
%  | $$  | $$| $$_____/| $$      | $$  \  $$$/   /$$__  $$  | $$ /$$| $$| $$  | $$| $$  | $$
%  | $$$$$$$/|  $$$$$$$| $$      | $$   \  $/   |  $$$$$$$  |  $$$$/| $$|  $$$$$$/| $$  | $$
%  |_______/  \_______/|__/      |__/    \_/     \_______/   \___/  |__/ \______/ |__/  |__/
%                                                                                           
%                                                                                           
%              
\subsection{Derivation of Space-Time DPG Formulation}
We start with \eqref{eq:compressibleNSFirstOrder} and multiply by test functions $\mathbb{S}$ (symmetric tensor), $\bftau$, $v_c$, $\bfv_m$, $v_e$, 
then integrate by parts over each space-time element $K$:
\begin{subequations}
\label{eq:compressibleNSBF}
\begin{align}
	\LRp{\frac{1}{\mu}\mathbb{D},\mathbb{S}}+\LRp{2\bfu,\Div\mathbb{S}}-\LRp{\frac{2}{3}\bfu,\Grad\trace{\mathbb{S}}}
	-\LRa{\frac{4}{3}\hat\bfu,\mathbb{S}\bfn_x}&=0\\
	\LRp{\frac{Pr}{C_p\mu}\bfq,\bftau}-\LRp{T,\Div\bftau}+\LRa{\hat T,\tau_n}&=0\\
	-\LRp{\vecttwo{\rho\bfu}{\rho},\Gradxt v_c}+\LRa{\hat t_c,v_c}&=\LRp{f_c,v_c}\\
	-\LRp{\vecttwo{\rho\bfu\otimes\bfu+\rho RT\bfI-\mathbb{D}}{\rho\bfu},\Gradxt\bfv_m}+\LRa{\hat\bft_m,\bfv_m}&=\LRp{\bff_m,\bfv_m}\\
	-\LRp{\vecttwo{\rho\bfu\LRp{C_v T+\frac{1}{2}\bfu\cdot\bfu}+\bfu\rho RT+\bfq-\bfu\cdot\mathbb{D}}{\rho\LRp{C_v T+\frac{1}{2}\bfu\cdot\bfu}},\Gradxt v_e}
	+\LRa{\hat t_e,v_e}&=\LRp{f_e,v_e}\,,
\end{align}
\end{subequations}
where 
\begin{equation*}
\begin{aligned}
\hat\bfu&=\trace(\bfu)\\
\hat T&=\trace(T)\\
\hat t_c&=\trace\LRp{\rho\bfu}\cdot\bfn_x
+\trace\LRp{\rho}n_t\\
\hat\bft_m&=\trace\LRp{\rho\bfu\otimes\bfu+\rho RT\bfI-\mathbb{D}}\cdot\bfn_x
+\trace\LRp{\rho\bfu} n_t\\
\hat t_e&=\trace\LRp{\rho\bfu\LRp{C_v T+\frac{1}{2}\bfu\cdot\bfu}+\bfu\rho RT+\bfq-\bfu\cdot\mathbb{D}}\cdot\bfn_x
+\trace\LRp{\rho\LRp{C_v T+\frac{1}{2}\bfu\cdot\bfu}}n_t\,.
\end{aligned}
\end{equation*}
Note that integrating $\mathbb{S}$ against the symmetric gradient only picks up the symmetric part.
This is a much more complicated system of equations than we had for the space-time heat equation, but the situation has many similarities.
Test function $\bftau\in\HdivK$ where the divergence is taken only over spatial dimensions, $v_c,v_e\in\HOneK$, and $\bfv_m\in\HOneVecK$.
These are all familiar spaces from our work with the heat equation.
Unfortunately, $\mathbb{S}$ has some weird requirements: each $d\times d$ components must be at least in $L^2(K)$, $\Div\mathbb{S}\in\LVecK$, and
$\Grad\trace{\mathbb{S}}\in\LVecK$.
In practice, we will probably just seek each component in $\HOneK$.

%   /$$       /$$                                         /$$                       /$$     /$$                    
%  | $$      |__/                                        |__/                      | $$    |__/                    
%  | $$       /$$ /$$$$$$$   /$$$$$$   /$$$$$$   /$$$$$$  /$$ /$$$$$$$$  /$$$$$$  /$$$$$$   /$$  /$$$$$$  /$$$$$$$ 
%  | $$      | $$| $$__  $$ /$$__  $$ |____  $$ /$$__  $$| $$|____ /$$/ |____  $$|_  $$_/  | $$ /$$__  $$| $$__  $$
%  | $$      | $$| $$  \ $$| $$$$$$$$  /$$$$$$$| $$  \__/| $$   /$$$$/   /$$$$$$$  | $$    | $$| $$  \ $$| $$  \ $$
%  | $$      | $$| $$  | $$| $$_____/ /$$__  $$| $$      | $$  /$$__/   /$$__  $$  | $$ /$$| $$| $$  | $$| $$  | $$
%  | $$$$$$$$| $$| $$  | $$|  $$$$$$$|  $$$$$$$| $$      | $$ /$$$$$$$$|  $$$$$$$  |  $$$$/| $$|  $$$$$$/| $$  | $$
%  |________/|__/|__/  |__/ \_______/ \_______/|__/      |__/|________/ \_______/   \___/  |__/ \______/ |__/  |__/
%                                                                                                                  
%                                                                                                                  
%   
\subsubsection{Linearization}
We follow a standard residual-Jacobian linearization procedure coupled with a Gauss-Newton solve.
Let $U=\LRc{\rho,\bfu,T,\mathbb{D},\bfq,\hat\bfu,\hat e,\hat t_c,\hat\bft_m,\hat t_e}$ be a group solution variable which we can decompose into two parts:
$U:=\tilde U+\Delta U$, where
$\tilde U = \LRc{\tilde\rho,\tilde\bfu,\tilde T,\tilde{\mathbb{D}},\bs0,\bs0,0,0,\bs0,0}$ is the previous iteration approximation, 
and $\Delta U=\LRc{\Delta\rho,\Delta\bfu,\Delta T,\Delta\mathbb{D},\bfq,\hat\bfu,\hat e,\hat t_c,\hat\bft_m,\hat t_e}$ is the update.
Note that $\tilde U$ only contains terms which participate in nonlinearities in \eqref{eq:compressibleNSBF} 
while $\Delta U$ contains the full linear terms and the updates to the nonlinear terms.
Also, we drop the $\Delta$ and $\tilde\cdot$ notation for linear terms.
Define residual $R(U)$ as the left hand side of \eqref{eq:compressibleNSBF} minus the right hand side.
% Let $\tilde U$ be an approximate solution for the minimization of the residual. 
% We wish to solve for an increment $\Delta U$ such that $U=\tilde U+\Delta U$ is a better approximation of the true solution.
Approximating $R(U)=0$ by $R(\tilde U)+R'(\tilde U)\Delta U=0$, where $R'(\tilde U)$ is the Jacobian of $R$ evaluated at $\tilde U$, we get a linear system:
\begin{equation}
	R'(\tilde U)\Delta U=-R(\tilde U)\,.
\end{equation}
This is an instance of a Gauss-Newton nonlinear solve.
We only need to define our Jacobian and residual for each component of \eqref{eq:compressibleNSBF}. 
The Jacobian of our compressible Navier-Stokes system, $R'(\tilde U)\Delta U$ is
\begin{equation}
\label{eq:compressibleJacobian}
\begin{aligned}
	&\LRp{\frac{1}{\mu}\Delta\mathbb{D},\mathbb{S}}+\LRp{2\Delta\bfu,\Div\mathbb{S}}-\LRp{\frac{2}{3}\Delta\bfu,\Grad\trace{\mathbb{S}}}
	-\LRa{\frac{4}{3}\hat\bfu,\mathbb{S}\bfn_x}\\
	%
	&+\LRp{\frac{Pr}{C_p\mu}\bfq,\bftau}-\LRp{\Delta T,\Div\bftau}+\LRa{\hat T,\tau_n}\\
	%
	&-\LRp{\vecttwo{\Delta\rho\tilde\bfu+\tilde\rho\Delta\bfu}
	{\Delta\rho},\Gradxt v_c}
	+\LRa{\hat t_c,v_c}\\
	%
	&-\LRp{\vecttwo{\Delta\rho\tilde\bfu\otimes\tilde\bfu+\tilde\rho\Delta\bfu\otimes\tilde\bfu+\tilde\rho\tilde\bfu\otimes\Delta\bfu
	+\LRp{\Delta\rho R\tilde T+\tilde\rho R\Delta T}\bfI-\Delta\mathbb{D}}
	{\Delta\rho\tilde\bfu+\tilde\rho\Delta\bfu},\Gradxt\bfv_m}
	+\LRa{\hat\bft_m,\bfv_m}\\
	%
	&-\LRp{\vectthree{[C_v\Delta\rho\tilde T\tilde\bfu+C_v\tilde\rho\Delta T\tilde\bfu+C_v\tilde\rho\tilde T\Delta\bfu
	+\frac{1}{2}\LRp{\Delta\rho\tilde\bfu\cdot\tilde\bfu\tilde\bfu+\tilde\rho\Delta\bfu\cdot\tilde\bfu\tilde\bfu
	+\tilde\rho\tilde\bfu\cdot\Delta\bfu\tilde\bfu+\tilde\rho\tilde\bfu\cdot\tilde\bfu\Delta\bfu}}
	{+R\LRp{\Delta\rho\tilde T\tilde\bfu+\tilde\rho\Delta T\tilde\bfu+\tilde\rho\tilde T\Delta\bfu}
	+\bfq-\Delta\bfu\cdot\tilde{\mathbb{D}}-\tilde\bfu\cdot\Delta\mathbb{D}]}
	{C_v\Delta\rho\tilde T+C_v\tilde\rho\Delta T
	+\frac{1}{2}\LRp{\Delta\rho\tilde\bfu\cdot\tilde\bfu+\tilde\rho\Delta\bfu\cdot\tilde\bfu+\tilde\rho\tilde\bfu\cdot\Delta\bfu}},\Gradxt v_e}\\
	&+\LRa{\hat t_e,v_e}\,.
\end{aligned}
\end{equation}
The residual, $R(\tilde U)$, is then
\begin{equation}
\begin{aligned}
	&\LRp{\frac{1}{\mu}\tilde{\mathbb{D}},\mathbb{S}}+\LRp{2\tilde\bfu,\Div\mathbb{S}}-\LRp{\frac{2}{3}\tilde\bfu,\Grad\trace{\mathbb{S}}}\\
	&-\LRp{\tilde T,\Div\bftau}\\
	&-\LRp{\vecttwo{\tilde\rho\tilde\bfu}{\tilde\rho},\Gradxt v_c}-\LRp{f_c,v_c}\\
	&-\LRp{\vecttwo{\tilde\rho\tilde\bfu\otimes\tilde\bfu+\tilde\rho R\tilde T\bfI-\tilde{\mathbb{D}}}{\tilde\rho\tilde\bfu},
	\Gradxt\bfv_m}-\LRp{\bff_m,\bfv_m}\\
	&-\LRp{\vecttwo{\tilde\rho\tilde\bfu\LRp{C_v\tilde T+\frac{1}{2}\tilde\bfu\cdot\tilde\bfu}+\tilde\bfu\tilde\rho R\tilde T
	-\tilde\bfu\cdot\tilde{\mathbb{D}}}{\tilde\rho\LRp{C_v\tilde T+\frac{1}{2}\tilde\bfu\cdot\tilde\bfu}},
	\Gradxt v_e}-\LRp{f_e,v_e}\,.
\end{aligned}
\end{equation}

%   /$$$$$$$$                       /$$           /$$   /$$                                  
%  |__  $$__/                      | $$          | $$$ | $$                                  
%     | $$     /$$$$$$   /$$$$$$$ /$$$$$$        | $$$$| $$  /$$$$$$   /$$$$$$  /$$$$$$/$$$$ 
%     | $$    /$$__  $$ /$$_____/|_  $$_/        | $$ $$ $$ /$$__  $$ /$$__  $$| $$_  $$_  $$
%     | $$   | $$$$$$$$|  $$$$$$   | $$          | $$  $$$$| $$  \ $$| $$  \__/| $$ \ $$ \ $$
%     | $$   | $$_____/ \____  $$  | $$ /$$      | $$\  $$$| $$  | $$| $$      | $$ | $$ | $$
%     | $$   |  $$$$$$$ /$$$$$$$/  |  $$$$/      | $$ \  $$|  $$$$$$/| $$      | $$ | $$ | $$
%     |__/    \_______/|_______/    \___/        |__/  \__/ \______/ |__/      |__/ |__/ |__/
%                                                                                            
%                                                                                            
%      
\subsubsection{Test Norm}
The most obvious first choice for test norm in the local solve is the graph norm, which comes from the 
problem adjoint.
We start by grouping terms in \eqref{eq:compressibleJacobian} by trial variable to get
\begin{equation}
\begin{aligned}
&\LRp{\Delta\mathbb{D},\frac{1}{\mu}\mathbb{S}+\Grad\bfv_m+\Grad v_e\otimes\tilde\bfu}\\
+&\LRp{\bfq,\frac{Pr}{C_p\mu}\bftau-\Grad v_e}\\
+&\left(\Delta\rho,-\tilde\bfu\cdot\Grad v_c-\frac{\partial v_c}{\partial t}
-\tilde\bfu\otimes\tilde\bfu:\Grad\bfv_m
-R\tilde T\Div\bfv_m-\tilde\bfu\cdot\frac{\partial\bfv_m}{\partial t}\right.\\
&\left.-C_v\tilde T\tilde\bfu\cdot\Grad v_e-\frac{1}{2}\tilde\bfu\cdot\tilde\bfu\tilde\bfu\cdot\Grad v_e
-R\tilde T\tilde\bfu\Grad v_e
-C_v\tilde T\frac{\partial v_e}{\partial t}-\frac{1}{2}\tilde\bfu\cdot\tilde\bfu\frac{\partial v_e}{\partial t}
\right)\\
+&\left(\Delta\bfu,2\Div\mathbb{S}-\frac{2}{3}\Grad\trace{\mathbb{S}}
-\tilde\rho\Grad v_c
-\tilde\rho\tilde\bfu\cdot\Grad\bfv_m-\tilde\rho\Grad\bfv_m\cdot\tilde\bfu
-\tilde\rho\frac{\partial\bfv_m}{\partial t}
-C_v\tilde\rho\tilde T\Grad v_e\right.\\
&\left.
-\frac{1}{2}\tilde\rho\tilde\bfu\cdot\tilde\bfu\Grad v_e
-\frac{1}{2}\tilde\rho\tilde\bfu\cdot\Grad v_e\tilde\bfu
-\frac{1}{2}\tilde\rho\Grad v_e\cdot\tilde\bfu\tilde\bfu
-R\tilde\rho\tilde T_\Grad v_e
+\tilde{\mathbb{D}}\cdot\Grad v_e
-\frac{1}{2}\tilde\rho\tilde\bfu\frac{\partial v_e}{\partial t}
-\frac{1}{2}\tilde\rho\tilde\bfu\frac{\partial v_e}{\partial t}
\right)\\
+&\left(\Delta T,-\Div\bftau
-R\tilde\rho\Div\bfv_m
-C_v\tilde\rho\tilde\bfu\Grad v_e
-R\tilde\rho\tilde\bfu\Grad v_e
-C_v\tilde\rho\frac{\partial v_e}{\partial t}
\right)\\
+&\LRp{\hat\bfu,-\frac{4}{3}\mathbb{S}\bfn_x}\\
+&\LRp{\hat T,\tau_n}\\
+&\LRp{\hat t_c,v_c}\\
+&\LRp{\hat\bft_m,\bfv_m}\\
+&\LRp{\hat t_e,v_e}\,.
\end{aligned}	
\end{equation}

Then the graph norm would be defined by
\begin{equation}
\begin{aligned}
&\norm{\frac{1}{\mu}\mathbb{S}+\Grad\bfv_m+\Grad v_e\otimes\tilde\bfu}^2\\
+&\norm{\frac{Pr}{C_p\mu}\bftau-\Grad v_e}^2\\
+&\left\|
-\tilde\bfu\cdot\Grad v_c-\frac{\partial v_c}{\partial t}-\tilde\bfu\otimes\tilde\bfu:\Grad\bfv_m
-R\tilde T\Div\bfv_m-\tilde\bfu\cdot\frac{\partial\bfv_m}{\partial t}\right.\\
&\left.-C_v\tilde T\tilde\bfu\cdot\Grad v_e-\frac{1}{2}\tilde\bfu\cdot\tilde\bfu\tilde\bfu\cdot\Grad v_e
-R\tilde T\tilde\bfu\Grad v_e
-C_v\tilde T\frac{\partial v_e}{\partial t}-\frac{1}{2}\tilde\bfu\cdot\tilde\bfu\frac{\partial v_e}{\partial t}
\right\|^2\\
+&\left\|2\Div\mathbb{S}-\frac{2}{3}\Grad\trace{\mathbb{S}}
-\tilde\rho\Grad v_c
-\tilde\rho\tilde\bfu\cdot\Grad\bfv_m-\tilde\rho\Grad\bfv_m\cdot\tilde\bfu
-\tilde\rho\frac{\partial\bfv_m}{\partial t}
-C_v\tilde\rho\tilde T\Grad v_e\right.\\
&\left.
-\frac{1}{2}\tilde\rho\tilde\bfu\cdot\tilde\bfu\Grad v_e
-\frac{1}{2}\tilde\rho\tilde\bfu\cdot\Grad v_e\tilde\bfu
-\frac{1}{2}\tilde\rho\Grad v_e\cdot\tilde\bfu\tilde\bfu
-R\tilde\rho\tilde T\Grad v_e
+\tilde{\mathbb{D}}\cdot\Grad v_e
-\frac{1}{2}\tilde\rho\tilde\bfu\frac{\partial v_e}{\partial t}
-\frac{1}{2}\tilde\rho\tilde\bfu\frac{\partial v_e}{\partial t}
\right\|^2\\
+&\left\|-\Div\bftau
-R\tilde\rho\Div\bfv_m
-C_v\tilde\rho\tilde\bfu\Grad v_e
-R\tilde\rho\tilde\bfu\Grad v_e
-C_v\tilde\rho\frac{\partial v_e}{\partial t}
\right\|^2\\
+&\alpha_c\norm{v_c}^2
+\alpha_m\norm{\bfv_m}^2
+\alpha_e\norm{v_e}^2
+\alpha_s\norm{\mathbb{S}}^2
+\alpha_f\norm{\bftau}^2
\,,
\end{aligned}
\end{equation}
where $\alpha_c$, $\alpha_m$, $\alpha_e$, $\alpha_s$, and $\alpha_f$ are scaling constants, usually one.

Unfortunately, the graph norm develops unresolvable internal boundary layers in the optimal test functions leading to a non-robust
local solve for steady convection-diffusion or Navier-Stokes,
% is known to not be robust for steady convection-diffusion or Navier-Stokes,
and we saw that non-robustness manifest when we tried to use this norm for transient simulations as well.
For steady state DPG, we developed a robust test norm for convection-diffusion and drew analogies to create 
a robust norm for Navier-Stokes.
A similar analysis for transient convection-diffusion has not been done (this is part of the proposed work), 
so we are on shakier footing developing a robust norm for transient Navier-Stokes.
Nevertheless, we can make some guesses about how to modify the test norm in order to obtain some preliminary results.
The graph norm has proven to be sufficient for simulations of pure convection. 
So an obvious first guess might be to take the graph norm on the convective quantities and decouple the viscous terms.
Indeed, this selection proved to be more robust for the test problems considered in the next section.
This modified graph norm is then:
\begin{equation}
\begin{aligned}
&\norm{\Grad\bfv_m+\Grad v_e\otimes\tilde\bfu}^2\\
+&\norm{-\Grad v_e}^2\\
+&\left\|
-\tilde\bfu\cdot\Grad v_c-\frac{\partial v_c}{\partial t}-\tilde\bfu\otimes\tilde\bfu:\Grad\bfv_m
-R\tilde T\Div\bfv_m-\tilde\bfu\cdot\frac{\partial\bfv_m}{\partial t}\right.\\
&\left.-C_v\tilde T\tilde\bfu\cdot\Grad v_e-\frac{1}{2}\tilde\bfu\cdot\tilde\bfu\tilde\bfu\cdot\Grad v_e
-R\tilde T\tilde\bfu\Grad v_e
-C_v\tilde T\frac{\partial v_e}{\partial t}-\frac{1}{2}\tilde\bfu\cdot\tilde\bfu\frac{\partial v_e}{\partial t}
\right\|^2\\
+&\left\|
-\tilde\rho\Grad v_c
-\tilde\rho\tilde\bfu\cdot\Grad\bfv_m-\tilde\rho\Grad\bfv_m\cdot\tilde\bfu
-\tilde\rho\frac{\partial\bfv_m}{\partial t}
-C_v\tilde\rho\tilde T\Grad v_e\right.\\
&\left.
-\frac{1}{2}\tilde\rho\tilde\bfu\cdot\tilde\bfu\Grad v_e
-\frac{1}{2}\tilde\rho\tilde\bfu\cdot\Grad v_e\tilde\bfu
-\frac{1}{2}\tilde\rho\Grad v_e\cdot\tilde\bfu\tilde\bfu
-R\tilde\rho\tilde T\Grad v_e
+\tilde{\mathbb{D}}\cdot\Grad v_e
-\frac{1}{2}\tilde\rho\tilde\bfu\frac{\partial v_e}{\partial t}
-\frac{1}{2}\tilde\rho\tilde\bfu\frac{\partial v_e}{\partial t}
\right\|^2\\
+&\left\|
-R\tilde\rho\Div\bfv_m
-C_v\tilde\rho\tilde\bfu\Grad v_e
-R\tilde\rho\tilde\bfu\Grad v_e
-C_v\tilde\rho\frac{\partial v_e}{\partial t}
\right\|^2\\
+&\frac{1}{\mu}\norm{\mathbb{S}}^2
+\norm{2\Div\mathbb{S}-\frac{2}{3}\Grad\trace{\mathbb{S}}}^2
+\frac{Pr}{c_p\mu}\norm{\bftau}^2
+\norm{\Div\bftau}^2\\
+&\norm{v_c}^2
+\norm{\bfv_m}^2
+\norm{v_e}^2
\,.
\end{aligned}
\end{equation}
From a number of numerical tests, it appears that this norm is not completely robust, but it does seem to perform somewhat better
than the standard graph norm.

%   /$$   /$$                                             /$$                     /$$
%  | $$$ | $$                                            |__/                    | $$
%  | $$$$| $$ /$$   /$$ /$$$$$$/$$$$   /$$$$$$   /$$$$$$  /$$  /$$$$$$$  /$$$$$$ | $$
%  | $$ $$ $$| $$  | $$| $$_  $$_  $$ /$$__  $$ /$$__  $$| $$ /$$_____/ |____  $$| $$
%  | $$  $$$$| $$  | $$| $$ \ $$ \ $$| $$$$$$$$| $$  \__/| $$| $$        /$$$$$$$| $$
%  | $$\  $$$| $$  | $$| $$ | $$ | $$| $$_____/| $$      | $$| $$       /$$__  $$| $$
%  | $$ \  $$|  $$$$$$/| $$ | $$ | $$|  $$$$$$$| $$      | $$|  $$$$$$$|  $$$$$$$| $$
%  |__/  \__/ \______/ |__/ |__/ |__/ \_______/|__/      |__/ \_______/ \_______/|__/
%                                                                                    
%                                                                                    
% 
\subsection{Numerical Results}
We consider two 1D test problems as verification.
The Sod shock tube and Noh implosion both have analytical solutions derived based on an inviscid flow assumption (Euler's equations).
However, in the absence of viscosity, Euler's equations can have multiple solutions, and most numerical methods introduce a certain amount of
artificial viscosity in order to select a unique solution.
Most schemes also require the artificial viscosity to scale in some sense with mesh size so that they can effectively handle shocks.
We run our simulations without any artificial viscosity, but in order to get a well-posed problem, we do introduce a small amount of physical
viscosity: $\mu=10^{-5}$ for Sod and $\mu=10^{-3}$ for Noh.
Essentially we are just simulating low viscosity Navier-Stokes as a stand-in for the unsolvable pure Euler.
We mentioned previously that the test norm we are using is not entirely robust, and in fact these viscosity values were on the lower end of what we
could simulate with this preliminary norm.
Following the same polynomial representation as we did in the section on local conservation (Section \ref{sec:lcNumerics}), the field variables were represented with quadratics.

\subsubsection{Sod Shock Tube}
The Sod shock tube problem was developed by Gary Sod in 1978\cite{Sod1978}, and has proven to be a popular problem for verification 
of compressible Navier-Stokes and Euler solvers.
It serves to verify that a numerical method can effectively handle a rarefaction wave, material discontinuity, and shock wave
all in one domain.
The domain of interest is a shock tube of length 1 with a material interface in the middle. The material on the left has initial conditions of 
$(\rho_L,p_L,u_L)=(1,1,0)$ while the material on the right has $(\rho_R,p_R,u_R)=(0.125,0.1,0)$; both materials have $\gamma=1.4$. 
The $t=0$ the interface between the materials is broken, 
and shock wave propagates into the right material, while a rarefaction wave moves left. The analytical solution is self-similar, but it is common to take
$t=0.2$ as a final time.
At this time the shock wave and rarefaction waves have not hit the boundaries, 
so it is sufficient to set boundary conditions corresponding to the initial conditions.
In our case, we set $\hat t_c=\hat t_m=\hat t_e=0$ on the left and right boundaries, while the fluxes are set equal to the discontinuous
initial conditions on the $t=0$ boundary. 
No boundary condition is required on the $t=0.2$ boundary since the equations are hyperbolic in time.
We solve this with one continuous time slab starting with only 4 space-time elements.

The results are plotted in Figure~\ref{fig:sod} for three different refinement levels: the initial coarse mesh, 7 adaptive refinements, and 14 refinements.
The coarsest mesh is obviously not sufficient to resolve the features of the flow, but it is at least somewhat representative of the exact solution.
We see significant overshoots and undershoots as we start to pick up on the shock, but these die away as we resolve to the viscous length scale.

\begin{figure}[p]
\centering
\begin{subfigure}[c]{0.3\textwidth}
\centering
\includegraphics[width=\textwidth]{SpaceTimeCNS/Sod1e-5/den1.pdf}
\caption{Density on initial mesh}
\label{fig:sod_den0}
\end{subfigure}
\begin{subfigure}[c]{0.3\textwidth}
\centering
\includegraphics[width=\textwidth]{SpaceTimeCNS/Sod1e-5/den8.pdf}
\caption{After 7 refinements}
\label{fig:sod_den7}
\end{subfigure}
\begin{subfigure}[c]{0.3\textwidth}
\centering
\includegraphics[width=\textwidth]{SpaceTimeCNS/Sod1e-5/den15.pdf}
\caption{After 14 refinements}
\label{fig:sod_den14}
\end{subfigure}
\begin{subfigure}[c]{0.3\textwidth}
\centering
\includegraphics[width=\textwidth]{SpaceTimeCNS/Sod1e-5/vel1.pdf}
\caption{Velocity on initial mesh}
\label{fig:sod_vel0}
\end{subfigure}
\begin{subfigure}[c]{0.3\textwidth}
\centering
\includegraphics[width=\textwidth]{SpaceTimeCNS/Sod1e-5/vel8.pdf}
\caption{After 7 refinements}
\label{fig:sod_vel7}
\end{subfigure}
\begin{subfigure}[c]{0.3\textwidth}
\centering
\includegraphics[width=\textwidth]{SpaceTimeCNS/Sod1e-5/vel15.pdf}
\caption{After 14 refinements}
\label{fig:sod_vel14}
\end{subfigure}
\begin{subfigure}[c]{0.3\textwidth}
\centering
\includegraphics[width=\textwidth]{SpaceTimeCNS/Sod1e-5/pres1.pdf}
\caption{Pressure on initial mesh}
\label{fig:sod_pres0}
\end{subfigure}
\begin{subfigure}[c]{0.3\textwidth}
\centering
\includegraphics[width=\textwidth]{SpaceTimeCNS/Sod1e-5/pres8.pdf}
\caption{After 7 refinements}
\label{fig:sod_pres7}
\end{subfigure}
\begin{subfigure}[c]{0.3\textwidth}
\centering
\includegraphics[width=\textwidth]{SpaceTimeCNS/Sod1e-5/pres15.pdf}
\caption{After 14 refinements}
\label{fig:sod_pres14}
\end{subfigure}
\begin{subfigure}[c]{0.45\textwidth}
\centering
\includegraphics[width=\textwidth]{SpaceTimeCNS/Sod1e-5/mesh1.png}
\caption{Density with initial mesh}
\label{fig:sod_mesh0}
\end{subfigure}
\begin{subfigure}[c]{0.45\textwidth}
\centering
\includegraphics[width=\textwidth]{SpaceTimeCNS/Sod1e-5/mesh8.png}
\caption{Density with mesh after 7 refinements}
\label{fig:sod_mesh7}
\end{subfigure}
\begin{subfigure}[c]{0.9\textwidth}
\centering
\includegraphics[width=\textwidth]{SpaceTimeCNS/Sod1e-5/mesh15.png}
\caption{Density with mesh after 14 refinements}
\label{fig:sod_mesh14}
\end{subfigure}
\caption{Sod problem with final time $t=0.2$}
\label{fig:sod}
\end{figure}

\subsubsection{Noh Implosion}
The Noh implosion problem\cite{Noh1987} is another standard test for Euler solvers.
The initial conditions are of an ideal gas with $\gamma=5/3$, zero pressure, uniform initial density of 1, 
and uniform velocity toward the center of the domain.
An infinitely strong shock propagates outward at a speed of 1/3.
For 1D flow, the post shock density jumps to 4.
We run this problem to a final time of $t=1.0$.
The longer time nature of this problem recommended the use of multiple time slabs rather than a single solve like the previous problem.
We run with four time slabs of thickness 0.25 each with 4 initial space-time elements.
We run the first slab to 8 adaptive refinements and set the initial conditions on the next slab to the refined solution on the previous slab.

Each of the slabs are put together into one long time solution in Figure~\ref{fig:noh}. 
Again we plot the solution on the initial mesh, a halfway resolved mesh, and a final mesh after 8 refinement steps.
We get some very odd behavior around the shock on the middle mesh, but this goes away by the final mesh.
We see the same behavior with overshoots and undershoots that we saw with the Sod problem. 

\begin{figure}[p]
\centering
\begin{subfigure}[c]{0.3\textwidth}
\centering
\includegraphics[width=\textwidth]{SpaceTimeCNS/Noh1e-3/den1.pdf}
\caption{Density on initial mesh}
\label{fig:noh_den0}
\end{subfigure}
\begin{subfigure}[c]{0.3\textwidth}
\centering
\includegraphics[width=\textwidth]{SpaceTimeCNS/Noh1e-3/den5.pdf}
\caption{After 4 refinements}
\label{fig:noh_den4}
\end{subfigure}
\begin{subfigure}[c]{0.3\textwidth}
\centering
\includegraphics[width=\textwidth]{SpaceTimeCNS/Noh1e-3/den9.pdf}
\caption{After 8 refinements}
\label{fig:noh_den8}
\end{subfigure}
\begin{subfigure}[c]{0.45\textwidth}
\centering
\includegraphics[width=0.65\textwidth]{SpaceTimeCNS/Noh1e-3/mesh1.png}
\caption{Density with initial mesh}
\label{fig:noh_mesh0}
\end{subfigure}
\begin{subfigure}[c]{0.45\textwidth}
\centering
\includegraphics[width=0.65\textwidth]{SpaceTimeCNS/Noh1e-3/mesh5.png}
\caption{Density with mesh after 4 refinements}
\label{fig:noh_mesh4}
\end{subfigure}
\begin{subfigure}[c]{0.9\textwidth}
\centering
\includegraphics[width=0.65\textwidth]{SpaceTimeCNS/Noh1e-3/mesh9.png}
\caption{Density with mesh after 8 refinements}
\label{fig:noh_mesh8}
\end{subfigure}
\caption{Noh problem with final time $t=1.0$}
\label{fig:noh}
\end{figure}

\bibliographystyle{plain}
\bibliography{../Papers}

\end{document}


