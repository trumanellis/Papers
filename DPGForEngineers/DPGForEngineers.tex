\documentclass{article}
\usepackage{amsmath, amssymb, amsthm, mathtools}
\usepackage{graphicx}
\usepackage{cancel}
\usepackage{color}
\usepackage{caption}
\usepackage{subcaption}
%\usepackage{subfigure}
\usepackage{pgf,tikz}
\usetikzlibrary{arrows}

\captionsetup{compatibility=false}

\newcommand{\bs}[1]{\boldsymbol{#1}}
\newcommand{\norm}[1]{\left\| #1 \right\|}
\newcommand{\snorm}[1]{\left| #1 \right|}
\newcommand{\LRp}[1]{\left( #1 \right)}
\newcommand{\LRs}[1]{\left[ #1 \right]}
\newcommand{\LRc}[1]{\left\{ #1 \right\}}
\newcommand{\LRa}[1]{\left\langle #1 \right\rangle}
\newcommand{\LRb}[1]{\left| #1 \right|}
\newcommand{\Grad}{\ensuremath{\nabla}}
\newcommand{\Gradxt}{\ensuremath{\nabla_{xt}}}
\newcommand{\Div}{\ensuremath{\nabla\cdot}}
\newcommand{\Divxt}{\ensuremath{\nabla_{xt}\cdot}}
\newcommand{\Curl}{\ensuremath{\nabla\times}}
\newcommand{\bfH}{\mbox{\boldmath $H$}}
\newcommand{\bfsigma}{\boldsymbol\sigma}
\newcommand{\bfvarsigma}{\boldsymbol\varsigma}
\newcommand{\bftau}{\boldsymbol\tau}
\newcommand{\bfbeta}{\boldsymbol\beta}
\newcommand{\bflambda}{\boldsymbol\lambda}
\newcommand{\bfpsi}{\boldsymbol\psi}
\newcommand{\bfu}{\boldsymbol u}
\newcommand{\bfv}{\boldsymbol v}
\newcommand{\bfV}{\boldsymbol V}
\newcommand{\bfZ}{\boldsymbol Z}
\newcommand{\bfz}{\boldsymbol z}
\newcommand{\bfW}{\boldsymbol W}
\newcommand{\bfw}{\boldsymbol w}
\newcommand{\bfm}{\boldsymbol m}
\newcommand{\bfM}{\boldsymbol M}
\newcommand{\bbM}{\mathbb{M}}
\newcommand{\bfq}{\boldsymbol q}
\newcommand{\bfU}{\boldsymbol U}
\newcommand{\bfS}{\boldsymbol S}
\newcommand{\bbS}{\mathbb{S}}
\newcommand{\bbD}{\mathbb{D}}
\newcommand{\bfK}{\boldsymbol K}
\newcommand{\bbK}{\mathbb{K}}
\newcommand{\bfn}{\boldsymbol n}
\newcommand{\bff}{\boldsymbol f}
\newcommand{\bfF}{\boldsymbol F}
\newcommand{\bbF}{\mathbb{F}}
\newcommand{\bfg}{\boldsymbol g}
\newcommand{\bfG}{\boldsymbol G}
\newcommand{\bfC}{\boldsymbol C}
\newcommand{\bft}{\boldsymbol t}
\newcommand{\bfT}{\boldsymbol T}
\newcommand{\bfI}{\boldsymbol I}
\newcommand{\bbI}{\mathbb{I}}
\newcommand{\bfx}{\boldsymbol x}
\newcommand{\uh}{\widehat{u}}
\newcommand{\fnh}{\widehat{f}_n}
\newcommand{\LQ}{L^2\LRp{Q}}
\newcommand{\LK}{L^2\LRp{K}}
\newcommand{\LVecK}{\mathbf{L}^2\LRp{K}}
\newcommand{\LVecQ}{\mathbf{L}^2\LRp{Q}}
\newcommand{\HdivK}{\bfH(\text{div},K)}
\newcommand{\HdivOmega}{\bfH(\text{div},\Omega)}
% \newcommand{\HdivOmegaLT}{\bfH(\text{div},\Omega)\times L^2([0,T])}
\newcommand{\HdivQ}{\bfH(\text{div}_{xt},Q)}
\newcommand{\HOneK}{H^{1}(K)}
\newcommand{\HOneVecK}{\bfH^{1}(K)}
\newcommand{\HOneQ}{H^{1}(Q)}
\newcommand{\HOneOmegah}{H^{-1}(\Omega_h)}
\newcommand{\HdivOmegah}{\bfH(\text{div},\Omega_h)}
\newcommand{\vdeltau}{v_{\delta\bs u_h}}
\newcommand{\taudeltau}{\bftau_{\delta\bs u_h}}
\newcommand{\ip}[1]{\left\langle #1 \right\rangle}
\newcommand{\pd}[2]{\frac{\partial#1}{\partial#2}}
\newcommand{\pt}[1]{\frac{\partial#1}{\partial t}}
\newcommand{\ppd}[2]{\frac{\partial^2#1}{\partial#2^2}}
\newcommand{\pdd}[3]{\frac{\partial^2#1}{\partial#2\partial#3}}
\newcommand{\der}[2]{\frac{\mathrm{d}#1}{\mathrm{d}#2}}
\newcommand{\Oh}{\Omega_h}
\newcommand{\jump}[1] {\ensuremath{\LRs{\![#1]\!}}}
\newcommand{\Gh}{\Gamma_h}
\newcommand{\mcU}{\mathcal{U}}
\newcommand{\mcUh}{\hat{\mathcal{U}}}
\newcommand{\LOmega}{L^2\LRp{\Omega_h}}

\newcommand{\eqnref}[1]{\eqref{eq:#1}}

\DeclareMathOperator*{\argmin}{arg\,min}
\DeclareMathOperator*{\trace}{tr}

\def\arrtwo#1#2#3#4{\left[
\begin{array}{cc}
#1\; & #2\\
#3\; & #4\\
\end{array}
\right]}
\def\arrthree#1#2#3#4#5#6#7#8#9{\left[
\begin{array}{ccc}
#1\; & #2\; & #3\\
#4\; & #5\; & #6\\
#7\; & #8\; & #9\\
\end{array}
\right]}
\def\arrthreeone#1#2#3{\left[
\begin{array}{ccc}
#1\; & #2\; & #3\\
\end{array}
\right]}
\def\vecttwo#1#2{\left(
\begin{array}{c}
#1\\
#2\\
\end{array}
\right)}
\def\svecttwo#1#2{\left[
\begin{array}{c}
#1\\
#2\\
\end{array}
\right]}
\def\vectthree#1#2#3{\left(
\begin{array}{c}
#1\\
#2\\
#3\\
\end{array}
\right)}
\def\svectthree#1#2#3{\left[
\begin{array}{c}
#1\\
#2\\
#3\\
\end{array}
\right]}

\renewcommand{\arraystretch}{1.2}

\def\etal{{\it et al.~}}

\usepackage[margin=1.25in]{geometry}

\title{DPG for Engineers}
\author{Truman E. Ellis, Jesse L. Chan, Leszek F. Demkowicz, Jay Gopalakrishnan}

\begin{document}
\maketitle

\begin{abstract}
We discuss and explain the discontinuous Petrov-Galerkin finite element method with an engineering audience in mind.
\end{abstract}

\section{Motivation and Introductory Concepts}
The discontinuous Petrov-Galerkin finite element method is a promising new framework 
for automated computing of a broad class of partial differential equations.

\subsection{A Framework for Automated Computing}
% Stability => Multiphysics
% Stability => Easy meshing

\subsection{Mathematical Fundamentals}

\subsubsection{Banach, Hilbert, and Sobolev Spaces}
% Basic explanation with examples
Consider a bounded domain, $\Omega\in\mathbb{R}^d$ and a vector space of functions defined on $\Omega$.
A Banach space is a complete normed vector space.
An example of which is $L_1(\Omega)$ which is the set of all functions defined on $\Omega$ such that the $L_1$ norm
$\norm{u}_{L_1(\Omega)}:=\int_\Omega|u|$ is finite.
A Hilbert space is a Banach space where the norm is defined through an inner product.
An example of which is $L_2(\Omega)$ for which the norm is
$\norm{u}_{L_2(\Omega)}:=\LRp{u,u}_{L_2(\Omega)}^{\frac{1}{2}}:=\LRp{\int_\Omega u^2}^{\frac{1}{2}}$.
Finally, a Sobolev space is a Hilbert space where the inner product contains derivatives of the function.
A common Sobolev space is $H^1(\Omega)$ where $\norm{u}_{H^1(\Omega)}^2:=\norm{u}_{L_2(\Omega)}^2+\norm{\Grad u}_{L_2(\Omega)}^2$.

\subsubsection{Variational Formulations, Dual Spaces, and Riesz Maps}
Consider a Banach space $U$. The dual space, $U'$ is the set of all linear functionals on $U$ such that the pairing $\LRa{u,w}_{U\times U'}\in\mathbb{R}$ 
for all $u\in U$ and $w\in U'$.
Any PDE problem can abstractly be written as: find solution $u\in U$ such that
\[
Bu=l
\]
where $B:U\rightarrow V'$ maps functions to some dual space and $l\in U'$.
By the calculus of variations, this is equivalent to the problem: find solution $u\in U$ such that
\[
\LRa{Bu,v}=\LRa{l,v}
\]
for all functions $v\in V$ and $v$ is called a test function.
This defines a bilinear form $b(u,v)=f(v)$ where $u$ is often called the trial function.
Now let $U$ and $V$ be Hilbert rather than merely Banach.
The Riesz representation theorem defines a bijective, isometric mapping from every member in $V$ to a corresponding member in the dual space $V'$.
We denote this mapping $R_V:V\rightarrow V'$.
Thus, we can take our operator equation representing the strong form of our PDE
\[
Bu-l=0\quad\in V'
\]
and map it back to $V$ via the inverse Riesz map:
\[
R_V^{-1}(Bu-l)=0\quad\in V.
\]

% \subsubsection{Broken Sobolev Spaces}
% Quick definition of Sobolev spaces

\subsection{Ritz Methods and Least Squares}

\subsection{SUPG -- Or Stability Through the Appropriate Test Space}

\section{Development of a DPG Method for the Poisson Equation}
% Start with a specific example then generalize
% Alternatively do convection-diffusion or reaction-diffusion

\section{DPG for Well Posed PDEs}

\end{document}