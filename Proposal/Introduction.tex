% -*- root: Proposal.tex -*-
\documentclass[Proposal.tex]{subfiles} 
\begin{document}
\chapter{Introduction}
\section{Motivation} 

Computational science has revolutionized the engineering design process -- enabling design analysis
and optimization to be done virtually before expensive physical prototypes need to be built.
However, some fields of engineering analysis lend themselves to a computational approach much easier
than others. 
Fluid dynamics has long been one of the most challenging engineering disciplines to simulate via numerical techniques.
Aside from the inherent modeling challenges presented by fluid turbulence, many fluid flows can be characterized as singularly perturbed problems 
-- problems in which the viscosity length scale is many orders of magnitude smaller than the large scale features of the flow.
This has necessitated the need for meshes with large gradations in resolution to enable resolution of boundary layers while being computationally efficient in the free stream.
Traditionally, these meshes would be custom designed by a domain expert who could predict which parts of the domain would need more resolution than others. 
On top of this, many numerical techniques would fail to converge unless the presented initial mesh was in the ``asymptotic regime'', 
i.e. the physics could by somewhat sufficiently represented.
These requirements made mesh generation a laborious and far from automated procedure.

\subsection{A Robust Adaptive Method for CFD}
The failure of many numerical methods in the ``pre-asymptotic regime'' can be characterized mathematically as a loss of stability on coarse meshes.
The stability characteristics of a broad class of finite element methods can be analyzed according to the Lady\v{z}enskaja-Babu\v{s}ka-Brezzi condition.
Leszek Demkowicz and Jay Gopalakrishnan first proposed the discontinuous Petrov-Galerkin method in 2009\cite{DPG1} in order to address stability issues for a 
very broad class of problems. The DPG method automatically satisfies stability criteria by construction which enables DPG simulations to remain stable and 
convergent even in the pre-asymptotic regime. 
By nature, the DPG method also comes with a built-in error representation function, effectively eliminating the need for other a posteriori error estimators.
Practically, this means that a simulation could start with just the coarsest mesh necessary to represent the geometry of the solution and adaptively refine toward a resolved solution in a very automatic way.
Carried to its logical conclusion, this capability could significantly cut down on the time intensive manual mesh generation (and tweaking) that dominates a good amount of simulation and analysis time.
Where a current numerical method might falter on a poorly designed mesh, necessitating an engineer to manually enter the problem and fix the offending mesh nodes, a DPG simulation would converge on the poor mesh, mark the offending cells, refine, and continue toward a solution.

Another benefit to the enhanced stability properties of DPG is the ability to consider high order and $hp$-adaptive methods. 
Many popular numerical methods for CFD (such as the discontinuous Galerkin method) are stable for low polynomial orders, but require additional stabilizing terms for higher orders. 
Additionally, one of the longstanding issues with $hp$-adaptive techniques was that they suffered stability problems when the polynomial order rose to high. 
Polynomial order presents no issue at all to DPG methods -- allowing us to recover the high order convergence rates of high uniform $p$ methods or even the exponential convergence rates of $hp$ methods.

The biggest limitation to past explorations of the DPG method is that they were all limited to steady state problems.
Obviously this seriously limits the variety of interesting problems we could consider. 
The easiest extension of steady DPG to transient problems would be to do an implicit time stepping technique in time and use DPG for only the spatial solve at each time step.
We did indeed explore this approach, but it didn't seem to be a natural fit with the adaptive features of DPG.
Clearly the CFL condition was not binding since we were interested in implicit time integration schemes, but the CFL condition can be a guiding principle for temporal accuracy in this case.
So if we are interested in temporally accurate solutions, we are limited by the fact that our smallest mesh elements (which may be order of magnitude smaller than the largest elements) are constrained to proceed at a much smaller time step than the mesh as a whole. 
We can either restrict the whole mesh to the smallest time step, or we can attempt some sort of local time stepping.
A space-time DPG formulation presents an attractive choice as we will be able to preserve our natural adaptivity from the steady problems while extending it in time.
Thus we achieve an adaptive solution technique for transient problems in a unified framework.
The obvious downside to such an approach is that for 2D spatial problems, we now have to compute on a three dimensional mesh while a spatially 3D problem becomes four dimensional.

\subsection{Investigating a New Methodology}
Much of science is driven by curiosity, and this especially holds for computational science. 
There is inherent value in exploring new methodologies because they may hold the keys to solving new problems or old problems in a better way.
A new method may also help us to better understand existing methods. 
The variational multiscale approach to finite element analysis helped to elucidate on some of the success of the much older streamwise upwind Petrov-Galerkin method while generalizing and improving it.
The DPG method itself can be viewed as a generalization of least-squares finite elements or even of mixed methods. 

Curiosity similarly motivates the desire to explore a space-time DPG formulation for computational fluid dynamics. 
Based on our past experience with steady DPG, we anticipate space-time DPG to be a very interesting technique that could extend the automaticity of DPG in very novel ways.

\subsection{DPG + X}
DPG is admittedly, a very costly method at present. 
We have ideas about how to reduce the effective cost, but DPG may never be as fast as more traditional methods designed explicitly for DPG.
Ultimately, there is no reason why we can't combine DPG with another method to gain the benefits of both.
We could let DPG handle the initial coarse mesh and adaptively start refining toward a mesh that is sufficiently fine for another method to take over.
The other method could then use traditional \emph{a posteriori} error approximation to arrive at a fully resolved solution.
This leverages the benefits of using DPG in an automated way on coarse meshes where the cost is less significant while benefiting from the 
computational efficiency of whatever method is coupled to it.
If the other method is finite element based, this could possibly be done as simply as swapping out the test functions being used 
-- perhaps the mesh is fine enough that we can do without the optimal test functions.
We only mention this as a possible use of DPG; we are not going to look into such coupling in this research.

\section{Literature Review}
We start this literature review by looking at various numerical methods that have been popular in the simulation of fluid mechanical problems.
We then branch out to discuss the development of space-time finite elements in for various application domains.
Finally we explore some of the recent developments in the discontinuous Petrov-Galerkin finite element method.

\subsection{Methods for Computational Fluid Dynamics}
Computational fluid dynamics has been one of the driving forces behind numerical analysis since computers first became available for
scientific research and has followed the progression as simple methods give rise to more sophisticated ones with the maturation of computational
science as a discipline.
Finite difference methods were a popular choice in the early days of CFD, but these slowly gave way to finite volumes as the dominant choice.
As the analysis techniques in computational science have matured, it has been increasingly desirable to be able to prove certain properties of
numerical methods.
The solid mathematical foundation of the finite element method renders it especially nice for analysis, and in recent years finite elements have 
been developing a growing following among CFD practitioners.

\subsubsection{Finite Difference and Finite Volume Methods}
Finite difference methods approximate derivatives in the strong form of the equation under consideration with finite difference approximations.
These methods were first popularized for conservation laws by Lax who also introduced the idea of numerical flux and ideal of a monotone scheme.
For fluid dynamics applications, popular finite difference schemes use numerical fluxes to reconstruct approximate derivatives at certain mesh points.

Finite volume methods can be derived from applying finite difference principles to the integral form of the conservation law under consideration.
They are often derived by reference to a \emph{control volume}. 
The primary benefit of finite volume methods over their finite difference counterparts is that they are much easier to develop for general unstructured meshes.
Finite difference schemes typically require uniform or smoothly varying structured grids.

The presence of shocks in compressible Navier-Stokes simulations present a difficult problem for any numerical method.
The so called Gibbs phenomenon causes polynomial representations of unresolved discontinuous fields to develop undershoots and overshoots.
The length scale of shocks in the solution of the Navier-Stokes equations is often on the order of several mean free paths of the fluid under consideration.
So any simulation that does not resolve down to this level is going to have to deal with Gibbs.
This can be a problem when the undershoots threaten to take density or energy negative which can quickly cause the entire solution to
lose stability and return garbage.
The three classical techniques used to counter this possibility in finite difference and finite volume schemes are artificial viscosity, 
total variation diminishing schemes, and slope limiters.

\subsubsection{Stabilized Finite Element Methods}
\paragraph{SUPG}
\paragraph{VMS}
\paragraph{DG}
\paragraph{HDG}

\subsection{Space-time finite elements}
Oden (first to propose), Bob Haber, Tayfun Tezduyar, Neum\"{u}ller
\cite{Klaij2006}
\cite{Rhebergen2013}
% \cite{Haber2006}

\subsection{DPG}
General ideas 1-2 pages


\section{Goal}

\subsection{Organization of this Proposal}
\end{document}