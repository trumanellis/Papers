% -*- root: Proposal.tex -*-
\documentclass[Proposal.tex]{subfiles} 
\begin{document}
We summarize some of our completed work on a locally conservative DPG formulation that was invented to address mass loss concerns for standard DPG.
Locally conservative methods hold a special place for numerical analysts in
the field of fluid dynamics.
Perot\cite{Perot2011} argues
\begin{quote}
Accuracy, stability, and consistency are the mathematical concepts that are
typically used to analyze numerical methods for partial differential equations
(PDEs). These important tools quantify how well the mathematics of a PDE is
represented, but they fail to say anything about how well the physics of the
system is represented by a particular numerical method. In practice, physical
fidelity of a numerical solution can be just as important (perhaps even more
important to a physicist) as these more traditional mathematical concepts. A
numerical solution that violates the underlying physics (destroying mass or
entropy, for example) is in many respects just as flawed as an unstable
solution.
\end{quote}
There are also some mathematically attractive reasons to pursue local
conservation. The Lax-Wendroff theorem guarantees that a convergent numerical
solution to a system of hyperbolic conservation laws will converge to the
correct weak solution.

The discontinuous Petrov-Galerkin finite element method has been described as
least squares finite elements with a twist. The key difference is that least
squares methods seek to minimize the residual of the solution in the $L^2$
norm, while DPG seeks the minimization in a dual norm realized through the
inverse Riesz map. Exact mass conservation has been an issue that has long plagued
least squares finite elements. Several approaches have been
used to try to adress this. Bochev \etal\cite{Bochev2010} accomplish local
conservation by using a pointwise divergence free velocity space in the Stokes
formulation.  Chang and Nelson\cite{ChangNelson1997} developed the
\emph{restricted LSFEM}\cite{ChangNelson1997} by augmenting the least squares
equations with Lagrange multipliers explicitly enforcing mass conservation
element-wise. Our conservative formulation of DPG takes a similar approach and
both methods share similar negative of transforming a minimization method to a
saddle point problem. In the interest of crediting Chang and Nelson's
restricted LSFEM, we call the following locally conservative DPG method the
restricted DPG method (RDPG).


\section{DPG is a minimum residual method}
Roberts \etal presents a brief history and derivation of DPG with optimal test functions in
\cite{DPGStokes}. We follow his derivation of the standard DPG method as a
minimum residual method. Let $U$ be the trial Hilbert space and $V$ the test
Hilbert space for a well-posed variational problem $b(u,v)=l(v)$. In operator
form this is $Bu=l$, where $B:U\rightarrow V'$ and $\LRa{Bu,v}=b(u,v)$. We seek to minimize the
residual for the discrete space $U_h\subset U$:
\begin{equation}
u_h=\argmin_{u_h\in U_h}\frac{1}{2}\norm{Bu_h-l}^2_{V'}\,.
\label{minresidual}
\end{equation}
Recalling that the Riesz operator $R_V:V\rightarrow V'$ is an isometry defined
by
\[
\LRa{R_Vv,\delta v}=\LRp{v,\delta v}_V,\quad\forall\delta v\in V,
\]
we can use the Riesz inverse to minimize in the $V$-norm rather than its dual:
\begin{equation}
\frac{1}{2}\norm{Bu_h-l}^2_{V'}=\frac{1}{2}\norm{R_V^{-1}(Bu_h-l)}^2_V
=\frac{1}{2}\LRp{R_V^{-1}(Bu_h-l),R_V^{-1}(Bu_h-l)}_V\,.
\label{eq:rieszapplied}
\end{equation}
The first order optimality condition for \eqnref{rieszapplied} requires
the G\^ateaux derivative to be zero in all directions $\delta u \in
U_h$, i.e.,
\[
\left(R_V^{-1}(Bu_h-l),R_V^{-1}B\delta u\right)_V = 0, \quad \forall \delta u \in U.
\]
By definition of the Riesz operator, this is equivalent to
\begin{equation}
\LRa{Bu_h-l,R_V^{-1}B\delta u_h}=0\quad\forall\delta u_h\in U_h\,.
\label{eq:DPGbilinearform}
\end{equation}
Now, we can identify $v_{\delta u_h}\coloneqq R_V^{-1}B\delta u_h$ as the
optimal test function for trial function $\delta u_h$. Define $T:=R_V^{-1}B:U_h\rightarrow V$ as the trial-to-test operator. Now we can rewrite
\eqnref{DPGbilinearform} as
\begin{equation}
b(u_h,v_{\delta u_h})=l(v_{\delta u_h}).
\label{eq:DPGmethod}
\end{equation}
The DPG method then is to solve \eqnref{DPGmethod} with optimal test functions
$v_{\delta u_h}\in V$ that solve the auxiliary problem
\begin{equation}
\LRp{v_{\delta u_h},\delta v}_V=\LRa{R_Vv_{\delta u_h},\delta v}
=\LRa{B\delta u_h,\delta v}=b(\delta u_h,\delta v)\quad\forall\delta v\in V.
\label{eq:optimaltestproblem}
\end{equation}
Using a continuous test basis would result in a global solve for every optimal
test function. Therefore DPG uses a discontinuous test basis which makes each
solve element-local and much more computationally tractable. Of course,
\eqnref{optimaltestproblem} still requires the inversion of the
infinite-dimensional Riesz map, but approximating $V$ by a finite
dimensional space, $V_h$, which is of a higher polynomial degree than $U_h$ (hence
``enriched space'') works well in practice.

No assumptions have been made so far on the definition of the inner product on
$V$. In fact, proper choice of $\LRp{\cdot,\cdot}_V$ can make the difference
between a solid DPG method and one that suffers from robustness issues.


\section{Element conservative convection-diffusion}
We now proceed to develop a locally conservative formulation of DPG for
convection-diffusion type problems, but there are a few terms that we need to
define first. If $\Omega$ is our problem domain, then we can partition it into
finite elements $K$ such that
\[
\overline{\Omega} = \bigcup_K  \bar{K},\: \quad K \text { open},
\]
with corresponding {\em skeleton} $\Gamma_h$ and {\em interior
  skeleton} $\Gamma_h^0$,
\[
\Gamma_h := \bigcup_K \partial K\qquad \Gamma_h^0 := \Gamma_h - \Gamma.
\]
We define broken Sobolev spaces element-wise:
\[
\begin{array}{rl}
H^1(\Omega_h) & := \prod_K H^1(K), \\[8pt]
\bfH(\text{div},\Omega_h) & := \prod_K \bfH(\text{div},K).
\end{array}
\]
We also need the trace spaces:
\[
\begin{array}{rl}
H^\frac{1}{2}(\Gamma_h) & := \left\{ \hat{v} = \{\hat{v}_K \} \in \prod_K H^{1/2}(\partial K) \: :
\: \exists v \in H^1(\Omega) : v|_{\partial K} = \hat{v}_K \right\}, \\[8pt]
H^{-\frac{1}{2}}(\Gamma_h) & := \left\{ {\hat{\sigma}}_n = \{ {\hat{\sigma}}_{Kn} \}\in \prod_K H^{-1/2}(\partial K) \: : \: \exists \bfsigma \in \bfH(\text{div},\Omega)
: {\hat{\sigma}}_{Kn} = (\bfsigma \cdot \bfn)|_{\partial K} \right\},
\end{array}
\]
which are developed more precisely in \cite{DPGStokes}.

\subsection{Derivation}
Now that we have briefly outlined the abstract DPG method, let us apply it to
the convection-diffusion equation. The strong form of the steady
convection-diffusion problem with homogeneous Dirichlet boundary conditions reads
\[
\left\{
\begin{array}[c]{rrl}
\div(\bfbeta u)-\epsilon\Delta u & =f & \text{in }\Omega\\
u & =0 & \text{on }\Gamma\,,
\end{array}
\right.
\]
where $u$ is the property of interest, $\bs\beta$ is the convection vector,
and $f$ is the source term. Nonhomogeneous Dirichlet and Neumann boundary
conditions are straightforward but would add technicality to the following
discussion. Let us write this as an equivalent system of first
order equations:
\begin{align*}
\div(\bfbeta u-\bfsigma)&=f\\
\frac{1}{\epsilon}\bfsigma-\Grad u&=\bs0\,.
\end{align*}
If we then multiply the first equation by some scalar test function $v$ and the
bottom equation by some vector-valued test function $\bftau$, we can integrate by
parts over each element $K$:
\begin{equation}
\label{eq:preultraweak}
\begin{aligned}
-(\bfbeta u-\bfsigma,\nabla v)_K+((\bfbeta
u-\bfsigma)\cdot\mathbf{n},v)_{\partial K}&=(f,v)_K\\
\frac{1}{\epsilon}(\bfsigma,\bftau)_K+(u,\nabla\cdot\bftau)_K
-(u,\tau_n)_{\partial K}&=0\,.
\end{aligned}
\end{equation}
The discontinuous Petrov-Galerkin method refers to the fact that we are using
discontinuous optimal test functions that come from a space differing from the
trial space. It does not specify our choice of trial space. Nevertheless, many
versions of DPG in the literature (convection-diffusion \cite{DPG6},
linear elasticity \cite{BramwellDemkowiczGopalakrishnanQiu11}, linear
acoustics \cite{DemkowiczGopalakrishnanMugaZitelli12}, Stokes
\cite{DPGStokes}) associate DPG with the so-called ``ultra-weak formulation.''
We will follow the same derivation for the convection-diffusion equation, but
we emphasize that other formulations are available (in particular, the
Primal DPG\cite{PrimalDPG} method presents an alternative with
continuous trial functions). Thus, we seek field variables $u\in L^2(K)$ and
$\bfsigma\in\mathbf{L^2}(K)$. Mathematically, this leaves their traces on element
boundaries undefined, and in a manner similar to the hybridized discontinuous
Galerkin method, we define new unknowns for trace $\hat u$ and flux $\hat t$.
Applying these definitions to \eqnref{preultraweak} and adding the two
equations together, we arrive at our desired variational problem.

Find
$\bs u:=(u,\bfsigma,\hat u,\hat t)
\in\bs U:=L^2(\Omega_h)\times \bs L^2(\Omega_h)\times H^{1/2}(\Gamma_h)\times H^{-1/2}(\Gamma_h)$
such that
\begin{align}
\label{eq:variationalFormulation}
\underbrace{-(\bfbeta u-\bfsigma,\nabla v)_K+(\hat t,v)_{\partial K}
+ \frac{1}{\epsilon}(\bfsigma,\bftau)_K
+(u,\nabla\cdot\bftau)_K
-(\hat u,\tau_n)_{\partial K}}_{b(\mathbf{u}, \mathbf{v})}
&=\underbrace{\:(f,v)_K\genfrac{}{}{0pt}{}{}{}}_{l(\mathbf{v})} &\text{in }\Omega \\
\hat u&=0 &\text{on }\Gamma
\end{align}
for all $\bs v:=(v,\bftau)\in
\bs V:=H^1(\Omega_h)\times\bfH(\text{div},\Omega_h)$.

% jesse robustness section

We note that, for convection-diffusion problems, we are particularly
interested in designing a \textit{robust} DPG method.  Specifically, we are
interested in designing methods whose behavior does not change as the
diffusion parameter $\epsilon$ becomes very small.  Naive Galerkin methods for
convection-diffusion tend to suffer from a lack of robustness; specifically,
the finite element error is bounded by a constant factor of the best
approximation error, but the constant is often proportional to
$\epsilon^{-1}$.  Our aim is to design a DPG method with this in mind.  We
follow the methodology introduced by Heuer and Demkowicz in
\cite{DemkowiczHeuer}: the ultra-weak variational formulation for
convection-diffusion can be refactored as
\[
b\LRp{\LRp{u,\bfsigma,\uh,\hat t},\LRp{\bftau,v}} =
\sum_{K\in \Oh}\LRs{\LRa{\hat t,v}_{\delta K}
+\LRa{\uh,\tau_n}_{\delta K} + \LRp{u,\div \bftau
-\bfbeta\cdot\Grad v}_{L^2(K)}
+\LRp{\bfsigma,\frac{1}{\epsilon} \bftau + \Grad v}_{L^2(K)}},
\]
modulo application of boundary data.  If we choose specific
\textit{conforming} test functions satisfying the adjoint equations
\begin{align*}
\div \bftau - \bfbeta \cdot \Grad v &= u,\\
\frac{1}{\epsilon} \bftau + \Grad v &= \bfsigma,
\end{align*}
then evaluating $b\LRp{\LRp{u,\bfsigma,\uh,\fnh},\LRp{\bftau,v}}$ at these
specific test functions returns back $\norm{u}^2 + \norm{\bfsigma}^2$, the $L^2$
norm of our field variables.  Multiplying and dividing through by the test
norm $\norm{v}_V$, we have
\[
\norm{u}_{\L}^2 + \norm{\bfsigma}_{\L}^2 =
b\LRp{\LRp{u,\bfsigma,\uh,\fnh},\LRp{\bftau,v}} =
\frac{b\LRp{\LRp{u,\bfsigma,\uh,\fnh},\LRp{\bftau,v}}}{\norm{v}_V}\norm{v}_V
\leq \norm{u,\bfsigma,\uh,\fnh}_E\norm{v}_V,
\]
where
\[
\norm{u,\bfsigma,\uh,\fnh}_E = \sup_{v\in
V\setminus\LRc{0}}\frac{b\LRp{\LRp{u,\bfsigma,\uh,\fnh},\LRp{\bftau,v}}}{\norm{v}_V}
\]
is the DPG energy norm.  If we can robustly bound the test norm $\norm{v}_V
\lesssim \LRp{\norm{u}_{\L}^2+\norm{\bfsigma}^2_{\L}}^{1/2}$ (i.e. derive a
bound from above with a constant independent of $\epsilon$), then we can
divide through to get
\begin{equation}
\LRp{\norm{u}_{\L}^2 + \norm{\bfsigma}_{\L}^2}^{\frac{1}{2}} \lesssim
\norm{u,\bfsigma,\uh,\fnh}_E.
\label{eq:robustBound}
\end{equation}
In other words, the energy norm in which DPG is optimal bounds independently
of $\epsilon$ the $L^2$ norm; as we drive our energy error down to zero, we
can expect that the $L^2$ error will also decrease regardless of $\epsilon$.

We note that the construction of the test norm $\norm{v}_V$ for a robust DPG
method depends on two things: the test norm, as well as the adjoint equation.
In \cite{DemkowiczHeuer}, the standard problem with Dirichlet conditions
enforced over the entire boundary was considered; in
\cite{ChanHeuerThanhDemkowicz2012}, boundary conditions were chosen for the
forward problem such that the induced adjoint problem was regularized and
contained no strong boundary layers, allowing for the construction of a
stronger test norm on $V$.  We adopt a slight modification of the test norm
introduced in \cite{ChanHeuerThanhDemkowicz2012} for numerical experiments
here, which is motivated and explained in more detail in
% Section~\secref{sec:confusionPlate}.


Having reviewed and laid the foundation for DPG methods, we can now formulate our conservative DPG scheme.  % added by Jesse
Let $\bs U_h:=U_h\times\bs S_h\times\hat U_h\times\hat F_h\subset L^2(\Omega_h)\times\bs
L^2(\Omega_h)\times H^{\frac{1}{2}}(\Gamma_h)\times H^{-\frac{1}{2}}(\Gamma_h)$
be a finite-dimensional subspace, and let $\bs u_h:=(u_h.\bfsigma_h,\hat
u_h\hat t_h)\in\bs U_h$ be the group variable. The element conservative DPG scheme is
derived from the Lagrangian:
\begin{equation}
L(\bs u_h,\lambda_k)=\frac{1}{2}\norm{R_V^{-1}(b(\bs
u_h,\cdot)-(f,\cdot))}^2_{\bs V}-\sum_K\lambda_K(b(\bs u_h,(1_K,\bs0))-l((1_K,\bs0)))\,,
\label{eq:lagrangian}
\end{equation}
where $(1_K,\bs0)$ is the test function in which $v=1$ on element $K$ and 0 elsewhere and $\bftau=\bs0$ everywhere.

Taking the G\^ateaux derivatives as before, we arrive at the following system
of equations:
\begin{equation}
\left\{
\begin{array}[c]{rll}
b(\bs u_h,T(\delta\bs u_h))-\sum_K\lambda_K b(\bs u_h,(1_K,\bs0))
&=l(T(\delta\bs u_h)) & \forall\delta\bs u_h\in\bs U_h\\
b(\bs u_h,(1_K,\bs0)) &=l((1_K,\bs0)) & \forall K\,,
\end{array}
\right.
\label{eq:conservativeSystem}
\end{equation}
where $T:=R_V^{-1}B:\bs U_h\rightarrow\bs V$ is the same trial-to-test operator as in the original formulation.

Denote $T(\delta\bfu_h)=\LRp{\vdeltau,\taudeltau}\in H^1(\Omega_h)\times\bfH(div,\Omega_h)$.
Then, putting \eqref{eq:conservativeSystem} into more concrete terms for
convection-diffusion, we get:
\begin{equation}
\left\{
\begin{array}[c]{rll}
-(\bfbeta u-\bfsigma,\nabla \vdeltau)+\langle\hat t,\vdeltau\rangle
+ \frac{1}{\epsilon}(\bfsigma,\taudeltau)
+(u,\nabla\cdot\taudeltau)
-\langle\hat u,\taudeltau\cdot\bs n\rangle\\
-\sum_K\lambda_K (\delta\hat t,(1_K,\bs0))
&=(f,\vdeltau) & \forall\delta\bs u_h\in\bs U_h\\
\langle\hat t,(1_K,\bs0)\rangle &=(f,1_K) & \forall K\,.
\end{array}
\right.
\label{eq:conservativeConfusionSystem}
\end{equation}

\subsection{Stability analysis}
In the following analysis, we neglect the error due to the approximation of optimal test functions.
We follow the classical Brezzi's theory \cite{Brezzi1974,DBB05} for an abstract
mixed problem:
\begin{equation}
\left\{
\begin{array}{lll}
\bfu \in \bfU, p \in Q\\
a(\bfu,\bfw) + c(p,\bfw) & = l(\bfw) & \forall \bfw \in \bfU \\
c(q,\bfu) & = g(q) & \forall q \in Q
\end{array}
\right.
\end{equation}
where $\bfU,Q$ are Hilbert spaces, and $a,c,l,g$ denote the appropriate
bilinear and linear forms. Note that $a(\bfu,\bfw)=b(\bfu,T\bfw)=(T\bfu,T\bfw)_V$ in
the notation from the previous section.

Let function $\bfpsi$ denote the $\bfH(\text{div},\Omega)$ extension of flux $\hat{t}$
that realizes the minimum in the definition of the quotient (minimum energy
extension) norm.
The choice of norm for the Lagrange multipliers $\lambda_K$ is implied
by the quotient norm used for $H^{-1/2}(\Gamma_h)$ and continuity
bound for form $c(p,\bfw)$ representing the constraint:
\begin{equation}
\begin{array}{lll}
| c(\sum_K \lambda_K (1_K,{\bf 0}),(u,\bfsigma,\hat{u},\hat{t})) |
& = | \sum_K \lambda_K \langle \hat{t}, 1_K \rangle_{\partial K} | \\[8pt]
& = | \sum_K \lambda_K \langle v_n , 1_K \rangle_{\partial K} | \\[8pt]
& = | \sum_K \lambda_K \int_K \text{div} \bfpsi \: 1_K  | \\[8pt]
& \leq \sum_K  \lambda_K || \text{div} \bfpsi ||_{L^2(K)} \mu(K)^{1/2} \\[8pt]
& \leq (\sum_K \mu(K) \lambda_K^2)^{1/2} \: (\sum_K || \text{div} \bfpsi ||_{L^2(K)}^2 )^{1/2} \\[8pt]
& \leq \underbrace{\left(\sum_K \mu(K) \lambda_K^2\right)^{1/2}}_{=: || \bflambda ||} ||
\hat{t} ||_{H^{-1/2}(\Gamma_h)}\,,
\end{array}
\end{equation}
where $\mu(K)$ stands for the area (measure) of element $K$.
We proceed now with the discussion of the discrete inf-sup stability constants. We skip
index $h$ in the notation.

\paragraph{Inf Sup Condition} relating spaces $\bfU$ and $Q$ reads as follows:
\begin{equation}
   \sup_{\bfw \in \bfU} \frac{| c(p,\bfw) |}{|| \bfw ||_{\bfU}} \geq \beta ||
   p ||_Q\,.
\end{equation}
Let
\begin{equation}
R\, : \, L^2(\Omega) \ni q \to \bfpsi \in \bfH(\text{div},\Omega) \cap \bfH^1(\Omega)
=\bfH^1(\Omega)
\end{equation}
be
the continuous right inverse of the divergence operator constructed by
Costabel and McIntosh in \cite{CostabelMcIntosh}.
Let $\bfpsi_h$ denote the classical, lowest order Raviart-Thomas (RT) interpolant of
function
\begin{equation}
\bfpsi = R (\sum_K \lambda_K 1_K) \: .
\end{equation}
Note that $\text{div} \bfpsi_h = \text{div} \bfpsi = \lambda_K$ in element $K$.

Classical $h$-interpolation interpolation error estimates for the lowest error
Raviart-Thomas elements and continuity of operator $R$ imply the stability estimate:
\begin{equation}
\begin{array}{lll}
|| \bfpsi_h || & \leq || \bfpsi_h - \bfpsi || + || \bfpsi ||\\[8pt]
&\leq C h || \bfpsi ||_{H^1} +  || \bfpsi || \\[8pt]
& \leq C || \text{div} \bfpsi || = C (\sum_K \mu(K) \lambda_K^2)^{1/2}\,.
\end{array}
\label{eq:stab}
\end{equation}
Above, $C$ is a generic, mesh independent constant incorporating constant from
the interpolation error estimate and continuity constant of $R$.
Let $\hat{t}$ be the trace of $\bfpsi_h$. We have then,
\begin{equation}
\sup_{\hat{t} \in H^{-1/2}(\Gamma_h)} \frac{|  \sum_K \lambda_K \langle
\hat{t},1_K \rangle_{\partial K} |}{|| \hat{t} ||_{H^{-1/2}(\Gamma_h)}}
\geq \frac{| \sum_K \lambda_K \int_K \text{div} \bfpsi_h \: 1_K  |}
{|| \bfpsi_h ||_{H(\text{div},\Omega)}}
\geq \frac{1}{C} (\sum_K \mu(K) \lambda_K^2)^{1/2}\,,
\end{equation}
where $C$ is the constant from stability estimate~\eqref{eq:stab}.

Notice that we have considered traces of lowest order Raviart-Thomas elements
for the discretization of flux $\hat{t}$. The inf-sup condition for the lowest
order RT spaces implies automatically the analogous condition for elements of
arbitrary order; increasing the dimension of space $U$ only makes the discrete
inf-sup constant bigger.

\paragraph{Inf Sup in Kernel Condition} is satisfied automatically due to the use of optimal
test functions. First of all, we characterize the ``kernel'' space:
\begin{equation}
\begin{array}{rl}
\bfU_0  := & \LRc{ \bfw \in \bfU \, : \, c(q,\bfw) = 0 \quad \forall q \in Q} \\[8pt]
%  = &\{ (u,\bfsigma,\hat{u},\hat{t}) \, : \, \langle \hat{t},1_K \rangle = 0
%  \quad \forall K \}\,.
\end{array}
\end{equation}

In other words, the kernel space contains only the equilibrated fluxes.
With $\bfu \in \bfU_0$, we have then:
\begin{equation}
   \sup_{\bfw \in \bfU_0} \frac{| a(\bfu,\bfw) |}{|| \bfw ||_{\bfU} }
   \geq \frac{| b(\bfu,T \bfu) |}{|| \bfu ||}
   = \frac{| b(\bfu,T \bfu) |}{|| T\bfu ||}\frac{||T\bfu||}{||\bfu||}
   = \sup_{(v,\bftau)}\frac{| b((u,\bfsigma,\hat{u},\hat{t}), (v,\bftau)) |}{|| (v,\bftau) ||}
   \frac{||T\bfu||}{||\bfu||}
   \geq \gamma^2 || (u,\bfsigma,\hat{u},\hat{t}) ||\,,
\end{equation}
where $\gamma$ is the stability constant for the standard continuous DPG formulation.
The first inequality follows as we plug in the definition for $a$ and pick
$\bfw=\bfu$. The second equality is trivial, while the next one follows by definition of the optimal test
functions given through the trial-to-test operator $T$. The finally inequality
springs from the fact that
$\sup_{\bfv}\frac{|b(\bfu,\bfv)|}{||\bfv||}\geq\gamma||\bfu||$ and
$||T\bfu||_V=||R_V^{-1}B\bfu||_V=||B\bfu||_{V'}\geq\gamma||u||$.

With both discrete inf-sup constants in place, we have the standard result: the FE error
is bounded by the best approximation error. Notice that the exact Lagrange multipliers
are zero, so the best approximation error involves only solution $(u,\bfsigma,\hat{u},\hat{t})$.

\subsubsection{Robustness analysis}
\label{sec:robustnessAnalysis}

Recall the line of analysis leading to the construction of robust
test norms allowing us to bound the $L^2$ error of the field variables by the
energy error, \eqref{eq:robustBound}. With robust test norms, we have
\begin{equation}
\begin{array}{ll}
   \LRp{|| u - u_h ||^2+ || \bfsigma - \bfsigma_h ||^2}^{\frac{1}{2}}
& \lesssim || (u - u_h, \bfsigma - \bfsigma_h, \hat{u} - \hat{u}_h, \hat{t} - \hat{t}_h ||_E \\[8pt]
&= \inf_{(w_h,\bfvarsigma_h,\hat{w},\hat{r}_h)}
|| (u - w_h, \bfsigma - \bfvarsigma_h, \hat{u} - \hat{w}_h, \hat{t} - \hat{r}_h ||_E\,.
\end{array}
\end{equation}
The last equality follows from the fact that DPG method delivers the best approximation
error in the energy norm (minimizes the residual). This is no longer true for the
restricted version. So, can we claim robustness in the sense of the inequality above
for the restricted version as well?

One possible way to attack the problem is to switch to the energy norm in the Brezzi stability
analysis. Dealing with the ``inf-sup in kernel'' condition is simple. Upon replacing
the original norm of solution $\bfu$ with the energy norm, both constant $\gamma$ and continuity
constant become unity. In order to investigate the robustness of inf-sup  constant $\beta$,
we need to realize first what the energy norm of flux $\hat{t}$ is. Given an element $K$,
we solve for the optimal test functions corresponding to flux $\hat{t}$,
\begin{equation}
\left\{
\begin{array}{ll}
v_K \in H^1(K),\, \bftau_K \in \bfH(\text{div},K)\\[8pt]
((v_K,\bftau_K), (\delta v,\delta \bftau))_V = \langle \hat{t},\delta v\rangle_{\partial K}
\quad \forall \delta v \in H^1(K), \delta \bftau \in \bfH(\text{div},K)\,.
\end{array}
\right.
\label{eq:local_problem}
\end{equation}
The energy norm of $\hat{t}$ is then equal to
\begin{equation}
|| \hat{t} ||_E^2 = \sum_K || (v_K,\bftau_K) ||_V^2\,.
\end{equation}
We need to establish sufficient conditions under which the inf-sup and continuity constants for
the bilinear form representing the constraint are independent of viscosity $\epsilon$.

Let us start with the inf-sup condition,
\begin{equation}
\sup_{\hat{t}} \frac{| \sum_K \lambda_K \langle \hat{t},1_K \rangle |}{|| \hat{t} ||_E}
\geq  \beta \LRp{\sum_K \mu(K) \lambda_K^2}^{1/2}\,.
\end{equation}
As in the previous analysis, we select for $\hat{t}$ the trace of Raviart-Thomas
interpolant $\bfpsi_h$ of $\bfpsi = R (\sum_K \lambda_K 1_K)$ where $R$ is the right-inverse
of the divergence operator constructed by Costabel and McIntosh. The only change compared
with the previous analysis, is the evaluation of norm of $\hat{t}_h$. For this, we need
to solve the local problems:
\begin{align}
((v,\bftau), (\delta v, \delta \bftau))_V &= \langle \hat{t}, \delta v \rangle_{\partial K}
= \int_K \text{div} \bfpsi_h \, \delta v = \int_K \text{div} \bfpsi \, \delta v
\nonumber\\
&= \int_K \lambda_K \delta v
= \lambda_K (1_K,\delta v)_K  \quad \forall \delta v \in
H^1(K)\,\forall\delta\bftau\in\bfH(\text{div},K)\,.
\label{eq:local}
\end{align}
We need then an upper bound of the energy norm of $(v_h,\bftau_h)$:
$$
\LRp{\sum_K || (v,\bftau) ||_V^2}^{1/2}\,.
$$
Substituting $(v,\bftau)$ for $(\delta v,\delta\bftau)$ in~\eqref{eq:local}, we get,
\begin{equation}
|| (v,\bftau) ||_V^2 = \lambda_K (1_K, v_K)\,.
\end{equation}
If we have a robust stability estimate:
\begin{equation}
(1_K, v_K) \leq C \mu(K)^{1/2} || (v,\bftau) ||_K
\label{eq:robustEst1}
\end{equation}
(i.e. constant $C$ is independent of $\epsilon$), then
\begin{equation}
||  (v,\bftau) ||_V \leq C \mu(K)^{1/2} | \lambda_K |
\end{equation}
and, eventually as needed,
\begin{equation}
\sum_K || (v,\bftau) ||_V^2  \leq C^2 \sum_K \mu(K) \lambda_K^2\,,
\end{equation}
which leads to the robust estimate of inf-sup constant $\beta$. For example, it is sufficient if
\begin{equation}
|| v ||_{L^2(K)} \leq || (v,\bftau) ||_V \, .
\label{eq:robust_est1}
\end{equation}
Notice that the stability analysis with the energy norm was, in a sense, easier than
with the quotient norm. Only the divergence of the interpolant $\bfpsi_h$
enters~\eqref{eq:local}
and it coincides with the divergence of $\bfpsi$.

We arrive at a similar situation in the continuity estimate of
$$
\sum_K \lambda_K \langle \hat{t}, 1_K \rangle\,.
$$
Testing with $(1_K,{\bf 0})$ in the local problem~\eqref{eq:local_problem}, we obtain,
\begin{equation}
((v,\bftau),(1_K,{\bf 0}))_V = \langle \hat{t}, 1_K \rangle_{\partial K}\,.
\label{eq:local2}
\end{equation}
If we have a robust estimate,
\begin{equation}
| ((v,\bftau),(1_K,{\bf 0}))_V | \leq C \mu(K)^{1/2} \, || (v,\bftau) ||_V\,,
\label{eq:robust_est2}
\end{equation}
then
\begin{equation}
| \sum_K \lambda_K \langle \hat{t}, 1_K \rangle | \leq C (\sum_K \mu(K) \lambda_K^2)^{1/2}
\, (\sum_K || (v,\bftau) ||_V^2)^{1/2}
= C (\sum_K \mu(K) \lambda_K^2)^{1/2}  || \hat{t} ||_E\,,
\end{equation}
as needed.

For instance, condition~\eqref{eq:robust_est2} will be satisfied if the test inner product
in~\eqref{eq:local2} reduces to the $L^2$ term only,
\begin{equation}
((v,\bftau),(1_K,{\bf 0}))_V = (v,1_K)_{L^2(K)} \: .
\end{equation}
With the robust stability and continuity constants for the mixed problem, the energy error
of solution $(u,\bfsigma,\hat{u},\hat{t})$ (and Lagrange multipliers $\lambda_K$ as well)
is bounded robustly by the {\em best approximation error} of  $(u,\bfsigma,\hat{u},\hat{t})$
measured in the energy norm. We arrive thus at the same situation as in the standard
DPG method.

\subsection{Robust test norms}
\subsubsection{A model problem}
\subsubsection{A modification of the robust test norm}
\subsubsection{Adaptation for a locally conservative formulation}
\subsubsection{Proof of robust stability estimate}


\section{Application to other fluid model problems}

\subsection{Inviscid Burgers' equation}

\subsection{Stokes flow}


\section{Numerical Experiments}

\subsection{Erickson-Johnson model problem}

\subsection{Vortex problem}

\subsection{Discontinuous source problem}

\subsection{Inviscid Burgers' problem}

\subsection{Stokes flow around a cylinder}

\subsection{Stokes flow over a backward facing step}

\end{document}