% -*- root: Proposal.tex -*-
\documentclass[Proposal.tex]{subfiles} 
\begin{document}
\chapter{Proposed work}
In this chapter we outline the proposed work of solving the Navier-Stokes equations with space-time DPG. 
Our emphasis will be on incompressible Navier-Stokes, but time permitting, we would like to investigate the compressible equations as well.


\section{Space-time DPG for Compressible Navier-Stokes}
For the sake of narrative, we start with the full compressible Navier-Stokes equations and then simplify to the incompressible case.
The compressible Navier-Stokes equations are
\begin{align}
\frac{\partial}{\partial t}\svectthree{\rho}{\rho\bfu}{\rho e_0}
+\Div\svectthree{\rho\bfu}{\rho\bfu\otimes\bfu+p\bfI-\mathbb{D}}{\rho\bfu e_0+\bfu p+\bfq-\bfu\cdot\mathbb{D}}
%TODO: Possible error above. cfd-online seems to have T^T
=\svectthree{f_c}{\bff_m}{f_e}\,,
\end{align}
where $\rho$ is the density, $\bfu$ is the velocity, $p$ is the pressure, $\bfI$ is the identity matrix,
$\mathbb{D}$ is the deviatoric stress tensor or viscous stress, $e_0$ is the total energy, $\bfq$ is the heat flux, 
and $f_c$, $\bff_m$, and $f_e$ are the source terms for the continuity, momentum, and energy equations, respectively.
Assuming Stokes hypothesis that $\lambda=-\frac{2}{3}\mu$, 
\begin{equation*}
	\mathbb{D}=2\mu\bfS^*=2\mu\LRs{\frac{1}{2}\LRp{\Grad\bfu+\LRp{\Grad\bfu}^T}-\frac{1}{3}\Div\bfu\bfI}\,,
\end{equation*}
where $\bfS^*$ is the trace-less viscous strain rate tensor.
The heat flux is given by Fourier's law:
\begin{equation*}
	\bfq=-C_p\frac{\mu}{Pr}\Grad T\,,
\end{equation*}
where $C_p$ is the specific heat at constant pressure and $Pr$ is the laminar Prandtl number: $Pr:=\frac{C_p\mu}{\lambda}$.
We need to close these equations with an equation of state. An ideal gas assumption gives
\begin{equation*}
	\gamma:=\frac{C_p}{C_v}\,,\quad p=\rho RT\,,\quad e=C_v T\,,\quad C_p-C_v=R\,,
\end{equation*}
where $\gamma$ is the ratio of specific heats, $C_v$ is the specific heat at constant volume, $R$ is the gas constant,
$e$ is the internal energy, $T$ is the temperature,
and $\gamma$, $C_p$, $C_v$, and $R$ are constant properties of the fluid.
The total energy is defined by
\begin{equation*}
	e_0=e+\frac{1}{2}\bfu\cdot\bfu\,.
\end{equation*}

We can write our first order system of equations in space-time as follows:
\begin{subequations}
\label{eq:compressibleNSFirstOrder}
\begin{align}
	\mathbb{D}-\mu\LRp{\Grad\bfu+\LRp{\Grad\bfu}^T}+\frac{2\mu}{3}\Div\bfu\bfI&=0\\
	\bfq+C_p\frac{\mu}{Pr}\Grad T&=0\\
	\Divxt\vecttwo{\rho\bfu}{\rho}&=f_c\\
	\Divxt\vecttwo{\rho\bfu\otimes\bfu+\rho RT\bfI-\mathbb{D}}{\rho\bfu}&=\bff_m\\
	\Divxt\vecttwo{\rho\bfu\LRp{C_v T+\frac{1}{2}\bfu\cdot\bfu}+\bfu\rho RT+\bfq-\bfu\cdot\mathbb{D}}{\rho\LRp{C_v T+\frac{1}{2}\bfu\cdot\bfu}}&=f_e\,,
\end{align}
\end{subequations}
where our solution variables are $\rho$, $\bfu$, $T$, $\mathbb{D}$, and $\bfq$.

\subsection{Derivation of Space-Time DPG Formulation}
We start with \eqref{eq:compressibleNSFirstOrder} and multiply by test functions $\mathbb{S}$ (symmetric tensor), $\bftau$, $v_c$, $\bfv_m$, $v_e$, 
then integrate by parts over each space-time element $K$:
\begin{subequations}
\label{eq:compressibleNSBF}
\begin{align}
	\LRp{\mathbb{D},\mathbb{S}}+\LRp{2\mu\bfu,\Div\mathbb{S}}-\LRp{\frac{2\mu}{3}\bfu,\Grad\mathbb{S}_{ii}}
	+\LRa{2\mu\hat u,\mathbb{S}\bfn_x}+\LRa{\frac{2\mu}{3}\hat u,\mathbb{S}\bfn_x}&=0\\
	\LRp{\bfq,\bftau}-\LRp{C_p\frac{\mu}{Pr}T,\Div\bftau}+\LRa{\hat T,\tau_n}&=0\\
	-\LRp{\vecttwo{\rho\bfu}{\rho},\Gradxt v_c}+\LRa{\hat t_c,v_c}&=\LRp{f_c,v_c}\\
	-\LRp{\vecttwo{\rho\bfu\otimes\bfu+\rho RT\bfI-\mathbb{D}}{\rho\bfu},\Gradxt\bfv_m}+\LRa{\hat\bft_m,\bfv_m}&=\LRp{\bff_m,\bfv_m}\\
	-\LRp{\vecttwo{\rho\bfu\LRp{C_v T+\frac{1}{2}\bfu\cdot\bfu}+\bfu\rho RT+\bfq-\bfu\cdot\mathbb{D}}{\rho\LRp{C_v T+\frac{1}{2}\bfu\cdot\bfu}},\Gradxt v_e}
	+\LRa{\hat t_e,v_e}&=\LRp{f_e,v_e}\,,
\end{align}
\end{subequations}
where 
\begin{equation*}
\begin{aligned}
\hat\bfu&=\trace(\bfu)\\
\hat T&=\trace(T)\\
\hat t_c&=\trace\LRp{\rho\bfu}\cdot\bfn_x
+\trace\LRp{\rho}n_t\\
\hat\bft_m&=\trace\LRp{\rho\bfu\otimes\bfu+\rho RT\bfI-\mathbb{D}}\cdot\bfn_x
+\trace\LRp{\rho\bfu} n_t\\
\hat t_e&=\trace\LRp{\rho\bfu\LRp{C_v T+\frac{1}{2}\bfu\cdot\bfu}+\bfu\rho RT+\bfq-\bfu\cdot\mathbb{D}}\cdot\bfn_x
+\trace\LRp{\rho\LRp{C_v T+\frac{1}{2}\bfu\cdot\bfu}}n_t\,.
\end{aligned}
\end{equation*}
Note that integrating $\mathbb{S}$ against the symmetric gradient only picks up the symmetric part.
This is a much more complicated system of equations than we had for the space-time heat equation, but the situation has many similarities.
Test function $\bftau\in\HdivK$ where the divergence is taken only over spatial dimensions, $v_c,v_e\in\HOneK$, and $\bfv_m\in\HOneVecK$.
These are all familiar spaces from our work with the heat equation.
Unfortunately, $\mathbb{S}$ has some weird requirements: each $d\times d$ components must be at least in $L^2(K)$, $\Div\mathbb{S}\in\LVecK$, and
$\Grad\mathbb{S}_{ii}\in\LVecK$.
In practice, we will probably just seek each component in $\HOneK$.

\subsubsection{Linearization}
We follow the same linearization process as we have for previous problems.
Let $U=\LRc{\rho,\bfu,T,\mathbb{D},\bfq,\hat u,\hat e,\hat t_c,\hat\bft_m,\hat t_e}$, 
then define residual $R(U)$ as the left hand side of \eqref{eq:compressibleNSBF} minus the right hand side.
Let $\tilde U$ be an approximate solution for the minimization of the residual. 
We wish to solve for an increment $\Delta U$ such that $U=\tilde U+\Delta U$ is a better approximation of the true solution.
Approximating $R(U)=0$ by $R(\tilde U)+R'(\tilde U)\Delta U=0$, where $R'(\tilde U)$ is the Jacobian of $R$ evaluated at $\tilde U$, we get a linear system:
\begin{equation}
	R'(\tilde U)\Delta U=-R(\tilde U)\,.
\end{equation}
We only need to define our Jacobian and residual for each component of \eqref{eq:compressibleNSBF}. 
Note that we can exclude our traces and fluxes from the residual since these unknowns are only involved in linear terms.
As such, we don't update these variables each iteration based on the previous iteration.
Instead, for example, we consider $\tilde{\hat u}=0$, and let $\hat u=\Delta\hat u$ after each iteration.
To indicate this different treatment, we drop the delta and tilde notation on these variables in the following discussion.
The Jacobian of our compressible Navier-Stokes system, $R'(\tilde U)\Delta U$ is
\begin{equation}
\label{eq:compressibleJacobian}
\begin{aligned}
	&\LRp{\Delta\mathbb{D},\mathbb{S}}+\LRp{2\mu\Delta\bfu,\Div\mathbb{S}}-\LRp{\frac{2\mu}{3}\Delta\bfu,\Grad\mathbb{S}_{ii}}
	+\LRa{2\mu\hat u,\mathbb{S}\bfn_x}+\LRa{\frac{2\mu}{3}\hat u,\mathbb{S}\bfn_x}\\
	%
	&+\LRp{\Delta\bfq,\bftau}-\LRp{C_p\frac{\mu}{Pr}\Delta T,\Div\bftau}+\LRa{\hat T,\tau_n}\\
	%
	&-\LRp{\vecttwo{\Delta\rho\tilde\bfu+\tilde\rho\Delta\bfu}
	{\Delta\rho},\Gradxt v_c}
	+\LRa{\hat t_c,v_c}\\
	%
	&-\LRp{\vecttwo{\Delta\rho\tilde\bfu\otimes\tilde\bfu+\tilde\rho\Delta\bfu\otimes\tilde\bfu+\tilde\rho\tilde\bfu\otimes\Delta\bfu
	+\LRp{\Delta\rho\frac{R}{C_v}\tilde e+\tilde\rho RT}\bfI-\Delta\mathbb{D}}
	{\Delta\rho\tilde\bfu+\tilde\rho\Delta\bfu},\Gradxt\bfv_m}
	+\LRa{\hat\bft_m,\bfv_m}\\
	%
	&-\LRp{\vectthree{[C_v\Delta\rho\tilde T\tilde\bfu+C_v\tilde\rho\Delta T\tilde\bfu+C_v\tilde\rho\tilde T\Delta\bfu
	+\frac{1}{2}\LRp{\Delta\rho\tilde\bfu\cdot\tilde\bfu\tilde\bfu+\tilde\rho\Delta\bfu\cdot\tilde\bfu\tilde\bfu
	+\tilde\rho\tilde\bfu\cdot\Delta\bfu\tilde\bfu+\tilde\rho\tilde\bfu\cdot\tilde\bfu\Delta\bfu}}
	{+R\LRp{\Delta\rho\tilde T\tilde\bfu+\tilde\rho\Delta T\tilde\bfu+\tilde\rho\tilde T\Delta\bfu}
	+\Delta\bfq-\Delta\bfu\cdot\tilde{\mathbb{D}}-\tilde\bfu\cdot\Delta\mathbb{D}]}
	{C_v\Delta\rho\tilde T+C_v\tilde\rho\Delta T
	+\frac{1}{2}\LRp{\Delta\rho\tilde\bfu\cdot\tilde\bfu+\tilde\rho\Delta\bfu\cdot\tilde\bfu+\tilde\rho\tilde\bfu\cdot\Delta\bfu}},\Gradxt v_e}\\
	&+\LRa{\hat t_e,v_e}\,.
\end{aligned}
\end{equation}
The residual, $R(\tilde U)$, is then
\begin{equation}
\begin{aligned}
	&\LRp{\tilde{\mathbb{D}},\mathbb{S}}+\LRp{2\mu\tilde\bfu,\Div\mathbb{S}}-\LRp{\frac{2\mu}{3}\tilde\bfu,\Grad\mathbb{S}_{ii}}\\
	&+\LRp{\tilde\bfq,\bftau}-\LRp{C_p\frac{\mu}{Pr}\tilde T,\Div\bftau}\\
	&-\LRp{\vecttwo{\tilde\rho\tilde\bfu}{\tilde\rho},\Gradxt v_c}-\LRp{f_c,v_c}\\
	&-\LRp{\vecttwo{\tilde\rho\tilde\bfu\otimes\tilde\bfu+\tilde\rho R\tilde T\bfI-\tilde{\mathbb{D}}}{\tilde\rho\tilde\bfu},
	\Gradxt\bfv_m}-\LRp{\bff_m,\bfv_m}\\
	&-\LRp{\vecttwo{\tilde\rho\tilde\bfu\LRp{C_v\tilde T+\frac{1}{2}\tilde\bfu\cdot\tilde\bfu}+\tilde\bfu\tilde\rho R\tilde T
	+\tilde\bfq-\tilde\bfu\cdot\tilde{\mathbb{D}}}{\tilde\rho\LRp{C_v\tilde T+\frac{1}{2}\tilde\bfu\cdot\tilde\bfu}},
	\Gradxt v_e}-\LRp{f_e,v_e}\,.
\end{aligned}
\end{equation}

\subsubsection{Test Norm}

\subsection{Problems to consider}
The original inspiration to pursue a space-time DPG formulation stemmed from the inability to solve one particular steady-state compressible flow problem.
Chan\cite{JesseDissertation} was unable to achieve convergence on the supersonic Carter plate problem\cite{Carter1973} with Reynolds numbers over 10,000.
This convergence failure was also reported in \cite{KirkDissertation}.

The Carter flat plate problem simulates laminar supersonic flow over an infinitesimally thin infinite flat plate.
An oblique shock forms at the leading edge of the plate. 
A laminar boundary layer forms at the same location along the length of the plate.
The problem domain is $\Omega=[0,2]\times[0,1]$.
Chan assigned the following boundary conditions.
\begin{description}
	\item[Symmetry boundary conditions:] Upstream of the plate and on the upper surface, $u_n=q_n=\frac{\partial u_s}{\partial n}=0$.
	This implies that $u_2=q_2=\mathbb{D}_{12}=0$. 
	We can enforce these conditions with boundary conditions on the fluxes and traces. $u_2=0$ can clearly be enforced via $\hat u_2=0$.
	Enforcing $q_n=0$ is just as easy, though less obvious - it requires setting $\hat t_e=0$.
	Recall that 
	\begin{equation*}
	\hat t_e=\trace\LRp{\rho\bfu\LRp{C_v T+\frac{1}{2}\bfu\cdot\bfu}+\bfu\rho RT+\bfq-\bfu\cdot\mathbb{D}}\cdot\bfn_x
	+\trace\LRp{\rho\LRp{C_v R+\frac{1}{2}\bfu\cdot\bfu}}n_t\,.
	\end{equation*}
	On the symmetry boundaries, $n_t=0$, and $\bfn_x=(0,\pm1)$ depending whether we are on the top or bottom surface. 
	Furthermore, $u_2=0$. Taken together, every term in $\hat t_e$ drops out except for the $\trace(\bfq)\cdot\bfn_x=\pm\trace(q_2)$ term.
	Thus, on the symmetry boundaries, setting a boundary condition on $\hat t_e$ is equivalent to setting a boundary condition on the trace of $q_2$.
	A similar situation occurs when attempting to set boundary conditions on $\mathbb{D}_{12}$. Recall that
	\begin{equation*}
	\hat\bft_m=\trace\LRp{\rho\bfu\otimes\bfu+\rho RT\bfI-\mathbb{D}}\cdot\bfn_x
	+\trace\LRp{\rho\bfu} n_t\,.
	\end{equation*}
	On the symmetry boundaries, most of the terms drop out of the second component except for the $\mathbb{D}$ term.
	Thus, setting $\hat t_{m2}=0$ is equivalent to setting $\trace(\mathbb{D}_{12})=0$ on these boundaries.
	\item[Flat plate boundary conditions:]

\end{description}

The reported problem with the convergence was that the solution appeared to oscillate slightly upstream of the plate edge.
Our theory is that this oscillation is a product of an unresolved mesh, but on a fully resolved space-time mesh these oscillations will dissipate, 
and we can recover the steady state solution.

\section{Space-time DPG for 2D Incompressible Navier-Stokes}
The incompressible assumption simplifies the Navier-Stokes equations significantly.
The incompressible Navier-Stokes equations for a Newtonian fluid in conservation form are
\begin{align*}
  \rho\frac{\partial\bfu}{\partial t}+\Div\LRp{\rho\bfu\otimes\bfu-\mu\Grad\bfu+p\bfI}&=\bff\\
  \Div\bfu&=0\,,
  % \int_\Omega p&=0
\end{align*}
where $\bfu$ is the fluid velocity, $p$ is the pressure, $\rho$ is the density, $\mu$ is the first coefficient of viscosity (usually called ``viscosity''), 
and $\bff$ is the forcing. 
This is to be solved on a space-time domain $\Omega\times[t_0,T]$.
We can rewrite the momentum equation in terms of a full space-time divergence:
\begin{equation}
\Divxt\vecttwo{\bfu\otimes\bfu-\nu\Grad\bfu+p\bfI}{u}=\bff\,.
\end{equation}
Note that the pressure component of a solution will only be unique up to a constant. 
As a result, we impose a zero mean condition on the pressure: $\int_\Omega p=0$.
The deviatoric stress tensor is identified as
\begin{equation}
	\mathbb{D}_{ij}=\mu\LRp{\frac{\partial u_i}{\partial x_j}+\frac{\partial u_j}{\partial x_i}
	+\delta_{ij}\frac{\lambda}{\mu}\frac{\partial u_k}{\partial x_k}}\,,
\end{equation}
where $\lambda$ is the second coefficient of viscosity.
For an incompressible fluid, the last term is zero.

\subsection{Derivation of Space-Time DPG Formulation}
The subsequent derivation is for spatially 2D flow, but 1D and 3D are very similar.
Following standard procedure, we write

\subsection{Problems to consider}


\end{document}