% -*- root: Dissertation.tex -*-
\documentclass[Dissertation.tex]{subfiles}
\begin{document}
\graphicspath{{../Figures/}}
\chapter{Comparison of Primitive, Conservation, and Entropy Variables for Compressible Navier-Stokes}
\label{sec:VariableComparison}
% \section{Nonlinear Formulation}
In this appendix we discuss some work we did exploring a comparison between three formulations of the 
compressible Navier-Stokes equations: primitive variables, conservation variables, and entropy variables.
Primitive variables are the natural physically intuitive variables in which the Navier-Stokes 
equations are usually presented: density, velocity, and temperature.
Conservation variables are popular as they simplify time stepping algorithms. 
The independent variables are density, momentum, and total energy.
Entropy variables were proposed by Tom Hughes in \cite{HughesEntropyVariables}
and are selected such that the stiffness matrix in a Bubnov-Galerkin finite element 
discretization is symmetric. 
However the independent variables do not correspond to any intuitive physical quantity
and the resulting equations are the most nonlinear of the three.
Recalling the definitions from Chapter~\ref{sec:compressible}, we define the necessary
linear and nonlinear terms that fit within that framework.

\section{Primitive Variables}
We begin by recalling the definitions for primitive variables:
\begin{align*}
C_c&:=\rho\\
\bfC_m&:=\rho\bfu\\
C_e&:=\rho(C_v T+\frac{1}{2}\bfu\cdot\bfu)\\
\bfF_c&:=\rho\bfu\\
\mathbb{F}_m&:=\rho\bfu\otimes\bfu+\rho RT\bbI\\
\bfF_e&:=\rho\bfu\LRp{C_v T+\frac{1}{2}\bfu\cdot\bfu}+\bfu\rho RT\\
\bfK_c&:=\boldsymbol 0\\
\bbK_m&:=\LRp{\bbD+\bbD^T-\frac{2}{3}\trace(\bbD)\bbI}\\
\bfK_e&:=-\bfq+\bfu\cdot\LRp{\bbD+\bbD^T-\frac{2}{3}\trace(\bbD)\bbI}\\
\bbM_{\bbD}&:=\bbD\\
\bfM_{\bfq}&:=\frac{Pr}{C_p}\bfq\\
\bfG_{\bbD}&:=\bfu\\
G_{\bfq}&:=-T\,.
\end{align*}

\subsection{Linearized Terms}
The linearized terms are:
\begin{align*}
\pd{C_c(\tilde W)}{W}\Delta W&:=
	\Delta\rho\\
\pd{\bfC_m(\tilde W)}{W}\Delta W&:=
	\Delta\rho\tilde\bfu+\tilde\rho\Delta\bfu\\
\pd{C_e(\tilde W)}{W}\Delta W&:=
	C_v\Delta\rho\tilde T+C_v\tilde\rho\Delta T
	+\frac{1}{2}\LRp{\Delta\rho\tilde\bfu\cdot\tilde\bfu+\tilde\rho\Delta\bfu\cdot\tilde\bfu+\tilde\rho\tilde\bfu\cdot\Delta\bfu}\\
\pd{\bfF_c(\tilde W)}{W}\Delta W&:=
	\Delta\rho\tilde\bfu+\tilde\rho\Delta\bfu\\
\pd{\mathbb{F}_m(\tilde W)}{W}\Delta W&:=
	\Delta\rho\tilde\bfu\otimes\tilde\bfu
	+\tilde\rho\Delta\bfu\otimes\tilde\bfu
	+\tilde\rho\tilde\bfu\otimes\Delta\bfu
	+R\LRp{\Delta\rho\tilde T+\tilde\rho\Delta T}\bbI\\
\pd{\bfF_e(\tilde W)}{W}\Delta W&:=
	C_v\Delta\rho\tilde\bfu\tilde T+C_v\tilde\rho\Delta\bfu\tilde T+C_v\tilde\rho\tilde\bfu\Delta T
	\\&\quad
	+\frac{1}{2}\Delta\rho\tilde\bfu\tilde\bfu\cdot\tilde\bfu
	+\frac{1}{2}\tilde\rho\Delta\bfu\tilde\bfu\cdot\tilde\bfu
	+\frac{1}{2}\tilde\rho\tilde\bfu\Delta\bfu\cdot\tilde\bfu
	+\frac{1}{2}\tilde\rho\tilde\bfu\tilde\bfu\cdot\Delta\bfu
	\\&\quad
	+R\Delta\bfu\tilde\rho\tilde T
	+R\tilde\bfu\Delta\rho\tilde T
	+R\tilde\bfu\tilde\rho\Delta T\\
\pd{\bfK_c(\tilde W)}{W}\Delta W&:=
	\boldsymbol 0\\
\pd{\bbK_m(\tilde W)}{W}\Delta W&:=
	\LRp{\Delta\bbD+\Delta\bbD^T-\frac{2}{3}\trace(\Delta\bbD)\bbI}\\
\pd{\bfK_e(\tilde W)}{W}\Delta W&:=
	-\Delta\bfq+\Delta\bfu\cdot\LRp{\tilde\bbD+\tilde\bbD^T-\frac{2}{3}\trace(\tilde\bbD)\bbI}
	\\&\quad
	+\tilde\bfu\cdot\LRp{\Delta\bbD+\Delta\bbD^T-\frac{2}{3}\trace(\Delta\bbD)\bbI}\\
\pd{\bbM_{\bbD}(\tilde W)}{W}\Delta W&:=
	\Delta\bbD\\
\pd{\bfM_{\bfq}(\tilde W)}{W}\Delta W&:=
	\frac{Pr}{C_p}\Delta\bfq\\
\pd{\bfG_{\bbD}(\tilde W)}{W}\Delta W&:=
	\Delta\bfu\\
\pd{G_{\bfq}(\tilde W)}{W}\Delta W&:=
	-\Delta T\,.
\end{align*}

\section{Conservation Variables}
The definition of conservation variables is as follows: 
\begin{align*}
\rho&=\rho\\
\bfm&=\rho\bfu\\
E&=\rho\LRp{C_v T+\frac{1}{2}\bfu\cdot\bfu}\,.
\end{align*}

This gives us new definitions for our nonlinear terms:
\begin{align*}
C_c&:=\rho\\
\bfC_m&:=\bfm\\
C_e&:=E\\
\bfF_c&:=\bfm\\
\mathbb{F}_m&=\frac{\bfm\otimes\bfm}{\rho}+(\gamma-1)\LRp{E-\frac{\bfm\cdot\bfm}{2\rho}}\bbI\\
\bfF_e&=\gamma E\frac{\bfm}{\rho}-(\gamma-1)\frac{\bfm\cdot\bfm}{2\rho^2}\bfm\\
\bfK_c&:=\boldsymbol 0\\
\bbK_m&:=\LRp{\bbD+\bbD^T-\frac{2}{3}\trace(\bbD)\bbI}\\
\bfK_e&:=-\bfq+\frac{\bfm}{\rho}\cdot\LRp{\bbD+\bbD^T-\frac{2}{3}\trace(\bbD)\bbI}\\
\bbM_{\bbD}&:=\bbD\\
\bfM_{\bfq}&:=\frac{Pr}{C_p}\bfq\\
\bfG_{\bbD}&:=\frac{\bfm}{\rho}\\
G_{\bfq}&:=-\LRp{\frac{E-\frac{1}{2\rho}\bfm\cdot\bfm}{C_v\rho}}\,.
\end{align*}

\subsection{Linearized Terms}
After linearizing, we get the following:
\begin{align*}
\pd{C_c(\tilde W)}{W}\Delta W&:=
	\Delta\rho\\
\pd{\bfC_m(\tilde W)}{W}\Delta W&:=
	\Delta\bfm\\
\pd{C_e(\tilde W)}{W}\Delta W&:=
	\Delta E\\
\pd{\bfF_c(\tilde W)}{W}\Delta W&:=
	\Delta\bfm\\
\pd{\mathbb{F}_m(\tilde W)}{W}\Delta W&:=
	\frac{\Delta\bfm\otimes\tilde\bfm}{\tilde\rho}
	+\frac{\tilde\bfm\otimes\Delta\bfm}{\tilde\rho}
	-\frac{\tilde\bfm\otimes\tilde\bfm}{\tilde\rho^2}\Delta\rho
	\\&\quad
	+(\gamma-1)\LRp{\Delta E
	-\frac{\Delta\bfm\cdot\tilde\bfm}{2\tilde\rho}
	-\frac{\tilde\bfm\cdot\Delta\bfm}{2\tilde\rho}
	+\frac{\tilde\bfm\cdot\tilde\bfm}{2\tilde\rho^2}\Delta\rho
	}\bbI\\
\pd{\bfF_e(\tilde W)}{W}\Delta W&:=
	\gamma\LRp{
	\Delta E\frac{\tilde\bfm}{\tilde\rho}
	+\tilde E\frac{\Delta\bfm}{\tilde\rho}
	-\tilde E\frac{\tilde\bfm}{\tilde\rho^2}\Delta\rho}
	\\&\quad
	+(\gamma-1)\left(
	-\frac{\Delta\bfm\tilde\bfm\cdot\tilde\bfm}{2\tilde\rho^2}
	-\frac{\tilde\bfm\Delta\bfm\cdot\tilde\bfm}{2\tilde\rho^2}
	\right.\\&\quad\quad\left.
	-\frac{\tilde\bfm\tilde\bfm\cdot\Delta\bfm}{2\tilde\rho^2}
	+\frac{\tilde\bfm\tilde\bfm\cdot\tilde\bfm}{\tilde\rho^3}\Delta\rho\right)\\
\pd{\bfK_c(\tilde W)}{W}\Delta W&:=
	\boldsymbol 0\\
\pd{\bbK_m(\tilde W)}{W}\Delta W&:=
	\LRp{\Delta\bbD+\Delta\bbD^T-\frac{2}{3}\trace(\Delta\bbD)\bbI}\\
\pd{\bfK_e(\tilde W)}{W}\Delta W&:=
	-\Delta\bfq+\LRp{\frac{\Delta\bfm}{\tilde\rho}-\frac{\tilde\bfm}{\tilde\rho^2}\Delta\rho}
	\cdot\LRp{\tilde\bbD+\tilde\bbD^T-\frac{2}{3}\trace(\tilde\bbD)\bbI}
	\\&\quad
	+\frac{\tilde\bfm}{\tilde\rho}\cdot\LRp{\Delta\bbD+\Delta\bbD^T-\frac{2}{3}\trace(\Delta\bbD)\bbI}\\
\end{align*}
\begin{align*}
\pd{\bbM_{\bbD}(\tilde W)}{W}\Delta W&:=
	\Delta\bbD\\
\pd{\bfM_{\bfq}(\tilde W)}{W}\Delta W&:=
	\frac{Pr}{C_p}\Delta\bfq\\
\pd{\bfG_{\bbD}(\tilde W)}{W}\Delta W&:=
	\frac{\Delta\bfm}{\tilde\rho}-\frac{\tilde\bfm}{\tilde\rho^2}\Delta\rho\\
\pd{G_{\bfq}(\tilde W)}{W}\Delta W&:=
	-\left(\frac{\Delta E-\frac{1}{2\tilde\rho}\Delta\bfm\cdot\tilde\bfm
	-\frac{1}{2\tilde\rho}\tilde\bfm\cdot\Delta\bfm
	+\frac{1}{2\tilde\rho^2}\tilde\bfm\cdot\tilde\bfm\Delta\rho}{C_v\tilde\rho}
	\right.\\&\quad\left.
	-\frac{\tilde E-\frac{1}{2\tilde\rho}\tilde\bfm\cdot\tilde\bfm}{C_v\tilde\rho^2}\Delta\rho\right)
\,.
\end{align*}

\section{Entropy Variables}
Now we wish to do a change of variables to entropy variables:
\begin{align*}
V_c&=\frac{-E+(E-\frac{1}{2\rho}\bfm\cdot\bfm)\LRp{\gamma+1-\ln\LRs{\frac{(\gamma-1)(E-\frac{1}{2\rho}\bfm\cdot\bfm)}{\rho^\gamma}}}}
{E-\frac{1}{2\rho}\bfm\cdot\bfm}\\
\bfV_m&=\frac{\bfm}{E-\frac{1}{2\rho}\bfm\cdot\bfm}\\
V_e&=\frac{-\rho}{E-\frac{1}{2\rho}\bfm\cdot\bfm}\,,
\end{align*}
with reverse mapping:
\begin{align*}
\rho&=-\alpha V_e\\
\bfm&=\alpha\bfV_m\\
E&=\alpha\LRp{1-\frac{1}{2V_e}\bfV_m\cdot\bfV_m}\,,
\end{align*}
where 
\[
\alpha(V_c,\bfV_m,V_e)=\LRs{\frac{\gamma-1}{(-V_e)^\gamma}}^{\frac{1}{\gamma-1}}\exp\LRs{\frac{-\gamma+V_c-\frac{1}{2V_e}\bfV_m\cdot\bfV_m}{\gamma-1}}\,.
\]

The nonlinear terms are:
\begin{align*}
C_c&:=-\alpha V_e\\
\bfC_m&:=\alpha\bfV_m\\
C_e&:=\alpha\LRp{1-\frac{1}{2V_e}\bfV_m\cdot\bfV_m}\\
\bfF_c&=\alpha\bfV_m\\
\mathbb{F}_m&=\alpha\LRp{-\frac{\bfV_m\otimes\bfV_m}{V_e}+(\gamma-1)\bbI}\\
\bfF_e&=\alpha\frac{\bfV_m}{V_e}\LRp{\frac{1}{2V_e}\bfV_m\cdot\bfV_m-\gamma}\\
\bfK_c&:=\boldsymbol 0\\
\bbK_m&:=\LRp{\bbD+\bbD^T-\frac{2}{3}\trace(\bbD)\bbI}\\
\bfK_e&:=-\bfq+\frac{\bfV_m}{V_e}\cdot\LRp{\bbD+\bbD^T-\frac{2}{3}\trace(\bbD)\bbI}\\
\bbM_{\bbD}&:=\bbD\\
\bfM_{\bfq}&:=\frac{Pr}{C_p}\bfq\\
\bfG_{\bbD}&:=-\frac{\bfV_m}{V_e}\\
G_{\bfq}&:=\frac{1}{C_vV_e}\,.
\end{align*}

\subsection{Linearized Terms}
And the linearized terms for entropy variables are:
\begin{align*}
\pd{C_c(\tilde W)}{W}\Delta W&:=
	-\tilde V_e\pd{\alpha(\tilde W)}{W}\Delta W-\alpha(\tilde W)\Delta V_e
	\\
\pd{\bfC_m(\tilde W)}{W}\Delta W&:=
	\tilde\bfV_m\pd{\alpha(\tilde W)}{W}\Delta W+\alpha(\tilde W)\Delta\bfV_m
	\Delta\rho\tilde\bfu+\tilde\rho\Delta\bfu
	\\
\pd{C_e(\tilde W)}{W}\Delta W&:=
	\LRp{1-\frac{1}{2\tilde V_e}\tilde\bfV_m\cdot\tilde\bfV_m}\pd{\alpha(\tilde W)}{W}\Delta W
	\\&\quad
	-\alpha(\tilde W)\frac{1}{\tilde V_e}\tilde\bfV_m\cdot\Delta\bfV_m
	+\alpha(\tilde W)\frac{1}{2\tilde V_e^2}\tilde\bfV_m\cdot\tilde\bfV_m\Delta V_e
	\\
\pd{\bfF_c(\tilde W)}{W}\Delta W&:=
	\tilde\bfV_m\pd{\alpha(\tilde W)}{W}\Delta W+\alpha(\tilde W)\Delta\bfV_m
	\\
\pd{\mathbb{F}_m(\tilde W)}{W}\Delta W&:=
	\LRp{-\frac{\tilde\bfV_m\otimes\tilde\bfV_m}{\tilde V_e}+(\gamma-1)\bbI}\pd{\alpha(\tilde W)}{W}\Delta W
	\\&\quad
	+\alpha(\tilde W)\LRp{
	-\frac{\Delta\bfV_m\otimes\tilde\bfV_m}{\tilde V_e}
	-\frac{\tilde\bfV_m\otimes\Delta\bfV_m}{\tilde V_e}
	+\frac{\tilde\bfV_m\otimes\tilde\bfV_m}{\tilde V_e^2}\Delta V_e
	}
	\\
\pd{\bfF_e(\tilde W)}{W}\Delta W&:=
	\frac{\tilde\bfV_m}{\tilde V_e}
	\LRp{\frac{1}{2\tilde V_e}\tilde\bfV_m\cdot\tilde\bfV_m-\gamma}\pd{\alpha(\tilde W)}{W}\Delta W
	\\&\quad
	+\alpha(\tilde W)\left(
	\frac{\Delta\bfV_m}{\tilde V_e}\LRp{\frac{1}{2\tilde V_e}\tilde\bfV_m\cdot\tilde\bfV_m-\gamma}
	\right.\\&\quad\quad\left.
	-\frac{\tilde\bfV_m}{V_e^2}\LRp{\frac{1}{2\tilde V_e}\tilde\bfV_m\cdot\tilde\bfV_m-\gamma}\Delta V_e
	\right.\\&\quad\quad\left.
	+\frac{\tilde\bfV_m}{\tilde V_e}\LRp{
	\frac{1}{\tilde V_e}\tilde\bfV_m\cdot\Delta\bfV_m
	-\frac{1}{2\tilde V_e^2}\tilde\bfV_m\cdot\tilde\bfV_m\Delta V_e
	}
	\right)
	\\
\end{align*}
\begin{align*}
\pd{\bfK_c(\tilde W)}{W}\Delta W&:=
	\boldsymbol 0\\
\pd{\bbK_m(\tilde W)}{W}\Delta W&:=
	\LRp{\Delta\bbD+\Delta\bbD^T-\frac{2}{3}\trace(\Delta\bbD)\bbI}
	\\
\pd{\bfK_e(\tilde W)}{W}\Delta W&:=
	-\Delta\bfq+
	\LRp{\frac{\Delta\bfV_m}{\tilde V_e}-\frac{\tilde\bfV_m}{\tilde V_e^2}\Delta V_e}
	\cdot\LRp{\tilde\bbD+\tilde\bbD^T-\frac{2}{3}\trace(\tilde\bbD)\bbI}
	\\&\quad
	+\frac{\tilde\bfV_m}{\tilde V_e}\cdot\LRp{\Delta\bbD+\Delta\bbD^T-\frac{2}{3}\trace(\Delta\bbD)\bbI}\\
\pd{\bbM_{\bbD}(\tilde W)}{W}\Delta W&:=
	\Delta\bbD
	\\
\pd{\bfM_{\bfq}(\tilde W)}{W}\Delta W&:=
	\frac{Pr}{C_p}\Delta\bfq
	\\
\pd{\bfG_{\bbD}(\tilde W)}{W}\Delta W&:=
	-\LRp{\frac{\Delta\bfV_m}{\tilde V_e}-\frac{\tilde\bfV_m}{\tilde V_e^2}\Delta V_e}
	\\
\pd{G_{\bfq}(\tilde W)}{W}\Delta W&:=
	-\frac{1}{C_vV_e^2}\Delta V_e
	\\
% 	\,.
% \end{align*}
% \begin{align*}
\pd{\alpha(\tilde W)}{W}\Delta W&=
\LRs{\frac{\gamma-1}{(-\tilde V_e)^\gamma}}^{\frac{2-\gamma}{\gamma-1}}
\gamma(-\tilde V_e)^{-(\gamma+1)}
\exp\LRs{\frac{-\gamma+\tilde V_c-\frac{1}{2\tilde V_e}\tilde\bfV_m\cdot\tilde\bfV_m}{\gamma-1}}
\Delta V_e\\
&+\LRs{\frac{\gamma-1}{(-\tilde V_e)^\gamma}}^{\frac{1}{\gamma-1}}
\exp\LRs{\frac{-\gamma+\tilde V_c-\frac{1}{2\tilde V_e}\tilde\bfV_m\cdot\tilde\bfV_m}{\gamma-1}}
\frac{1}{\gamma-1}\\
&\LRp{\Delta V_c
-\frac{1}{\tilde V_e}\tilde\bfV_m\cdot\Delta\bfV_m
+\frac{1}{2\tilde V_e^2}\tilde\bfV_m\cdot\tilde\bfV_m\Delta V_e
}\,.
\end{align*}


\section{Numerical Experiments}
We perform a couple numerical experiments to compare the different formulations.
In Chapter \ref{sec:compressible} we used a incrementally decreased $\mu$ with every refinement step 
as this approach was found to produce cleaner refinement patterns; here we hold $\mu$ constant
for each problem to show that it is still possible to arrive at a converged solution, but
we end up with a less desirable final refinement pattern.

\subsection{Sod Shock Tube}
% Re=1e5, p=2, delta_p=2, nlMaxIters=10, NSDecoupled
We repeat the Sod shock tube problem described in Chapter \ref{sec:compressible} with $\mu=10^{-5}$,
$p=2$, $\Delta p=2$, and the NSDecoupled norm.
We omit plots of velocity and pressure as they don't really contribute anything new to the comparisons.
Comparing figures (\ref{fig:SodPrimitive} - \ref{fig:SodEntropy}), it seems that
primitive and entropy variables are of similiar quality, at least by the eyeball norm.
Entropy variables, on the other hand, suffer from much more extreme overshoots and 
undershoots compared to the other formulations.

\begin{figure}[ht]
\centering
\begin{subfigure}[t]{\textwidth}
\centering
\includegraphics[width=\textwidth]{Dissertation/Sod/FormulationComparison/primitive-den.pdf}
\caption{Density}
\end{subfigure}
% \begin{subfigure}[t]{0.7\textwidth}
% \centering
% \includegraphics[width=\textwidth]{Dissertation/Sod/FormulationComparison/primitive-vel.pdf}
% \caption{Velocity}
% \end{subfigure}
% \begin{subfigure}[t]{0.7\textwidth}
% \centering
% \includegraphics[width=\textwidth]{Dissertation/Sod/FormulationComparison/primitive-pres.pdf}
% \caption{Pressure}
% \end{subfigure}
\begin{subfigure}[t]{0.9\textwidth}
\centering
\includegraphics[width=\textwidth]{Sod/FormulationComparison/Form0Mesh15.png}
\caption{Final mesh colored by $\rho$}
\end{subfigure}
\caption{Sod problem with primitive variables}
\label{fig:SodPrimitive}
\end{figure}

\begin{figure}[ht]
\centering
\begin{subfigure}[t]{\textwidth}
\centering
\includegraphics[width=\textwidth]{Dissertation/Sod/FormulationComparison/conservation-den.pdf}
\caption{Density}
\end{subfigure}
% \begin{subfigure}[t]{0.7\textwidth}
% \centering
% \includegraphics[width=\textwidth]{Dissertation/Sod/FormulationComparison/conservation-vel.pdf}
% \caption{Velocity}
% \end{subfigure}
% \begin{subfigure}[t]{0.7\textwidth}
% \centering
% \includegraphics[width=\textwidth]{Dissertation/Sod/FormulationComparison/conservation-pres.pdf}
% \caption{Pressure}
% \end{subfigure}
\begin{subfigure}[t]{0.9\textwidth}
\centering
\includegraphics[width=\textwidth]{Sod/FormulationComparison/Form1Mesh15.png}
\caption{Final mesh colored by $\rho$}
\end{subfigure}
\caption{Sod problem with conservation variables}
\label{fig:SodConservation}
\end{figure}

\begin{figure}[ht]
\centering
\begin{subfigure}[t]{\textwidth}
\centering
\includegraphics[width=\textwidth]{Dissertation/Sod/FormulationComparison/entropy-den.pdf}
\caption{Density}
\end{subfigure}
% \begin{subfigure}[t]{0.7\textwidth}
% \centering
% \includegraphics[width=\textwidth]{Dissertation/Sod/FormulationComparison/entropy-vel.pdf}
% \caption{Velocity}
% \end{subfigure}
% \begin{subfigure}[t]{0.7\textwidth}
% \centering
% \includegraphics[width=\textwidth]{Dissertation/Sod/FormulationComparison/entropy-pres.pdf}
% \caption{Pressure}
% \end{subfigure}
\begin{subfigure}[t]{0.9\textwidth}
\centering
\includegraphics[width=\textwidth]{Sod/FormulationComparison/Form2Mesh15.png}
\caption{Final mesh colored by $V_c$}
\end{subfigure}
\caption{Sod problem with entropy variables}
\label{fig:SodEntropy}
\end{figure}

\subsection{Noh Implosion}
% Re=1e3, p=2, delta_p=2, nlMaxIters=10, timeslabs
We repeat the Noh problem from before with $\mu=10^{-3}$, $p=2$, $\Delta p=2$, and the NSDecoupled norm.
In Chapter \ref{sec:compressible} we simulated a half domain with a symmetry boundary condition
at the origin; here we compute the full domain.
The other difference is that this simulation was computed as a series of four time slabs 
rather than as one monolithic computation.
This means that the $[0,\frac{1}{4}]$ time slab was computed for 8 adaptive refinement steps
then the final solution was projected onto the $[\frac{1}{4},\frac{1}{2}]$ time slab as an 
initial condition.
This was repeated until we arrived at the $[\frac{3}{4},1]$ time slab, where the density traces
in Figure \ref{fig:NohDen} are taken.
We see more unwanted refinements in this computation compared to Chapter \ref{sec:compressible}
due to the spurious shock patterns that develop on coarse meshes.
We are not able to compare the entropy formulation for this problem since the initial conditions 
contain infinities under this formulation.
Again, primitive and conservation variables produce similar results.

\section{Conclusion}
The conclusion then is that since DPG already produces a symmetric, positive-definite stiffness
matrix, there is no reason to prefer entropy variables.
The choice between primitive and conservation variables depends on which one
is easier to implement as they will both give similar results.
We decided to stick with primitive variables as they were slightly simpler and
less nonlinear.

\begin{figure}[ht]
\centering
\includegraphics[width=\textwidth]{Noh/FormulationComparison/den9.pdf}
\caption{Density at final time}
\label{fig:NohDen}
\end{figure}

\begin{figure}[ht]
\centering
\begin{subfigure}[t]{0.9\textwidth}
\centering
\includegraphics[width=\textwidth]{Noh/FormulationComparison/Form0Mesh8.png}
\caption{Final mesh with primitive variables}
\end{subfigure}
\begin{subfigure}[t]{0.9\textwidth}
\centering
\includegraphics[width=\textwidth]{Noh/FormulationComparison/Form1Mesh8.png}
\caption{Final mesh with conservation variables}
\end{subfigure}
\caption{Noh meshes colored by $\rho$}
\label{fig:Noh}
\end{figure}

\end{document}
