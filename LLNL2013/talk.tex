\documentclass{beamer}
\usepackage{amsmath, amssymb, amsthm, mathtools}
\usepackage{graphicx}
\usepackage{cancel}
\usepackage{color}
\usepackage{caption}
\usepackage{subcaption}
%\usepackage{subfigure}
\usepackage{pgf,tikz}
\usetikzlibrary{arrows}

\captionsetup{compatibility=false}

\newcommand{\bs}[1]{\boldsymbol{#1}}
\newcommand{\norm}[1]{\left\| #1 \right\|}
\newcommand{\snorm}[1]{\left| #1 \right|}
\newcommand{\LRp}[1]{\left( #1 \right)}
\newcommand{\LRs}[1]{\left[ #1 \right]}
\newcommand{\LRc}[1]{\left\{ #1 \right\}}
\newcommand{\LRa}[1]{\left\langle #1 \right\rangle}
\newcommand{\LRb}[1]{\left| #1 \right|}
\newcommand{\Grad}{\ensuremath{\nabla}}
\newcommand{\Gradxt}{\ensuremath{\nabla_{xt}}}
\newcommand{\Div}{\ensuremath{\nabla\cdot}}
\newcommand{\Divxt}{\ensuremath{\nabla_{xt}\cdot}}
\newcommand{\Curl}{\ensuremath{\nabla\times}}
\newcommand{\bfH}{\mbox{\boldmath $H$}}
\newcommand{\bfsigma}{\boldsymbol\sigma}
\newcommand{\bfvarsigma}{\boldsymbol\varsigma}
\newcommand{\bftau}{\boldsymbol\tau}
\newcommand{\bfbeta}{\boldsymbol\beta}
\newcommand{\bflambda}{\boldsymbol\lambda}
\newcommand{\bfpsi}{\boldsymbol\psi}
\newcommand{\bfu}{\boldsymbol u}
\newcommand{\bfv}{\boldsymbol v}
\newcommand{\bfV}{\boldsymbol V}
\newcommand{\bfZ}{\boldsymbol Z}
\newcommand{\bfz}{\boldsymbol z}
\newcommand{\bfW}{\boldsymbol W}
\newcommand{\bfw}{\boldsymbol w}
\newcommand{\bfm}{\boldsymbol m}
\newcommand{\bfM}{\boldsymbol M}
\newcommand{\bbM}{\mathbb{M}}
\newcommand{\bfq}{\boldsymbol q}
\newcommand{\bfU}{\boldsymbol U}
\newcommand{\bfS}{\boldsymbol S}
\newcommand{\bbS}{\mathbb{S}}
\newcommand{\bbD}{\mathbb{D}}
\newcommand{\bfK}{\boldsymbol K}
\newcommand{\bbK}{\mathbb{K}}
\newcommand{\bfn}{\boldsymbol n}
\newcommand{\bff}{\boldsymbol f}
\newcommand{\bfF}{\boldsymbol F}
\newcommand{\bbF}{\mathbb{F}}
\newcommand{\bfg}{\boldsymbol g}
\newcommand{\bfG}{\boldsymbol G}
\newcommand{\bfC}{\boldsymbol C}
\newcommand{\bft}{\boldsymbol t}
\newcommand{\bfT}{\boldsymbol T}
\newcommand{\bfI}{\boldsymbol I}
\newcommand{\bbI}{\mathbb{I}}
\newcommand{\bfx}{\boldsymbol x}
\newcommand{\uh}{\widehat{u}}
\newcommand{\fnh}{\widehat{f}_n}
\newcommand{\LQ}{L^2\LRp{Q}}
\newcommand{\LK}{L^2\LRp{K}}
\newcommand{\LVecK}{\mathbf{L}^2\LRp{K}}
\newcommand{\LVecQ}{\mathbf{L}^2\LRp{Q}}
\newcommand{\HdivK}{\bfH(\text{div},K)}
\newcommand{\HdivOmega}{\bfH(\text{div},\Omega)}
% \newcommand{\HdivOmegaLT}{\bfH(\text{div},\Omega)\times L^2([0,T])}
\newcommand{\HdivQ}{\bfH(\text{div}_{xt},Q)}
\newcommand{\HOneK}{H^{1}(K)}
\newcommand{\HOneVecK}{\bfH^{1}(K)}
\newcommand{\HOneQ}{H^{1}(Q)}
\newcommand{\HOneOmegah}{H^{-1}(\Omega_h)}
\newcommand{\HdivOmegah}{\bfH(\text{div},\Omega_h)}
\newcommand{\vdeltau}{v_{\delta\bs u_h}}
\newcommand{\taudeltau}{\bftau_{\delta\bs u_h}}
\newcommand{\ip}[1]{\left\langle #1 \right\rangle}
\newcommand{\pd}[2]{\frac{\partial#1}{\partial#2}}
\newcommand{\pt}[1]{\frac{\partial#1}{\partial t}}
\newcommand{\ppd}[2]{\frac{\partial^2#1}{\partial#2^2}}
\newcommand{\pdd}[3]{\frac{\partial^2#1}{\partial#2\partial#3}}
\newcommand{\der}[2]{\frac{\mathrm{d}#1}{\mathrm{d}#2}}
\newcommand{\Oh}{\Omega_h}
\newcommand{\jump}[1] {\ensuremath{\LRs{\![#1]\!}}}
\newcommand{\Gh}{\Gamma_h}
\newcommand{\mcU}{\mathcal{U}}
\newcommand{\mcUh}{\hat{\mathcal{U}}}
\newcommand{\LOmega}{L^2\LRp{\Omega_h}}

\newcommand{\eqnref}[1]{\eqref{eq:#1}}

\DeclareMathOperator*{\argmin}{arg\,min}
\DeclareMathOperator*{\trace}{tr}

\def\arrtwo#1#2#3#4{\left[
\begin{array}{cc}
#1\; & #2\\
#3\; & #4\\
\end{array}
\right]}
\def\arrthree#1#2#3#4#5#6#7#8#9{\left[
\begin{array}{ccc}
#1\; & #2\; & #3\\
#4\; & #5\; & #6\\
#7\; & #8\; & #9\\
\end{array}
\right]}
\def\arrthreeone#1#2#3{\left[
\begin{array}{ccc}
#1\; & #2\; & #3\\
\end{array}
\right]}
\def\vecttwo#1#2{\left(
\begin{array}{c}
#1\\
#2\\
\end{array}
\right)}
\def\svecttwo#1#2{\left[
\begin{array}{c}
#1\\
#2\\
\end{array}
\right]}
\def\vectthree#1#2#3{\left(
\begin{array}{c}
#1\\
#2\\
#3\\
\end{array}
\right)}
\def\svectthree#1#2#3{\left[
\begin{array}{c}
#1\\
#2\\
#3\\
\end{array}
\right]}

\renewcommand{\arraystretch}{1.2}

\def\etal{{\it et al.~}}

\usetheme[secheader]{pecostalk}
\usepackage{comment}
% \usepackage{subfig}
\graphicspath{{figs/}}

\newcommand{\pecosbold}[1]{{\color{pecos2}{#1}}}
\newcommand{\pecosreallybold}[1]{{\color{pecos6}{#1}}}

\author[Truman. E. Ellis]{Truman E. Ellis}
\title[Multi-Resolution Viscosity Limiter]{Multi-Resolution Viscosity Limiter in
the BLAST High-Order Finite Element Hydrodynamics Code}
\institute{Institute for Computational and Engineering Sciences\\
The University of Texas at Austin}
\date{July 25, 2013}

\begin{document}

\begin{frame}
% \begin{center}
% \includegraphics[width=.8\linewidth]{grand_logo}\\
% \end{center}
\titlepage
\end{frame}
\begin{comment}
My name is Truman Ellis, and I am also working in Dr. Demkowicz's group on the
discontinuous Petrov-Galerkin method. My background is in aerospace
engineering and CFD, and my goal is to help develop DPG into an attractive
method for realistic problems in computational fluid dynamics. So the goal is
to work on the Euler equations and then build on Nate and Jesse's work on
laminar Navier-Stokes with some turbulence modeling.  But before I got into
that, we thought it would be useful to study a topic that keeps coming up from
the CFD community. Is DPG locally conservative?

This is a numerical characteristic close to the heart of many CFD
practitioners, and in order for DPG to gain a certain level of acceptance
among these circles, we need to address some of these concerns. There are also
some mathematically attractive reasons to pursue local conservation. The
Lax-Wendroff theorem guarantees that a convergent numerical solution to a
system of hyperbolic conservation laws will converge to the correct weak
solution. Also, we are focusing on local conservation on the
convection-diffusion equation as a proof of concept. We are working on
extending this work to more realistic flow simulations.
\end{comment}


%===============================================================================
% NEW SLIDE
%===============================================================================
\begin{frame}
\frametitle{A Summary of DPG}
Overview of Features
\begin{itemize}
\item Robust for singularly perturbed problems
\item Stable in the preasymptotic regime
\item Designed for adaptive mesh refinement
\end{itemize}
\bigskip

DPG is a minimum residual method:
\[
u_{h} = \underset{w_{h} \in U_{h}} \argmin \,\, \frac{1}{2}
\norm{Bw_{h}-l}_{V'}^{2}
\]
\vspace{-1em}
\[
\scalebox{1.8}{\ensuremath{\Updownarrow}}
\]
\vspace{-1em}
\[
b(u_h,R_V^{-1}B\delta u_h)
=l(R_V^{-1}B\delta u_h)
\quad\forall\delta u_h\in U_h
\]
where $v_{\delta u_h}:=R_V^{-1}B\delta u_h$ are the
\pecosbold{optimal test functions}.
\end{frame}
\begin{comment}
For the sake of avoiding a lot of repetition and making sure we all finish on
time, I'm going to offer an extremely condensed summary of DPG. Nate and Jesse
already covered this stuff with more rigor. So what are DPG's main selling
points? Why are we interested in applying it to complicated fluid problems
(eventually).
For one, DPG has proven very robust in the face of singularly
perturbed problems which holds promise for high Reynolds number flows.
You do not need a domain expert to craft well designed meshes for each new
problem. We are mathematically guaranteed to remain stable under very coarse
meshes while adaptively refining toward a solution.

And mathematically, how can we classify DPG? By derivations, it is a minimum
residual method. This means that through the choice of specific optimal test
functions we automatically minimize the residual in the dual norm of a
Hilbert space V. We also have a well-defined process to generate the optimal
test functions which is computationally feasible.
\end{comment}


% \bibliographystyle{plain}
% {\scriptsize
% \bibliography{../DPG.bib}
% }

\end{document}
