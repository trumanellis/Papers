\documentclass{beamer}
\usepackage{amsmath, amssymb, amsthm, mathtools}
\usepackage{graphicx}
\usepackage{cancel}
\usepackage{color}
\usepackage{caption}
\usepackage{subcaption}
%\usepackage{subfigure}
\usepackage{pgf,tikz}
\usetikzlibrary{arrows}

\captionsetup{compatibility=false}

\newcommand{\bs}[1]{\boldsymbol{#1}}
\newcommand{\norm}[1]{\left\| #1 \right\|}
\newcommand{\snorm}[1]{\left| #1 \right|}
\newcommand{\LRp}[1]{\left( #1 \right)}
\newcommand{\LRs}[1]{\left[ #1 \right]}
\newcommand{\LRc}[1]{\left\{ #1 \right\}}
\newcommand{\LRa}[1]{\left\langle #1 \right\rangle}
\newcommand{\LRb}[1]{\left| #1 \right|}
\newcommand{\Grad}{\ensuremath{\nabla}}
\newcommand{\Gradxt}{\ensuremath{\nabla_{xt}}}
\newcommand{\Div}{\ensuremath{\nabla\cdot}}
\newcommand{\Divxt}{\ensuremath{\nabla_{xt}\cdot}}
\newcommand{\Curl}{\ensuremath{\nabla\times}}
\newcommand{\bfH}{\mbox{\boldmath $H$}}
\newcommand{\bfsigma}{\boldsymbol\sigma}
\newcommand{\bfvarsigma}{\boldsymbol\varsigma}
\newcommand{\bftau}{\boldsymbol\tau}
\newcommand{\bfbeta}{\boldsymbol\beta}
\newcommand{\bflambda}{\boldsymbol\lambda}
\newcommand{\bfpsi}{\boldsymbol\psi}
\newcommand{\bfu}{\boldsymbol u}
\newcommand{\bfv}{\boldsymbol v}
\newcommand{\bfV}{\boldsymbol V}
\newcommand{\bfZ}{\boldsymbol Z}
\newcommand{\bfz}{\boldsymbol z}
\newcommand{\bfW}{\boldsymbol W}
\newcommand{\bfw}{\boldsymbol w}
\newcommand{\bfm}{\boldsymbol m}
\newcommand{\bfM}{\boldsymbol M}
\newcommand{\bbM}{\mathbb{M}}
\newcommand{\bfq}{\boldsymbol q}
\newcommand{\bfU}{\boldsymbol U}
\newcommand{\bfS}{\boldsymbol S}
\newcommand{\bbS}{\mathbb{S}}
\newcommand{\bbD}{\mathbb{D}}
\newcommand{\bfK}{\boldsymbol K}
\newcommand{\bbK}{\mathbb{K}}
\newcommand{\bfn}{\boldsymbol n}
\newcommand{\bff}{\boldsymbol f}
\newcommand{\bfF}{\boldsymbol F}
\newcommand{\bbF}{\mathbb{F}}
\newcommand{\bfg}{\boldsymbol g}
\newcommand{\bfG}{\boldsymbol G}
\newcommand{\bfC}{\boldsymbol C}
\newcommand{\bft}{\boldsymbol t}
\newcommand{\bfT}{\boldsymbol T}
\newcommand{\bfI}{\boldsymbol I}
\newcommand{\bbI}{\mathbb{I}}
\newcommand{\bfx}{\boldsymbol x}
\newcommand{\uh}{\widehat{u}}
\newcommand{\fnh}{\widehat{f}_n}
\newcommand{\LQ}{L^2\LRp{Q}}
\newcommand{\LK}{L^2\LRp{K}}
\newcommand{\LVecK}{\mathbf{L}^2\LRp{K}}
\newcommand{\LVecQ}{\mathbf{L}^2\LRp{Q}}
\newcommand{\HdivK}{\bfH(\text{div},K)}
\newcommand{\HdivOmega}{\bfH(\text{div},\Omega)}
% \newcommand{\HdivOmegaLT}{\bfH(\text{div},\Omega)\times L^2([0,T])}
\newcommand{\HdivQ}{\bfH(\text{div}_{xt},Q)}
\newcommand{\HOneK}{H^{1}(K)}
\newcommand{\HOneVecK}{\bfH^{1}(K)}
\newcommand{\HOneQ}{H^{1}(Q)}
\newcommand{\HOneOmegah}{H^{-1}(\Omega_h)}
\newcommand{\HdivOmegah}{\bfH(\text{div},\Omega_h)}
\newcommand{\vdeltau}{v_{\delta\bs u_h}}
\newcommand{\taudeltau}{\bftau_{\delta\bs u_h}}
\newcommand{\ip}[1]{\left\langle #1 \right\rangle}
\newcommand{\pd}[2]{\frac{\partial#1}{\partial#2}}
\newcommand{\pt}[1]{\frac{\partial#1}{\partial t}}
\newcommand{\ppd}[2]{\frac{\partial^2#1}{\partial#2^2}}
\newcommand{\pdd}[3]{\frac{\partial^2#1}{\partial#2\partial#3}}
\newcommand{\der}[2]{\frac{\mathrm{d}#1}{\mathrm{d}#2}}
\newcommand{\Oh}{\Omega_h}
\newcommand{\jump}[1] {\ensuremath{\LRs{\![#1]\!}}}
\newcommand{\Gh}{\Gamma_h}
\newcommand{\mcU}{\mathcal{U}}
\newcommand{\mcUh}{\hat{\mathcal{U}}}
\newcommand{\LOmega}{L^2\LRp{\Omega_h}}

\newcommand{\eqnref}[1]{\eqref{eq:#1}}

\DeclareMathOperator*{\argmin}{arg\,min}
\DeclareMathOperator*{\trace}{tr}

\def\arrtwo#1#2#3#4{\left[
\begin{array}{cc}
#1\; & #2\\
#3\; & #4\\
\end{array}
\right]}
\def\arrthree#1#2#3#4#5#6#7#8#9{\left[
\begin{array}{ccc}
#1\; & #2\; & #3\\
#4\; & #5\; & #6\\
#7\; & #8\; & #9\\
\end{array}
\right]}
\def\arrthreeone#1#2#3{\left[
\begin{array}{ccc}
#1\; & #2\; & #3\\
\end{array}
\right]}
\def\vecttwo#1#2{\left(
\begin{array}{c}
#1\\
#2\\
\end{array}
\right)}
\def\svecttwo#1#2{\left[
\begin{array}{c}
#1\\
#2\\
\end{array}
\right]}
\def\vectthree#1#2#3{\left(
\begin{array}{c}
#1\\
#2\\
#3\\
\end{array}
\right)}
\def\svectthree#1#2#3{\left[
\begin{array}{c}
#1\\
#2\\
#3\\
\end{array}
\right]}

\renewcommand{\arraystretch}{1.2}

\def\etal{{\it et al.~}}

\usetheme[secheader]{pecostalk}
\usepackage{comment}
% \usepackage{subfig}
\graphicspath{{figs/}}

\newcommand{\pecosbold}[1]{{\color{pecos2}{#1}}}
\newcommand{\pecosreallybold}[1]{{\color{pecos6}{#1}}}

\author[Truman. E. Ellis]{Truman E. Ellis}
\title[Locally Conservative DPG]{Locally Conservative Discontinuous
Petrov-Galerkin for Convection-Diffusion}
\institute{Institute for Computational and Engineering Sciences\\
The University of Texas at Austin}
\date{July 25, 2013}

\begin{document}

\begin{frame}
\begin{center}
\includegraphics[width=.8\linewidth]{grand_logo}\\
\end{center}
\titlepage
\end{frame}
\begin{comment}
My name is Truman Ellis, and I am also working in Dr. Demkowicz's group on the
discontinuous Petrov-Galerkin method. My background is in aerospace
engineering and CFD, and my goal is to help develop DPG into an attractive
method for realistic problems in computational fluid dynamics. So the goal is
to work on the Euler equations and then build on Nate and Jesse's work on
laminar Navier-Stokes with some turbulence modeling.  But before I got into
that, we thought it would be useful to study a topic that keeps coming up from
the CFD community. Is DPG locally conservative?

This is a numerical characteristic close to the heart of many CFD
practitioners, and in order for DPG to gain a certain level of acceptance
among these circles, we need to address some of these concerns. There are also
some mathematically attractive reasons to pursue local conservation. The
Lax-Wendroff theorem guarantees that a convergent numerical solution to a
system of hyperbolic conservation laws will converge to the correct weak
solution. Also, we are focusing on local conservation on the
convection-diffusion equation as a proof of concept. We are working on
extending this work to more realistic flow simulations.
\end{comment}


%===============================================================================
% NEW SLIDE
%===============================================================================
\begin{frame}
\frametitle{A Summary of DPG}
Overview of Features
\begin{itemize}
\item Robust for singularly perturbed problems
\item Stable in the preasymptotic regime
\item Designed for adaptive mesh refinement
\end{itemize}
\bigskip

DPG is a minimum residual method:
\[
u_{h} = \underset{w_{h} \in U_{h}} \argmin \,\, \frac{1}{2}
\norm{Bw_{h}-l}_{V'}^{2}
\]
\vspace{-1em}
\[
\scalebox{1.8}{\ensuremath{\Updownarrow}}
\]
\vspace{-1em}
\[
b(u_h,R_V^{-1}B\delta u_h)
=l(R_V^{-1}B\delta u_h)
\quad\forall\delta u_h\in U_h
\]
where $v_{\delta u_h}:=R_V^{-1}B\delta u_h$ are the
\pecosbold{optimal test functions}.
\end{frame}
\begin{comment}
For the sake of avoiding a lot of repetition and making sure we all finish on
time, I'm going to offer an extremely condensed summary of DPG. Nate and Jesse
already covered this stuff with more rigor. So what are DPG's main selling
points? Why are we interested in applying it to complicated fluid problems
(eventually).
For one, DPG has proven very robust in the face of singularly
perturbed problems which holds promise for high Reynolds number flows.
You do not need a domain expert to craft well designed meshes for each new
problem. We are mathematically guaranteed to remain stable under very coarse
meshes while adaptively refining toward a solution.

And mathematically, how can we classify DPG? By derivations, it is a minimum
residual method. This means that through the choice of specific optimal test
functions we automatically minimize the residual in the dual norm of a
Hilbert space V. We also have a well-defined process to generate the optimal
test functions which is computationally feasible.
\end{comment}


%===============================================================================
% NEW SLIDE
%===============================================================================
\begin{frame}
\frametitle{DPG for Convection-Diffusion}
Start with the strong-form PDE.
\[
\nabla\cdot(\bfbeta u)-\epsilon\Delta u = g
\]
Rewrite as a system of first-order equations.
\begin{align*}
\nabla\cdot(\bfbeta u-\bfsigma)&=g\\
\frac{1}{\epsilon}\bfsigma-\nabla u&=\boldsymbol0
\end{align*}
Multiply by test functions and integrate by parts over each element, $K$.
\begin{align*}
-(\bfbeta u-\bfsigma,\nabla v)_K+((\bfbeta
u-\bfsigma)\cdot\mathbf{n},v)_{\partial K}&=(g,v)_K\\
\frac{1}{\epsilon}(\bfsigma,\bftau)_K+(u,\nabla\cdot\bftau)_K
-(u,\tau_n)_{\partial K}&=0
\end{align*}
Use the ultraweak (DPG) formulation to obtain bilinear form $b(u,v)=l(v)$.
\begin{align*}
-(\bfbeta u-\bfsigma,\nabla v)_K&+(\hat f,v)_{\partial K}
+ \frac{1}{\epsilon}(\bfsigma,\bftau)_K\\
&+(u,\nabla\cdot\bftau)_K
-(\hat u,\tau_n)_{\partial K}=(g,v)_K
\end{align*}
\end{frame}
\begin{comment}
With that ultra-brief refresher, we can now apply DPG to the
convection-diffusion equation. We prefer to work with systems of first order
equations. Then, multiplying the top equation by a scalar valued test
function, v, and the second by a vector valued test function tau, we can
integrate by parts over each element, K. Combining the two equations, we get
our bilinear form for convection diffusion. We seek the field variable, u and
sigma in L2, but that leaves their traces undefined. So, in a manner similar
to the hybridized DG method, we define new unknowns for our traces and fluxes,
u hat and f hat.
\end{comment}


%===============================================================================
% NEW SLIDE
%===============================================================================
\begin{frame}
\frametitle{Local Conservation}
The local conservation law in convection diffusion is
\[
\int_{\partial K}\hat f=\int_K g\,,
\]
which is equivalent to having $\mathbf{v}_K:=\{v,\bftau\}=\{1_K,\boldsymbol0\}$ in the test space.
In general, this is not satisfied by the optimal test functions.
Following Moro et al\textsuperscript{\cite{MoroNguyenPeraire11}}, we
can enforce this condition with Lagrange multipliers:
\begin{align*}
L(u_h,\bflambda) = \frac{1}{2}\norm{R_V^{-1}(Bu_h-l)}_V^2
-\sum_K\lambda_K\underbrace{\langle Bu_h-l,\mathbf{v}_K\rangle}_
{\langle\hat f, 1_K\rangle_{\partial K}-\langle g,1_K\rangle_K}\,,
\end{align*}
where $\bflambda=\{\lambda_1,\cdots,\lambda_N\}$.
\end{frame}
\begin{comment}
So what does local conservation mean for the convection-diffusion equation. We
want the integral of our fluxes over the element faces to be balanced by any
source terms in the RHS. This is equivalent to having one in the test space.
You can see this if you look at the bilinear form on the previous slide and
plug one and zero in as test functions. Unfortunately the optimal test
functions do not always span constants. It turns out we can augment our test
space with constants through the use of Lagrange multipliers. So taking a
couple of steps backward in the abstract form, we can augment our system with
the Lagrange multipliers enforcing local conservation.
\end{comment}


%===============================================================================
% NEW SLIDE
%===============================================================================
\begin{frame}
\frametitle{Local Conservation}
Finding the critical points of $L(u,\bflambda)$, we get the following
equations.
\begin{align*}
\frac{\partial L(u_h,\bflambda)}{\partial u_h}&=b(u_h,R_V^{-1}B\delta u_h)
-l(R_V^{-1}B\delta u_h)\\
&{\color{red}-\sum_K\lambda_K b(\delta
u_h,\mathbf{v}_K)}=0\quad\forall\delta u_h\in U_h
\end{align*}
\[
\frac{\partial
L(u_h,\bflambda)}{\partial\lambda_K}=-b(u_h,\mathbf{v}_K)+l(\mathbf{v}_K)=0\quad\forall
K
\]
A few consequences:
\begin{itemize}
\item We've turned our minimization problem into a saddlepoint problem.
% \item New $\lambda_K$ DOFs can be statically condensed out.
\item Only need to find the optimal test function in the orthogonal complement
of constants. % Backup slide
\end{itemize}
\end{frame}
\begin{comment}
Now, proceeding forward again and setting the derivatives to zero, we have
turned our minimization problem into a saddle point problem. Note that we have
added extra DOFs equal to the number of mesh elements, but the structure of
the problem allows these to be statically condensed out. The most interesting
consequence of this modification (apart from enforcing local conservation) is
that we can modify the search space for our optimal test functions to be
the orthogonal complement of constants.
\end{comment}


%===============================================================================
% NEW SLIDE
%===============================================================================
\begin{frame}
\frametitle{Optimal Test Functions}
For each $\mathbf{u}=\{u,\bfsigma,\hat u,\hat f\}\in\mathbf{U}_h$, find
$\mathbf{v}_{\mathbf{u}}=\{v_\mathbf{u},\bftau_\mathbf{u}\}\in\mathbf{V}$ such that
\[
(\mathbf{v_u},\mathbf{w})_\mathbf{V}=b(\mathbf{u},\mathbf{w})\quad\forall\mathbf{w}\in\mathbf{V}
\]
where $\mathbf{V}$ becomes $\mathbf{V}_{p+\Delta p}$ in order to make this
computationally tractable.

We recently developed this modification to the \emph{robust test norm}
\textsuperscript{\cite{ChanHeuerThanhDemkowicz2012}} which behaves better in
the presence of singularities.
\begin{minipage}[t][1.5in]{\textwidth}
\begin{align*}
\norm{(v,\bftau)}^2_{\mathbf{V},\Omega_h}&=
\norm{\min\left\{\frac{1}{\sqrt{\epsilon}},\frac{1}{\sqrt{|K|}}\right\}\bftau}^2
+\norm{\nabla\cdot\bftau-\bfbeta\cdot\nabla v}^2\\
&+\norm{\bfbeta\cdot\nabla v}^2+\epsilon\norm{\nabla v}^2
\underbrace{\color{red}{
\begin{minipage}[c][0.3in][c]{0.9in}$
\only<1>{
\hspace{1.5ex}
+\norm{v}^2
}
\only<2>{
+\left(\frac{1}{|K|}\int_Kv\right)^2
}$
\end{minipage}
}}_{\only<1>{\text{No longer necessary}}\only<2>{\text{Zero mean term}}}
\end{align*}
\end{minipage}
\end{frame}
\begin{comment}
So how does this affect our search for optimal test functions? The process to
compute optimal test functions is as follows. For each trial function we are
considering, we want to find the test function in an enriched space that
satisfies the following relation - that the inner product of vu with w equals
the bilinear form acting on u and w for all w in V.

The choice of norm on V can significantly affect the robustness of the
solution. Unfortunately, the optimal test norm is not localizable, and for
convection diffusion, the quasi-optimal test norm has boundary layers arising
from the adjoint and thus has approximability issues. Fortunately a
couple of collaborators developed a robust test norm for convection-diffusion.
Sadly, this norm also has its issues as the final zero order term in the
expression is somewhat troublesome. (???)

Why can we replace the L2 term with the zero mean term?
Why is zero mean more friendly than L2?
- Not mesh dependent
- No boundary layers
\end{comment}


%===============================================================================
% NEW SLIDE
%===============================================================================
\begin{frame}
\frametitle{Erickson-Johnson Problem}
On domain $\Omega=[0,1]^2$, with $\beta=(1,0)^T$, $f=0$ and boundary
conditions
\[
\hat f=u_0,\quad\beta_n\le0\,,\quad\quad\hat u=0,\quad\beta_n > 0
\]
Separation of variabes gives an analytic solution
\[
u(x,y)=C_0+\sum_{n=1}^\infty C_n
\frac{\exp(r_2(x-1))-\exp(r_1(x-1))}{r_1\exp(-r_2)-r_2\exp(-r_1)}
\cos(n\pi y)
\]
\vspace{-5ex}
\begin{columns}[b]
\begin{column}{0.5\textwidth}
\begin{figure}[t]
\centering
\includegraphics[width=0.9\textwidth]{Erickson/modifiedError.pdf}

\end{figure}
\end{column}
\begin{column}{0.5\textwidth}
\begin{figure}[t]
\centering
\includegraphics[width=0.9\textwidth]{Erickson/modifiedFlux.pdf}

\end{figure}
\end{column}
\end{columns}
\end{frame}
\begin{comment}
\end{comment}


%===============================================================================
% NEW SLIDE
%===============================================================================
\begin{frame}
\frametitle{Skewed Convection-Diffusion Problem}
\only<1>{
After 8 refinements, $\epsilon=10^{-4}$, $\bfbeta=(2,1)^T$
\begin{columns}
\begin{column}{0.49\textwidth}
\begin{figure}
\centering
\includegraphics[width=1.0\textwidth]{Confusion/modified8c.png}
\end{figure}
\end{column}
\begin{column}{0.49\textwidth}
\begin{figure}
\centering
\includegraphics[width=1.0\textwidth]{Confusion/modified8c_mesh.png}
\end{figure}
\end{column}
\end{columns}
}
\only<2>{
\begin{figure}[t]
\centering
\includegraphics[width=0.7\textwidth]{Confusion/modifiedFlux.pdf}
\end{figure}
}
\end{frame}
\begin{comment}
\end{comment}


%===============================================================================
% NEW SLIDE
%===============================================================================
\begin{frame}
\frametitle{Double Glazing Problem}
\only<1>{
After 5 refinements, $\epsilon=10^{-2}$, $\bfbeta=\left(
\begin{array}{c}
2(2y-1)(1-(2x-1)^2) \\
-2(2x-1)(1-(2y-1)^2) \\
\end{array}\right)$
\begin{columns}
\begin{column}{0.49\textwidth}
\begin{figure}
\centering
\includegraphics[width=1.0\textwidth]{DoubleGlazing/modified5nc.png}

Nonconservative
\end{figure}
\end{column}
\begin{column}{0.49\textwidth}
\begin{figure}
\centering
\includegraphics[width=1.0\textwidth]{DoubleGlazing/modified5c.png}

Conservative
\end{figure}
\end{column}
\end{columns}
}
\only<2>{
\begin{figure}[t]
\centering
\includegraphics[width=0.7\textwidth]{DoubleGlazing/modifiedFlux.pdf}
\end{figure}
}
\end{frame}
\begin{comment}
\end{comment}


%===============================================================================
% NEW SLIDE
%===============================================================================
\begin{frame}
\frametitle{Vortex Problem}
After 6 refinements, $\epsilon=10^{-4}$, $\bfbeta=(-y,x)^T$
\only<1>{
\begin{columns}
\begin{column}{0.49\textwidth}
\begin{figure}
\centering
\includegraphics[width=1.0\textwidth]{Vortex/modified6nc.png}

Nonconservative
\end{figure}
\end{column}
\begin{column}{0.49\textwidth}
\begin{figure}
\centering
\includegraphics[width=1.0\textwidth]{Vortex/modified6c.png}

Conservative
\end{figure}
\end{column}
\end{columns}
}
\only<2>{
\begin{figure}[t]
\centering
\includegraphics[width=0.7\textwidth]{Vortex/modifiedFlux.pdf}
\end{figure}
}
\end{frame}
\begin{comment}
\end{comment}


%===============================================================================
% NEW SLIDE
%===============================================================================
\begin{frame}
\frametitle{Wedge Problem}
After 16 refinements, $\epsilon=10^{-1}$, $\bfbeta=(1,0)^T$
\only<1>{
\begin{columns}
\begin{column}{0.49\textwidth}
\begin{figure}
\centering
\includegraphics[width=0.75\textwidth]{Wedge/modified16c.png}
\end{figure}
\end{column}
\begin{column}{0.49\textwidth}
\begin{figure}
\centering
\includegraphics[width=0.75\textwidth]{Wedge/modified16c_mesh.png}
\end{figure}
\end{column}
\end{columns}
}
\only<2>{
\begin{figure}[t]
\centering
\includegraphics[width=0.7\textwidth]{Wedge/modifiedFlux.pdf}
\end{figure}
}
\end{frame}
\begin{comment}
\end{comment}


%===============================================================================
% NEW SLIDE
%===============================================================================
\begin{frame}
\frametitle{Inner Layer Problem}
After 8 refinements, $\epsilon=10^{-6}$,
$\bfbeta=(\frac{\sqrt{3}}{2},\frac{1}{2})^T$, $\hat f=
\begin{cases}
1, & y <=0.2\\
0, & y > 0.2
\end{cases}$
\only<1>{
\begin{columns}
\begin{column}{0.49\textwidth}
\begin{figure}
\centering
\includegraphics[width=1.0\textwidth]{InnerLayer/modified8nc_mesh.png}

Nonconservative
\end{figure}
\end{column}
\begin{column}{0.49\textwidth}
\begin{figure}
\centering
\includegraphics[width=1.0\textwidth]{InnerLayer/modified8c_mesh.png}

Conservative
\end{figure}
\end{column}
\end{columns}
}
\only<2>{
\begin{figure}[t]
\centering
\includegraphics[width=0.7\textwidth]{InnerLayer/modifiedFlux.pdf}
\end{figure}
}
\end{frame}
\begin{comment}
\end{comment}


%===============================================================================
% NEW SLIDE
%===============================================================================
\begin{frame}
\frametitle{Discontinuous Source Problem}
After 8 refinements, $\epsilon=10^{-3}$,
$\bfbeta=(0.5,1)^T/\sqrt{1.25}$, $\hat g=
\begin{cases}
1, & y >=2x\\
0, & y <2x
\end{cases}$
\only<1>{
\begin{columns}
\begin{column}{0.49\textwidth}
\begin{figure}
\centering
\includegraphics[width=1.0\textwidth]{Discontinuous/modified8nc.png}

Nonconservative
\end{figure}
\end{column}
\begin{column}{0.49\textwidth}
\begin{figure}
\centering
\includegraphics[width=1.0\textwidth]{Discontinuous/modified8c.png}

Conservative
\end{figure}
\end{column}
\end{columns}
}
\only<2>{
\begin{figure}[t]
\centering
\includegraphics[width=0.7\textwidth]{Discontinuous/modifiedFlux.pdf}
\end{figure}
}
\end{frame}
\begin{comment}
\end{comment}


%===============================================================================
% NEW SLIDE
%===============================================================================
\begin{frame}
\frametitle{Hemker Problem}
After 8 refinements, $\epsilon=10^{-3}$,
$\bfbeta=(1,0)^T$
\only<1>{
\begin{columns}
\begin{column}{0.49\textwidth}
\begin{figure}
\centering
\includegraphics[width=1.0\textwidth]{Hemker/modified8nc.png}

\includegraphics[width=1.0\textwidth]{Hemker/modified8nc_mesh.png}

Nonconservative
\end{figure}
\end{column}
\begin{column}{0.49\textwidth}
\begin{figure}
\centering
\includegraphics[width=1.0\textwidth]{Hemker/modified8c.png}

\includegraphics[width=1.0\textwidth]{Hemker/modified8c_mesh.png}

Conservative
\end{figure}
\end{column}
\end{columns}
}
\only<2>{
\begin{figure}[t]
\centering
\includegraphics[width=0.7\textwidth]{Hemker/modifiedFlux.pdf}
\end{figure}
}
\end{frame}
\begin{comment}
\end{comment}


%===============================================================================
% NEW SLIDE
%===============================================================================
\begin{frame}
\frametitle{Inviscid Burgers' Equation}
\only<1>{
\vspace{-0.5em}
\[
\frac{\partial u}{\partial t}+u\frac{\partial u}{\partial
x}=0
\quad\Leftrightarrow\quad
\nabla_{x,t}\cdot\left(\begin{array}{c}
\frac{u^2}{2}\\
u
\end{array}\right)=0
\]
\vspace{-1.5em}
\begin{columns}
\begin{column}{0.49\textwidth}
\begin{figure}
\centering
\includegraphics[width=1.0\textwidth]{Burgers/graph8nc.png}

Nonconservative
\end{figure}
\end{column}
\begin{column}{0.49\textwidth}
\begin{figure}
\centering
\includegraphics[width=1.0\textwidth]{Burgers/graph8c.png}

Conservative
\end{figure}
\end{column}
\end{columns}
}
\only<2>{
\begin{figure}[t]
\centering
\includegraphics[width=0.7\textwidth]{Burgers/graphFlux.pdf}
\end{figure}
}
\end{frame}
\begin{comment}
This problem is slightly off-topic in that we are no longer dealing with the
convection-diffusion equation. Instead we consider the inviscid Burgers
equation which can be formulated as a nonlinear hyperbolic conservation law,
letting us relate directly to the Lax-Wendroff theorem. A common argument for
local conservation is that shock speed can be off if we don't have a
conservative method, and the Burgers' equation is often given as an example.
We choose to work with a space-time formulation where x follows the x axis and
time the y. So without going through the DPG derivation, let's just see how
our two methods behave. Apart from some slight differences in refinement
pattern, the two methods appear nearly identical.
\end{comment}


%===============================================================================
% NEW SLIDE
%===============================================================================
\begin{frame}
\frametitle{Summary}
What have we done?
\begin{itemize}
\item We've turned our minimization problem into a saddlepoint problem.
\item The change is computationally feasible.
\item Mathematically, it gets rid of troublesome term.
\end{itemize}

\vspace{2ex}
Does it make a difference?
\begin{itemize}
\item Enforcement changes refinement strategy.
\item Standard DPG is nearly conservative in practice.
\item Usually we get the same results with better conservation.
\item Some improvement on condition number for local solves.
\end{itemize}
\vspace{2ex}
\visible<2->{\pecosbold{We need to study the effect on real fluid dynamics.}}
\end{frame}


\bibliographystyle{plain}
{\scriptsize
\bibliography{../DPG.bib}
}

\end{document}
