\documentclass[12pt,letterpaper]{article}
\usepackage[margin=1.0in]{geometry}
\title{Cover Letter}
\author{Truman E. Ellis}
\date{}
\begin{document}
\maketitle

My doctoral research has introduced me to the very fascinating field of stabilized finite element methods.
In particular, my time in graduate school coincided with the introduction of the discontinuous Petrov-Galerkin method 
(work for which my advisor, Leszek Demkowicz recently received the IACM Computational Mechanics Award)
and I have been privileged to witness DPG grow from early 1D convection-diffusion results to 
full Navier-Stokes solutions and a host of other challenging problems.
My work has been devoted to perfecting DPG for fluid dynamics applications, starting with a locally conservative variant and 
converging on space-time formulations for transient flows.
Beginning with the heat equation, I developed a theory and a framework for how space-time DPG should work and then quickly extended this to compressible
Navier-Stokes and obtained some very promising results on shock tube problems.
The rest of my dissertation (still in progress) will result in 2D incompressible and compressible simulations, but
even so, I feel as though I will have only scratched the surface when it comes to space-time DPG.
Our preliminary results have generated a lot of excitement at national and international conferences,
and we are convinced that this technique has a lot of potential for exploiting the features of future high performance computing systems.
I am coauthor on two papers which came out of my internships at Lawrence Livermore developing modern Lagrangian compressible flow solvers.
In addition to a couple of technical reports, my first paper as lead author was just published on the topic of locally conservative DPG 
and a second on space-time DPG was just submitted to Computers \& Fluids.
A third mathematically oriented paper on the robustness of space-time DPG is current in preparation,
as well as a comparison of simulations utilizing primitive, conservation, and entropy variables.
% and a fourth paper on our 2D results will probably be submitted before next spring.

Unfortunately, due to the developmental nature of my dissertation work, I won't have the opportunity to do more than a few standard test problems.
Being a very application-oriented person, I would relish the opportunity to work on some more challenging simulations.
So primarily I am excited about the von Neumann Fellowship as a means to advance research which I believe to have a bright future for computational mechanics.
I have past experience working at Lawrence Livermore (four summers) and admire the research culture created at the DOE labs. 
The lab environment aligns closely with my ideal combination of private industry and academia 
and I am excited by the possibility of conducting this research at Sandia.
I appreciate your time in consideration of my application, and I look forward to hearing from you.

\medskip
Sincerely,

\medskip
Truman Everett Ellis



\end{document}
