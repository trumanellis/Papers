\documentclass{article}
\usepackage{amsmath, amssymb, amsthm, mathtools}
\usepackage{graphicx}
\usepackage{cancel}
\usepackage{color}
\usepackage{caption}
\usepackage{subcaption}
%\usepackage{subfigure}
\usepackage{pgf,tikz}
\usetikzlibrary{arrows}

\captionsetup{compatibility=false}

\newcommand{\bs}[1]{\boldsymbol{#1}}
\newcommand{\norm}[1]{\left\| #1 \right\|}
\newcommand{\snorm}[1]{\left| #1 \right|}
\newcommand{\LRp}[1]{\left( #1 \right)}
\newcommand{\LRs}[1]{\left[ #1 \right]}
\newcommand{\LRc}[1]{\left\{ #1 \right\}}
\newcommand{\LRa}[1]{\left\langle #1 \right\rangle}
\newcommand{\LRb}[1]{\left| #1 \right|}
\newcommand{\Grad}{\ensuremath{\nabla}}
\newcommand{\Gradxt}{\ensuremath{\nabla_{xt}}}
\newcommand{\Div}{\ensuremath{\nabla\cdot}}
\newcommand{\Divxt}{\ensuremath{\nabla_{xt}\cdot}}
\newcommand{\Curl}{\ensuremath{\nabla\times}}
\newcommand{\bfH}{\mbox{\boldmath $H$}}
\newcommand{\bfsigma}{\boldsymbol\sigma}
\newcommand{\bfvarsigma}{\boldsymbol\varsigma}
\newcommand{\bftau}{\boldsymbol\tau}
\newcommand{\bfbeta}{\boldsymbol\beta}
\newcommand{\bflambda}{\boldsymbol\lambda}
\newcommand{\bfpsi}{\boldsymbol\psi}
\newcommand{\bfu}{\boldsymbol u}
\newcommand{\bfv}{\boldsymbol v}
\newcommand{\bfV}{\boldsymbol V}
\newcommand{\bfZ}{\boldsymbol Z}
\newcommand{\bfz}{\boldsymbol z}
\newcommand{\bfW}{\boldsymbol W}
\newcommand{\bfw}{\boldsymbol w}
\newcommand{\bfm}{\boldsymbol m}
\newcommand{\bfM}{\boldsymbol M}
\newcommand{\bbM}{\mathbb{M}}
\newcommand{\bfq}{\boldsymbol q}
\newcommand{\bfU}{\boldsymbol U}
\newcommand{\bfS}{\boldsymbol S}
\newcommand{\bbS}{\mathbb{S}}
\newcommand{\bbD}{\mathbb{D}}
\newcommand{\bfK}{\boldsymbol K}
\newcommand{\bbK}{\mathbb{K}}
\newcommand{\bfn}{\boldsymbol n}
\newcommand{\bff}{\boldsymbol f}
\newcommand{\bfF}{\boldsymbol F}
\newcommand{\bbF}{\mathbb{F}}
\newcommand{\bfg}{\boldsymbol g}
\newcommand{\bfG}{\boldsymbol G}
\newcommand{\bfC}{\boldsymbol C}
\newcommand{\bft}{\boldsymbol t}
\newcommand{\bfT}{\boldsymbol T}
\newcommand{\bfI}{\boldsymbol I}
\newcommand{\bbI}{\mathbb{I}}
\newcommand{\bfx}{\boldsymbol x}
\newcommand{\uh}{\widehat{u}}
\newcommand{\fnh}{\widehat{f}_n}
\newcommand{\LQ}{L^2\LRp{Q}}
\newcommand{\LK}{L^2\LRp{K}}
\newcommand{\LVecK}{\mathbf{L}^2\LRp{K}}
\newcommand{\LVecQ}{\mathbf{L}^2\LRp{Q}}
\newcommand{\HdivK}{\bfH(\text{div},K)}
\newcommand{\HdivOmega}{\bfH(\text{div},\Omega)}
% \newcommand{\HdivOmegaLT}{\bfH(\text{div},\Omega)\times L^2([0,T])}
\newcommand{\HdivQ}{\bfH(\text{div}_{xt},Q)}
\newcommand{\HOneK}{H^{1}(K)}
\newcommand{\HOneVecK}{\bfH^{1}(K)}
\newcommand{\HOneQ}{H^{1}(Q)}
\newcommand{\HOneOmegah}{H^{-1}(\Omega_h)}
\newcommand{\HdivOmegah}{\bfH(\text{div},\Omega_h)}
\newcommand{\vdeltau}{v_{\delta\bs u_h}}
\newcommand{\taudeltau}{\bftau_{\delta\bs u_h}}
\newcommand{\ip}[1]{\left\langle #1 \right\rangle}
\newcommand{\pd}[2]{\frac{\partial#1}{\partial#2}}
\newcommand{\pt}[1]{\frac{\partial#1}{\partial t}}
\newcommand{\ppd}[2]{\frac{\partial^2#1}{\partial#2^2}}
\newcommand{\pdd}[3]{\frac{\partial^2#1}{\partial#2\partial#3}}
\newcommand{\der}[2]{\frac{\mathrm{d}#1}{\mathrm{d}#2}}
\newcommand{\Oh}{\Omega_h}
\newcommand{\jump}[1] {\ensuremath{\LRs{\![#1]\!}}}
\newcommand{\Gh}{\Gamma_h}
\newcommand{\mcU}{\mathcal{U}}
\newcommand{\mcUh}{\hat{\mathcal{U}}}
\newcommand{\LOmega}{L^2\LRp{\Omega_h}}

\newcommand{\eqnref}[1]{\eqref{eq:#1}}

\DeclareMathOperator*{\argmin}{arg\,min}
\DeclareMathOperator*{\trace}{tr}

\def\arrtwo#1#2#3#4{\left[
\begin{array}{cc}
#1\; & #2\\
#3\; & #4\\
\end{array}
\right]}
\def\arrthree#1#2#3#4#5#6#7#8#9{\left[
\begin{array}{ccc}
#1\; & #2\; & #3\\
#4\; & #5\; & #6\\
#7\; & #8\; & #9\\
\end{array}
\right]}
\def\arrthreeone#1#2#3{\left[
\begin{array}{ccc}
#1\; & #2\; & #3\\
\end{array}
\right]}
\def\vecttwo#1#2{\left(
\begin{array}{c}
#1\\
#2\\
\end{array}
\right)}
\def\svecttwo#1#2{\left[
\begin{array}{c}
#1\\
#2\\
\end{array}
\right]}
\def\vectthree#1#2#3{\left(
\begin{array}{c}
#1\\
#2\\
#3\\
\end{array}
\right)}
\def\svectthree#1#2#3{\left[
\begin{array}{c}
#1\\
#2\\
#3\\
\end{array}
\right]}

\renewcommand{\arraystretch}{1.2}

\def\etal{{\it et al.~}}


\title{Robustness for Transient Problems}
\author{Truman E. Ellis and Jesse L. Chan}

\begin{document}
\maketitle

Consider domain $Q=\Omega\times[0,T]$ with boundary $\Gamma=\Gamma_-\cup\Gamma_+\cup\Gamma_0\cup\Gamma_T$ 
where $\Gamma_-$ is the spatial inflow boundary, $\Gamma_+$ is the spatial outflow boundary, $\Gamma_0$ is the initial time boundary, 
and $\Gamma_T$ is the final time boundary. Let $\Gamma_h$ denote the entire mesh skeleton.

Assume that boundary conditions are applied on the boundary $\Gamma_0\subset \Gamma$.  Recall that, for the ultra-weak variational formulation
\[
b\LRp{\LRp{u,\uh},v} = \LRp{u,A^*_h v}_{\L} + \LRa{\uh, \jump{v}}_{\Gh\setminus \Gamma_0}
\]
we can recover
\[
\norm{u}_{\LQ}^2 = b(u,v^*)
\]
for conforming $v^*$ satisfying the adjoint equation
\begin{align*}
A^* v^* &= u\\
v^* &= 0 \text{ on } \Gh\setminus\Gamma_0.
\end{align*}
Together, these give necessary conditions on the test norm $\norm{\cdot}_V$ such that we have $L^2$ robustness (this gives robustness in the variable $u$; for the first order formulation, conditions for $\sigma$ must also be shown).  
\[
\norm{u}_{\LQ}^2 = b(u,v^*) \leq \frac{b(u,v^*)}{\norm{v^*}_V} \norm{v^*}_V \leq \norm{u}_E \norm{v^*}_V
\]
Thus, showing $\norm{v^*}_V \lesssim \norm{u}_{\LQ}$ gives the result that $\norm{u}_{\LQ} \lesssim \norm{u}_E$.  


% \section{Reaction-Diffusion}

% Consider reaction diffusion
% \begin{align*}
% \pd{u}{t} + u - \epsilon \Delta u &= f\\
% u &= 0 \text{ on } \Gamma_1\\
% \pd{u}{n} &= 0 \text{ on } \Gamma_2\\
% u &= u_0 \text{ on } \Gamma_0.
% \end{align*}
% The adjoint equation satisfies
% \begin{align*}
% -\pd{v}{t} + v - \epsilon \Delta v &= u\\
% v &= 0 \text{ on } \Gamma_1\\
% \pd{v}{n} &= 0 \text{ on } \Gamma_2\\
% v &= 0 \text{ on } \Gamma_T.
% \end{align*}
% (The boundary conditions can be derived by taking the ultra-weak formulation and choosing boundary conditions such that the temporal flux and spatial flux terms $\LRa{\uh, \jump{\tau_n}}_{\Gamma_1}$ and $\LRa{\fnh,\jump{v}}_{\Gamma_2}$ are zero.)

% We can then derive that the test norm
% \[
% \norm{v}_V^2 = \norm{\pd{v}{t}}^2 + \norm{v}^2 + \epsilon\norm{\Grad v}^2 
% \]
% provides the necessary bound $\norm{v^*}_V \lesssim \norm{u}_{\L}$.

% To see, this we multiply the adjoint equation by two terms as follows:
% \begin{enumerate}
% \item Multiply by $v$ and integrate over $Q$ to get
% \[
% -\int_Q \pd{v}{t}v + \int_Q v^2 + \epsilon \int_Q \LRb{\Grad v}^2 - \epsilon \int_0^T\int_{\Gamma} \pd{v}{n}v = \int_Q uv.
% \]
% Noting that either $v = 0$ or $\pd{v}{n} = 0$ on the boundary removes the integral over $\Gamma$.  Next, we can factor the first term and use Young's inequality to get
% \[
% -\int_0^T  \pd{}{t}\int_{\Omega} v^2 + \norm{v}^2_Q + \epsilon \norm{\Grad v}^2_Q \leq \frac{1}{2}\norm{u}^2_Q + \frac{1}{2}\norm{v}^2_Q
% \]
% Integrating by parts the first term gives
% \[
% -\left.\int_{\Omega} v^2\right|_0^T + \frac{1}{2}\norm{v}^2_Q + \epsilon \norm{\Grad v}^2_Q \leq \frac{1}{2}\norm{u}^2_Q
% \]
% Using boundary condition $v=0$ at $t= T$ gives
% \[
% \frac{1}{2}\norm{v}^2_Q + \epsilon \norm{\Grad v}^2_Q \leq \int_{\Omega} v(t=0)^2 + \frac{1}{2}\norm{v}^2_Q + \epsilon \norm{\Grad v}^2_Q \leq \frac{1}{2}\norm{u}^2_Q.
% \]

% \item Multiply by $-\pd{v}{t}$ and integrate over $Q$.  Young's inequality changes the right hand side to 
% \[
% \int_Q\pd{v}{t}^2 - \int_Q v\pd{v}{t} + \epsilon\int_Q \Delta v \pd{v}{t} = \int_Q -u \pd{v}{t} \leq \frac{1}{2}\norm{u}_Q^2 + \frac{1}{2}\norm{\pd{v}{t}}_Q^2 .
% \]
% The term $\int_Q v\pd{v}{t}$ can be reduced to the positive contribution $\int_{\Omega}{v(t=0)}^2  $ as above.  We can then take the Laplacian term, integrate by parts in space to get
% \[
% \int_Q \Delta v \pd{v}{t} = \int_0^T \int_{\Omega} \Delta v \pd{v}{t} =  \int_0^T \int_{\Gamma} \pd{v}{t}\pd{v}{n} - \int_0^T \int_{\Omega}\Grad\LRp{\pd{v}{t}}\Grad v.
% \]
% Since either $v = 0$ or $\pd{v}{n} = 0$ on $\Gamma$, the first term disappears.  The second term can be bounded by noting
% \[
% - \int_0^T \int_{\Omega}\Grad\LRp{\pd{v}{t}}\Grad v = - \int_0^T \pd{}{t}\int_{\Omega}\LRb{\Grad v} ^2 = - \left.\int_{\Omega}\LRb{\Grad v}^2 \right|_0^T.
% \]
% Since $v = 0$ at $t=T$, $\Grad v = 0$ at $t=T$ as well, and we are left with the positive contribution $\int_{\Omega}\LRb{\Grad v(t=0)}^2$.  Then,
% \[
% \frac{1}{2}\norm{\pd{v}{t}}_Q^2 \leq \frac{1}{2}\norm{u}_Q.
% \]
% \end{enumerate}
% Together, these two show that, under test norm
% \[
% \norm{v}_V^2 = \norm{\pd{v}{t}}^2 + \norm{v}^2 + \epsilon\norm{\Grad v}^2,
% \]
% the adjoint equation $v^*$ satisfies
% \[
% \norm{v^*}_V \lesssim \norm{u}_{\L}
% \]
% and thus the DPG energy norm robustly bounds the $L^2$ norm from above
% \[
% \norm{u}_{\L} \lesssim \norm{u}_E.
% \]

\section{Convection-Diffusion}

Consider convection-diffusion
\begin{align*}
\frac{1}{\epsilon}\bfsigma-\Grad u &=0\\
\pd{u}{t} + \bfbeta\cdot\Grad u - \Div\bfsigma &= f\\
\pd{u}{n} &= 0 \text{ on } \Gamma_-\\
u &= 0 \text{ on } \Gamma_+\\
u &= u_0 \text{ on } \Gamma_0.
\end{align*}
Let $\tilde\bfbeta:=\vecttwo{\bfbeta}{1}$ and $\Gradxt:=\vecttwo{\Grad}{\pd{}{t}}$, then we can rewrite this as
\begin{align*}
\frac{1}{\epsilon}\bfsigma-\Grad u &=0\\
\tilde\bfbeta\cdot\Gradxt u - \Div\bfsigma &= f\\
% \Divxt\vecttwo{\bfbeta u-\bfsigma}{u}&=f\\
\pd{u}{n} &= 0 \text{ on } \Gamma_-\\
u &= 0 \text{ on } \Gamma_+\\
u &= u_0 \text{ on } \Gamma_0.
\end{align*}
We decompose the adjoint into three parts:
a discontinuous part
\begin{align*}
\frac{1}{\epsilon}\bftau_0+\Grad v_0 &=0\\
-\tilde\bfbeta\cdot\Gradxt v_0 + \Div\bftau_0 &= 0\\
\bftau_0\cdot\bfn_x &= \bftau\cdot\bfn_x \text{ on } \Gamma_-\cup\Gamma_0\\
v_0 &= v \text{ on } \Gamma_+\\
v_0 &= v \text{ on } \Gamma_T\\
\jump{v_0}&=\jump{v} \text{ on } \Gamma_h^0\\
\jump{\bftau_0\cdot\bfn_x}&=\jump{\bftau_0\cdot\bfn_x} \text{ on } \Gamma_{hx}^0\,,
\end{align*}
a continuous part with forcing term $g$
\begin{align*}
\frac{1}{\epsilon}\bftau_1+\Grad v_1 &=0\\
-\tilde\bfbeta\cdot\Gradxt v_1 + \Div\bftau_1 &= g\\
\bftau_1\cdot\bfn_x &= 0 \text{ on } \Gamma_-\\
v_1 &= 0 \text{ on } \Gamma_+\\
v_1 &= 0 \text{ on } \Gamma_T\,,
\end{align*}
and a continuous part with forcing $\bff$
\begin{align*}
\frac{1}{\epsilon}\bftau_2+\Grad v_2 &=\bff\\
-\tilde\bfbeta\cdot\Gradxt v_2 + \Div\bftau_2 &= 0\\
\bftau_2\cdot\bfn_x &= 0 \text{ on } \Gamma_-\\
v_2 &= 0 \text{ on } \Gamma_+\\
v_2 &= 0 \text{ on } \Gamma_T\,.
\end{align*}
(The boundary conditions can be derived by taking the ultra-weak formulation and choosing boundary conditions such that the temporal flux and spatial flux terms $\LRa{\uh, \jump{\tau_n}}_{\Gamma_{out}}$ and $\LRa{\hat t_n,\jump{v}}_{\Gamma_{in}}$ are zero.)

We can then derive that the test norm
\begin{align}
\label{eq:robustnorm}
\norm{\LRp{v,\bftau}}_{V,K}^2 &\coloneqq
\frac{1}{\epsilon}\norm{\bftau}_K^2
+ \norm{\Div \bftau - \tilde\bfbeta\cdot\Gradxt v}_K^2 \nonumber\\
&+\norm{\bfbeta\cdot \Grad v}_K^2
+ \epsilon\norm{\Grad v}_K^2
+ \norm{v}^2_K\,,
\end{align}
provides the necessary bound $\norm{v^*}_V \lesssim \norm{u}_{\LQ}$.

In the following lemmas we establish the following bounds:
\begin{itemize}
\item Bound on $\norm{\LRp{v_0,\bftau_0}}_V$.
\item Bound on $\norm{\LRp{v_1,\bftau_1}}_V$. 
Lemma~\ref{lem:convective} gives $\norm{\tilde\bfbeta\cdot\Gradxt v_1}\leq\norm{g}$.
Since $\Div\bftau_1=g+\tilde\bfbeta\cdot\Gradxt v_1$, 
\[
\norm{\Div\bftau_1}\leq\norm{g}+\norm{\tilde\bfbeta\cdot\Gradxt v_1}\leq 2\norm{g}.
\]
Or, the fact that $\Div\bftau-\tilde\bfbeta\cdot\Gradxt v_1=g$ clearly gives
\[
\norm{\Div \bftau - \tilde\bfbeta\cdot\Gradxt v_1}=\norm{g}.
\]
Also, clearly
\[
\norm{\bfbeta\cdot\Grad v_1}\leq\norm{\tilde\bfbeta\cdot\Gradxt v_1}\leq\norm{g}.
\]
Lemma~\ref{lem:l2} gives $\norm{v_1}^2+\epsilon\norm{\Grad v_1}^2\leq\norm{g}^2$.
Since $\epsilon^{1/2}\Grad v_1=-\epsilon^{-1/2}\bftau_1$,
\[
\frac{1}{\epsilon}\norm{\bftau_1}^2\leq\norm{g}^2.
\]
Thus, all $(v_1,\bftau_1)$ terms in \eqref{eq:robustnorm} are accounted for.
\item Bound on $\norm{\LRp{v_2,\bftau_2}}_V$. 
The fact that $\Div\bftau-\tilde\bfbeta\cdot\Gradxt v=0$ clearly gives
\[
\norm{\Div \bftau - \tilde\bfbeta\cdot\Gradxt v_2}=0\leq\norm{\bff}.
\]
Lemma~\ref{lem:l2} gives $\norm{v_2}^2+\epsilon\norm{\Grad v_2}^2\leq\epsilon\norm{\bff}^2$.
Since $\epsilon^{1/2}\Grad v_2=\bff-\epsilon^{-1/2}\bftau_2$,
\[
\frac{1}{\epsilon}\norm{\bftau_2}^2\leq(1+\epsilon)\norm{\bff}^2.
\]
Finally,
\[
\norm{\bfbeta\cdot\Grad v_2}\leq\norm{\bfbeta}_\infty\norm{\Grad v_2}\leq\norm{\bfbeta}_\infty\norm{\bff}.
\]
Thus, all $(v_2,\bftau_2)$ terms in \eqref{eq:robustnorm} are accounted for.
\end{itemize}

Our goal is to analyze the stability properties of the adjoint equations by deriving bounds of the form
$\norm{(v_1,\bftau_1)}_V\leq\norm{g}_\LQ$ and $\norm{(v_2,\bftau_2)}_V\leq\norm{f}_\LQ$.

\textcolor{red}{Insert conditions on $\bfbeta$}
\begin{lemma}
\label{lem:convective}
For the above conditions on $\bfbeta$,
\[
\norm{\tilde\bfbeta\cdot\Gradxt v_1}\leq \norm{g}.
\]
\end{lemma}
\begin{proof}
Multiply by $-\tilde\bfbeta\cdot\Gradxt v$ and integrate over $Q$ to get
\begin{equation}
\label{eq:adj1}
\norm{\tilde\bfbeta\cdot\Gradxt v}=-\int_Q g\tilde\bfbeta\cdot\Gradxt v+\int_Q\tilde\bfbeta\cdot\Gradxt v\Div\bftau\,.
\end{equation}
Note that
\begin{align*}
\frac{1}{\epsilon}\int_Q\tilde\bfbeta\cdot\Gradxt v\Div\bftau&=-\int_Q\tilde\bfbeta\cdot\Gradxt v\Div\Grad v \\
%
&=-\int_{\Gamma_x}\tilde\bfbeta\cdot\Gradxt v\Grad v\cdot\bfn_x
+\int_Q\Grad(\tilde\bfbeta\cdot\Gradxt v)\cdot\Grad v \\
%
&=-\int_{\Gamma_x}\tilde\bfbeta\cdot\Gradxt v\Grad v\cdot\bfn_x
+\int_Q(\Grad\tilde\bfbeta\cdot\Gradxt v)\cdot\Grad v\\
&\quad+\int_Q\tilde\bfbeta\cdot\Grad\Gradxt v\cdot\Grad v\\
%
&=-\int_{\Gamma_x}\tilde\bfbeta\cdot\Gradxt v\Grad v\cdot\bfn_x
+\int_Q(\Grad\bfbeta\cdot\Grad v)\cdot\Grad v\\
&\quad+\frac{1}{2}\int_Q\tilde\bfbeta\cdot\Gradxt(\Grad v\cdot\Grad v)\\
%
&=-\int_{\Gamma_x}\tilde\bfbeta\cdot\Gradxt v\Grad v\cdot\bfn_x
+\int_Q(\Grad\bfbeta\cdot\Grad v)\cdot\Grad v\\
&\quad+\frac{1}{2}\int_\Gamma\tilde\bfbeta\cdot\bfn(\Grad v\cdot\Grad v)
-\frac{1}{2}\int_Q\Divxt\tilde\bfbeta(\Grad v\cdot\Grad v)\\
%
&=-\int_{\Gamma_x}\tilde\bfbeta\cdot\Gradxt v\Grad v\cdot\bfn_x
+\int_Q(\Grad\bfbeta\cdot\Grad v)\cdot\Grad v\\
&\quad+\frac{1}{2}\int_\Gamma\tilde\bfbeta\cdot\bfn(\Grad v\cdot\Grad v)
-\frac{1}{2}\int_Q\Div\bfbeta(\Grad v\cdot\Grad v)\\
%
&=-\int_{\Gamma_x}\tilde\bfbeta\cdot\Gradxt v\Grad v\cdot\bfn_x
+\frac{1}{2}\int_\Gamma\tilde\bfbeta\cdot\bfn(\Grad v\cdot\Grad v)\\
&\quad+\int_Q\Grad v(\Grad\bfbeta-\frac{1}{2}\Div\bfbeta\bfI)\Grad v\\
\end{align*}
Plugging this into \eqref{eq:adj1}, we get
\begin{align*}
\norm{\tilde\bfbeta\cdot\Gradxt v}
&=-\int_Q g\tilde\bfbeta\cdot\Gradxt v
+\epsilon\int_Q\Grad v(\Grad\bfbeta-\frac{1}{2}\Div\bfbeta\bfI)\Grad v\\
&\quad-\epsilon\int_{\Gamma_x}\tilde\bfbeta\cdot\Gradxt v\Grad v\cdot\bfn_x
+\epsilon\frac{1}{2}\int_\Gamma\tilde\bfbeta\cdot\bfn(\Grad v\cdot\Grad v)\\
%
&=-\int_Q g\tilde\bfbeta\cdot\Gradxt v
+\epsilon\int_Q\Grad v(\Grad\bfbeta-\frac{1}{2}\Div\bfbeta\bfI)\Grad v\\
&\quad
-\int_{\Gamma_-}\tilde\bfbeta\cdot\Gradxt v\underbrace{\Grad v\cdot\bfn_x}_{=0}
-\int_{\Gamma_+}\LRp{\underbrace{\pd{v}{t}}_{=0}+\bfbeta\cdot\Grad v}\Grad v\cdot\bfn_x\\
&\quad
+\frac{1}{2}\int_{\Gamma_-}\underbrace{\bfbeta\cdot\bfn_x}_{<0}(\Grad v\cdot\Grad v)
+\frac{1}{2}\int_{\Gamma_+}\bfbeta\cdot\bfn_x(\Grad v\cdot\Grad v)\\
&\quad
+\frac{1}{2}\int_{\Gamma_0}\underbrace{n_t}_{<0}(\Grad v\cdot\Grad v)
+\frac{1}{2}\int_{\Gamma_T}n_t\underbrace{(\Grad v\cdot\Grad v)}_{=0}\\
%
&\leq-\int_Q g\tilde\bfbeta\cdot\Gradxt v
+\epsilon\int_Q\Grad v(\Grad\bfbeta-\frac{1}{2}\Div\bfbeta\bfI)\Grad v\\
&\quad
+\int_{\Gamma_+}\LRp{-\pd{v}{\bfn_x}\bfbeta
+\frac{1}{2}\bfbeta\cdot\bfn_x\Grad v}\cdot\Grad v\\
%
&=-\int_Q g\tilde\bfbeta\cdot\Gradxt v
+\epsilon\int_Q\Grad v(\Grad\bfbeta-\frac{1}{2}\Div\bfbeta\bfI)\Grad v\\
&\quad
+\int_{\Gamma_+}\LRp{-\pd{v}{\bfn_x}\bfbeta
+\frac{1}{2}\bfbeta\cdot\bfn_x\pd{v}{\bfn_x}\bfn_x}\cdot\pd{v}{\bfn_x}\bfn_x\\
%
&=-\int_Q g\tilde\bfbeta\cdot\Gradxt v
+\epsilon\int_Q\Grad v(\Grad\bfbeta-\frac{1}{2}\Div\bfbeta\bfI)\Grad v\\
&\quad
\underbrace{-\frac{1}{2}\int_{\Gamma_+}\LRp{\pd{v}{\bfn_x}}^2\bfbeta\cdot\bfn_x}_{<0}\\
%
&\leq-\int_Q g\tilde\bfbeta\cdot\Gradxt v
+\epsilon\int_Q\Grad v(\Grad\bfbeta-\frac{1}{2}\Div\bfbeta\bfI)\Grad v\\
%
&\leq-\frac{\norm{g}}{2}+\frac{\norm{\tilde\bfbeta\cdot\Gradxt v}}{2}
+\epsilon\int_Q\Grad v(\Grad\bfbeta-\frac{1}{2}\Div\bfbeta\bfI)\Grad v\\
%
&\leq-\frac{\norm{g}}{2}+\frac{\norm{\tilde\bfbeta\cdot\Gradxt v}}{2}
+\epsilon C\norm{\Grad v}^2
\end{align*}
\end{proof}

\begin{lemma}
\label{lem:l2}
For the duration of this lemma, let $v:=v_1+v_2$.
Then, for the above conditions on $\bfbeta$,
\[
\norm{v}^2+\epsilon\norm{\Grad v}^2\leq\norm{g}^2+\epsilon\norm{\bff}^2\,.
\]
\end{lemma}
\begin{proof}
Define $w=e^{t}v$ and note that $\pd{w}{t}=\LRp{\pd{v}{t}+v}e^{t}$ while
all spatial derivatives go through.
%  $\Grad w=\cancelto{0}{\Grad e^{t}} v+e^{t}\Grad v$ and
% $\Div(\bfbeta w)=\Div(\bfbeta)e^{t} v+\bfbeta\cdot e^{t}\Grad v$ and $\Delta w=e^{t}\Delta v$. 
% Also, $\Gradxt w=\pd{e^{t} v}{t}+\Grad{e^{t} v}=e^{t}(\Gradxt v-v)$.
% Plugging this into the adjoint equation, we get
Multiplying the adjoint by $w$ and integrating over $Q$ gives
% \begin{equation*}
% -\tilde\bfbeta\cdot\Gradxt(w)-\epsilon\Delta w=g-\epsilon\Div\bff
% \end{equation*}
% or 
% \begin{equation*}
% \tilde\bfbeta\cdot\Gradxt(v)-v+\epsilon\Delta v=e^{t}(-g+\epsilon\Div\bff)
% \end{equation*}
% Multiply by $-v$ and integrate to get
\begin{align*}
-\int_Q\tilde\bfbeta\cdot\Gradxt vw-\epsilon\Delta vw=\int_Qgw-\epsilon\int_Q\Div\bff w
\end{align*}
or
\begin{align*}
-\int_Qe^{t}v\tilde\bfbeta\cdot\Gradxt v-\epsilon\int_Q e^{t}v\Delta v=\int_Qe^{t}gv-\epsilon\int_Qe^{t}v\Div\bff
\end{align*}
Integrating by parts:
\begin{align*}
\int_Q\Divxt\LRp{e^{t}\tilde\bfbeta v}v&
-\int_\Gamma e^{t}\tilde\bfbeta\cdot\bfn v^2
+\epsilon\int_Qe^{t}\Grad v\cdot\Grad v
-\epsilon\int_{\Gamma_x}e^{t}v\cdot\Grad v\cdot\bfn_x
\\
&=
\int_Qe^{t}gv
+\epsilon\int_Qe^{t}\Grad v\cdot\bff
-\epsilon\int_{\Gamma_x}e^{t}v\bff\cdot\bfn_x
\end{align*}
Note that $\Divxt{e^{t}v\tilde\bfbeta}=e^{t}(\tilde\bfbeta\Gradxt v+v)$ if $\Div\bfbeta=0$.
% Dividing both sides by $e^{t}$ and moving some terms to the right hand side, we get
Moving some terms to the right hand side, we get
\begin{align*}
\int_Q e^{t}v^2
&+\int_Q\epsilon e^{t}\Grad v\cdot\Grad v
\\
&=
\int_Qe^{t}gv
+\epsilon\int_Qe^{t}\Grad v\cdot\bff
-\epsilon\int_{\Gamma_x}e^{t}v\bff\cdot\bfn_x\\
&\quad-\int_Qe^{t}\tilde\bfbeta\cdot\Gradxt vv
\quad+\int_\Gamma e^{t}\tilde\bfbeta\cdot\bfn v^2
\quad+\epsilon\int_{\Gamma_x}e^{t}v\cdot\Grad v\cdot\bfn_x
\end{align*}
Note that $1\leq\norm{e^t}_\infty=e^T$, 
then
\begin{align*}
\norm{v}^2
&+\epsilon\norm{\Grad v}^2
\\
&\leq
e^T\left(
\int_Qgv
+\epsilon\int_Q\Grad v\cdot\bff
-\epsilon\int_{\Gamma_-}v\underbrace{\bff\cdot\bfn_x}_{=\cancelto{0}{\tau_n}+\pd{v}{\bfn_x}}
-\epsilon\int_{\Gamma_+}\underbrace{v}_{=0}\bff\cdot\bfn_x
\right.\\
&\left.
\quad-\int_Q\tilde\bfbeta\cdot\Gradxt vv
\quad+\int_\Gamma \tilde\bfbeta\cdot\bfn v^2
\quad+\epsilon\int_{\Gamma_-}v\cdot\Grad v\cdot\bfn_x
\quad+\epsilon\int_{\Gamma_+}\underbrace{v}_{=0}\pd{v}{\bfn_x}
\right)
\\
&=
e^T\left(
\int_Qgv
+\epsilon\int_Q\Grad v\cdot\bff
\cancel{
-\epsilon\int_{\Gamma_-}v\pd{v}{\bfn_x}
\quad+\epsilon\int_{\Gamma_x}v\pd{v}{\bfn_x}
}
\right.\\
&\left.
\quad-\frac{1}{2}\int_Q\tilde\bfbeta\cdot\Gradxt v^2
\quad+\int_\Gamma \tilde\bfbeta\cdot\bfn v^2
\right)
\\
&=
e^T\left(
\int_Qgv
+\epsilon\int_Q\Grad v\cdot\bff
\right)\\
&\left.
\quad+\frac{1}{2}\int_Q\cancelto{0}{\Divxt\tilde\bfbeta} v^2
\quad-\frac{1}{2}\int_\Gamma\tilde\bfbeta\cdot\bfn v^2
\quad+\int_\Gamma \tilde\bfbeta\cdot\bfn v^2
\right)
\\
&=
e^T\left(
\int_Qgv
+\epsilon\int_Q\Grad v\cdot\bff
\right.\\
&\left.
\quad+\frac{1}{2}\LRp{\int_{\Gamma_0} \underbrace{-v^2}_{\leq 0}
\quad+\int_{\Gamma_T} \cancelto{0}{v^2}
\quad+\int_{\Gamma_-} \underbrace{\bfbeta\cdot\bfn_x v^2}_{\leq 0}
\quad+\int_{\Gamma_+} \bfbeta\cdot\bfn_x \cancelto{0}{v^2}
}
\right)
\\
&\leq
e^T\left(
\int_Qgv
+\epsilon\int_Q\Grad v\cdot\bff
\right)
\\
&\leq
e^T\left(
\frac{\norm{g}^2}{2}
+\epsilon\frac{\norm{\bff}^2}{2}
+\frac{\norm{v}^2}{2}
+\epsilon\frac{\norm{\Grad v}^2}{2}
\right)
\end{align*}
\end{proof}

\section{Entropy Norm and Navier-Stokes}
The adjoint for Navier-Stokes looks like
\begin{align*}
% \frac{1}{\mu}\bftau+\Grad v &=0\\
-\pd{v}{t}-A\cdot\Grad v - (K_{ij}v_{,j})_{,i} &= A_0 U\\
\bftau\cdot\bfn_x &= 0 \text{ on } \Gamma_-\\
v &= 0 \text{ on } \Gamma_+\\
v &= 0 \text{ on } \Gamma_T\,,
\end{align*}
We want to create a norm that is bounded by $(U,A_0U)=\norm{A_0^\frac{1}{2}U}^2$.
\begin{lemma}
\label{lem:convectiveNS}
For the above conditions on $\bfbeta$,
\[
\norm{\tilde\bfbeta\cdot\Gradxt v_1}\leq \norm{g}.
\]
\end{lemma}
\begin{proof}
\end{proof}

\section{Robustness for transient problems given spatial robustness}

Suppose we have the transient problem
\[
\pd{u}{t} + Au = f
\]
with initial condition $u(x,0) = u_0$.  Suppose that DPG is robust under the ultra-weak variational formulation for the steady problem
\[
\LRp{u,A^*_h v}_{\L} + \LRa{\uh, \jump{v}}_{\Gamma_h\setminus \Gamma_0} = \LRp{f,v}
\]
with test norm $\norm{v}_{V}$.  Then, can we show that 
\[
\norm{v}_{V,t} \coloneqq \norm{v}_V + \norm{\pd{v}{t}}_{\L}
\]
also leads to a robust upper bound of the $L^2$ norm by the DPG energy norm?  I believe this may be possible.  The adjoint equation for robustness for the transient problem gives
\[
-\pd{v}{t} + A^*v = u
\]
with $v = 0$ at $t=T$...  

\section{Transient Eriksson-Johnson}

We can derive a transient Eriksson-Johnson solution using separation of variables.  Consider
\[
\pd{u}{t} + \pd{u}{x} - \epsilon \Delta u = 0
\]
with boundary conditions
\begin{align*}
u &= 0 \text{ on } \Gamma_+,\\
u - \epsilon\pd{u}{n} &= u_0 - \epsilon\pd{u_0}{n} \text{ on }\Gamma_-,\\
\epsilon\pd{u}{n} &= 0 \text{ on } \Gamma_0,
\end{align*}
and initial condition $u(x,y,0) = u_0(x,y)$ that satisfies the given boundary data.  Assuming that $u(x,y,t) = X(x,y)T(t)$ and $Lu = \pd{u}{x} - \epsilon \Delta u$, we can plug this into the equation 
\[
\pd{u}{t} + Lu = 0
\]
and rearrange to get 
\[
-\frac{\pd{T}{t}}{T} = \frac{LX}{X} = C.
\]
This assumes then that $\pd{T}{t} = -CT$, or that $T(t) = e^{-Ct}$, and that $LX = CX$, or that $X$ is made up of the eigenfunctions of $L$.  

\end{document}