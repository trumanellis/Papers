\documentclass{article}
\usepackage{amsmath, amssymb, amsthm, mathtools}
\usepackage{graphicx}
\usepackage{cancel}
\usepackage{color}
\usepackage{caption}
\usepackage{subcaption}
%\usepackage{subfigure}
\usepackage{pgf,tikz}
\usetikzlibrary{arrows}

\captionsetup{compatibility=false}

\newcommand{\bs}[1]{\boldsymbol{#1}}
\newcommand{\norm}[1]{\left\| #1 \right\|}
\newcommand{\snorm}[1]{\left| #1 \right|}
\newcommand{\LRp}[1]{\left( #1 \right)}
\newcommand{\LRs}[1]{\left[ #1 \right]}
\newcommand{\LRc}[1]{\left\{ #1 \right\}}
\newcommand{\LRa}[1]{\left\langle #1 \right\rangle}
\newcommand{\LRb}[1]{\left| #1 \right|}
\newcommand{\Grad}{\ensuremath{\nabla}}
\newcommand{\Gradxt}{\ensuremath{\nabla_{xt}}}
\newcommand{\Div}{\ensuremath{\nabla\cdot}}
\newcommand{\Divxt}{\ensuremath{\nabla_{xt}\cdot}}
\newcommand{\Curl}{\ensuremath{\nabla\times}}
\newcommand{\bfH}{\mbox{\boldmath $H$}}
\newcommand{\bfsigma}{\boldsymbol\sigma}
\newcommand{\bfvarsigma}{\boldsymbol\varsigma}
\newcommand{\bftau}{\boldsymbol\tau}
\newcommand{\bfbeta}{\boldsymbol\beta}
\newcommand{\bflambda}{\boldsymbol\lambda}
\newcommand{\bfpsi}{\boldsymbol\psi}
\newcommand{\bfu}{\boldsymbol u}
\newcommand{\bfv}{\boldsymbol v}
\newcommand{\bfV}{\boldsymbol V}
\newcommand{\bfZ}{\boldsymbol Z}
\newcommand{\bfz}{\boldsymbol z}
\newcommand{\bfW}{\boldsymbol W}
\newcommand{\bfw}{\boldsymbol w}
\newcommand{\bfm}{\boldsymbol m}
\newcommand{\bfM}{\boldsymbol M}
\newcommand{\bbM}{\mathbb{M}}
\newcommand{\bfq}{\boldsymbol q}
\newcommand{\bfU}{\boldsymbol U}
\newcommand{\bfS}{\boldsymbol S}
\newcommand{\bbS}{\mathbb{S}}
\newcommand{\bbD}{\mathbb{D}}
\newcommand{\bfK}{\boldsymbol K}
\newcommand{\bbK}{\mathbb{K}}
\newcommand{\bfn}{\boldsymbol n}
\newcommand{\bff}{\boldsymbol f}
\newcommand{\bfF}{\boldsymbol F}
\newcommand{\bbF}{\mathbb{F}}
\newcommand{\bfg}{\boldsymbol g}
\newcommand{\bfG}{\boldsymbol G}
\newcommand{\bfC}{\boldsymbol C}
\newcommand{\bft}{\boldsymbol t}
\newcommand{\bfT}{\boldsymbol T}
\newcommand{\bfI}{\boldsymbol I}
\newcommand{\bbI}{\mathbb{I}}
\newcommand{\bfx}{\boldsymbol x}
\newcommand{\uh}{\widehat{u}}
\newcommand{\fnh}{\widehat{f}_n}
\newcommand{\LQ}{L^2\LRp{Q}}
\newcommand{\LK}{L^2\LRp{K}}
\newcommand{\LVecK}{\mathbf{L}^2\LRp{K}}
\newcommand{\LVecQ}{\mathbf{L}^2\LRp{Q}}
\newcommand{\HdivK}{\bfH(\text{div},K)}
\newcommand{\HdivOmega}{\bfH(\text{div},\Omega)}
% \newcommand{\HdivOmegaLT}{\bfH(\text{div},\Omega)\times L^2([0,T])}
\newcommand{\HdivQ}{\bfH(\text{div}_{xt},Q)}
\newcommand{\HOneK}{H^{1}(K)}
\newcommand{\HOneVecK}{\bfH^{1}(K)}
\newcommand{\HOneQ}{H^{1}(Q)}
\newcommand{\HOneOmegah}{H^{-1}(\Omega_h)}
\newcommand{\HdivOmegah}{\bfH(\text{div},\Omega_h)}
\newcommand{\vdeltau}{v_{\delta\bs u_h}}
\newcommand{\taudeltau}{\bftau_{\delta\bs u_h}}
\newcommand{\ip}[1]{\left\langle #1 \right\rangle}
\newcommand{\pd}[2]{\frac{\partial#1}{\partial#2}}
\newcommand{\pt}[1]{\frac{\partial#1}{\partial t}}
\newcommand{\ppd}[2]{\frac{\partial^2#1}{\partial#2^2}}
\newcommand{\pdd}[3]{\frac{\partial^2#1}{\partial#2\partial#3}}
\newcommand{\der}[2]{\frac{\mathrm{d}#1}{\mathrm{d}#2}}
\newcommand{\Oh}{\Omega_h}
\newcommand{\jump}[1] {\ensuremath{\LRs{\![#1]\!}}}
\newcommand{\Gh}{\Gamma_h}
\newcommand{\mcU}{\mathcal{U}}
\newcommand{\mcUh}{\hat{\mathcal{U}}}
\newcommand{\LOmega}{L^2\LRp{\Omega_h}}

\newcommand{\eqnref}[1]{\eqref{eq:#1}}

\DeclareMathOperator*{\argmin}{arg\,min}
\DeclareMathOperator*{\trace}{tr}

\def\arrtwo#1#2#3#4{\left[
\begin{array}{cc}
#1\; & #2\\
#3\; & #4\\
\end{array}
\right]}
\def\arrthree#1#2#3#4#5#6#7#8#9{\left[
\begin{array}{ccc}
#1\; & #2\; & #3\\
#4\; & #5\; & #6\\
#7\; & #8\; & #9\\
\end{array}
\right]}
\def\arrthreeone#1#2#3{\left[
\begin{array}{ccc}
#1\; & #2\; & #3\\
\end{array}
\right]}
\def\vecttwo#1#2{\left(
\begin{array}{c}
#1\\
#2\\
\end{array}
\right)}
\def\svecttwo#1#2{\left[
\begin{array}{c}
#1\\
#2\\
\end{array}
\right]}
\def\vectthree#1#2#3{\left(
\begin{array}{c}
#1\\
#2\\
#3\\
\end{array}
\right)}
\def\svectthree#1#2#3{\left[
\begin{array}{c}
#1\\
#2\\
#3\\
\end{array}
\right]}

\renewcommand{\arraystretch}{1.2}

\def\etal{{\it et al.~}}


\title{Robustness for transient problems}
\author{Truman E. Ellis and Jesse L. Chan}

\begin{document}
\maketitle

Assume that boundary conditions are applied on the boundary $\Gamma_0\subset \Gamma$.  Recall that, for the ultra-weak variational formulation
\[
b\LRp{\LRp{u,\uh},v} = \LRp{u,A^*_h v}_{\L} + \LRa{\uh, \jump{v}}_{\Gh\setminus \Gamma_0}
\]
we can recover
\[
\norm{u}_{\L}^2 = b(u,v^*)
\]
for conforming $v^*$ satisfying the adjoint equation
\begin{align*}
A^* v^* &= u\\
v^* &= 0 \text{ on } \Gh\setminus\Gamma_0.
\end{align*}
Together, these give necessary conditions on the test norm $\norm{\cdot}_V$ such that we have $L^2$ robustness (this gives robustness in the variable $u$; for the first order formulation, conditions for $\sigma$ must also be shown).  
\[
\norm{u}_{\L}^2 = b(u,v^*) \leq \frac{b(u,v^*)}{\norm{v^*}_V} \norm{v^*}_V \leq \norm{u}_E \norm{v^*}_V
\]
Thus, showing $\norm{v^*}_V \lesssim \norm{u}_{\L}$ gives the result that $\norm{u}_{\L} \lesssim \norm{u}_E$.  


\section{Reaction-diffusion}

Consider reaction diffusion
\begin{align*}
\pd{u}{t} + u - \epsilon \Delta u &= f\\
u &= 0 \text{ on } \Gamma_1\\
\pd{u}{n} &= 0 \text{ on } \Gamma_2\\
u(t=0) &= u_0.
\end{align*}
The adjoint equation satisfies
\begin{align*}
-\pd{v}{t} + v - \epsilon \Delta v &= u\\
v &= 0 \text{ on } \Gamma_1\\
\pd{v}{n} &= 0 \text{ on } \Gamma_2\\
v(t=T) &= 0.
\end{align*}
(The boundary conditions can be derived by taking the ultra-weak formulation and choosing boundary conditions such that the temporal flux and spatial flux terms $\LRa{\uh, \jump{\tau_n}}_{\Gamma_1}$ and $\LRa{\fnh,\jump{v}}_{\Gamma_2}$ are zero.)

We can then derive that the test norm
\[
\norm{v}_V^2 = \norm{\pd{v}{t}}^2 + \norm{v}^2 + \epsilon\norm{\Grad v}^2 
\]
provides the necessary bound $\norm{v^*}_V \lesssim \norm{u}_{\L}$.

To see, this we multiply the adjoint equation by two terms as follows:
\begin{enumerate}
\item Multiply by $v$ and integrate over $\Omega \times [0,T] = Q$ to get
\[
-\int_Q \pd{v}{t}v + \int_Q v^2 + \epsilon \int_Q \LRb{\Grad v}^2 - \epsilon \int_0^T\int_{\Gamma} \pd{v}{n}v = \int_Q uv.
\]
Noting that either $v = 0$ or $\pd{v}{n} = 0$ on the boundary removes the integral over $\Gamma$.  Next, we can factor the first term and use Young's inequality to get
\[
-\int_0^T  \pd{}{t}\int_{\Omega} v^2 + \norm{v}^2_Q + \epsilon \norm{\Grad v}^2_Q \leq \frac{1}{2}\norm{u}^2_Q + \frac{1}{2}\norm{v}^2_Q
\]
Integrating by parts the first term gives
\[
-\left.\int_{\Omega} v^2\right|_0^T + \frac{1}{2}\norm{v}^2_Q + \epsilon \norm{\Grad v}^2_Q \leq \frac{1}{2}\norm{u}^2_Q
\]
Using boundary condition $v=0$ at $t= T$ gives
\[
\frac{1}{2}\norm{v}^2_Q + \epsilon \norm{\Grad v}^2_Q \leq \int_{\Omega} v(t=0)^2 + \frac{1}{2}\norm{v}^2_Q + \epsilon \norm{\Grad v}^2_Q \leq \frac{1}{2}\norm{u}^2_Q.
\]

\item Multiply by $-\pd{v}{t}$ and integrate over $Q$.  Young's inequality changes the right hand side to 
\[
\int_Q\pd{v}{t}^2 - \int_Q v\pd{v}{t} + \epsilon\int_Q \Delta v \pd{v}{t} = \int_Q -u \pd{v}{t} \leq \frac{1}{2}\norm{u}_Q^2 + \frac{1}{2}\norm{\pd{v}{t}}_Q^2 .
\]
The term $\int_Q v\pd{v}{t}$ can be reduced to the positive contribution $\int_{\Omega}{v(t=0)}^2  $ as above.  We can then take the Laplacian term, integrate by parts in space to get
\[
\int_Q \Delta v \pd{v}{t} = \int_0^T \int_{\Omega} \Delta v \pd{v}{t} =  \int_0^T \int_{\Gamma} \pd{v}{t}\pd{v}{n} - \int_0^T \int_{\Omega}\Grad\LRp{\pd{v}{t}}\Grad v.
\]
Since either $v = 0$ or $\pd{v}{n} = 0$ on $\Gamma$, the first term disappears.  The second term can be bounded by noting
\[
- \int_0^T \int_{\Omega}\Grad\LRp{\pd{v}{t}}\Grad v = - \int_0^T \pd{}{t}\int_{\Omega}\LRb{\Grad v} ^2 = - \left.\int_{\Omega}\LRb{\Grad v}^2 \right|_0^T.
\]
Since $v = 0$ at $t=T$, $\Grad v = 0$ at $t=T$ as well, and we are left with the positive contribution $\int_{\Omega}\LRb{\Grad v(t=0)}^2$.  Then,
\[
\frac{1}{2}\norm{\pd{v}{t}}_Q^2 \leq \frac{1}{2}\norm{u}_Q.
\]
\end{enumerate}
Together, these two show that, under test norm
\[
\norm{v}_V^2 = \norm{\pd{v}{t}}^2 + \norm{v}^2 + \epsilon\norm{\Grad v}^2,
\]
the adjoint equation $v^*$ satisfies
\[
\norm{v^*}_V \lesssim \norm{u}_{\L}
\]
and thus the DPG energy norm robustly bounds the $L^2$ norm from above
\[
\norm{u}_{\L} \lesssim \norm{u}_E.
\]

\section{Convection-diffusion}

Truman, your turn :).

\section{Robustness for transient problems given spatial robustness}

Suppose we have the transient problem
\[
\pd{u}{t} + Au = f
\]
with initial condition $u(x,0) = u_0$.  Suppose that DPG is robust under the ultra-weak variational formulation for the steady problem
\[
\LRp{u,A^*_h v}_{\L} + \LRa{\uh, \jump{v}}_{\Gamma_h\setminus \Gamma_0} = \LRp{f,v}
\]
with test norm $\norm{v}_{V}$.  Then, can we show that 
\[
\norm{v}_{V,t} \coloneqq \norm{v}_V + \norm{\pd{v}{t}}_{\L}
\]
also leads to a robust upper bound of the $L^2$ norm by the DPG energy norm?  I believe this may be possible.  The adjoint equation for robustness for the transient problem gives
\[
-\pd{v}{t} + A^*v = u
\]
with $v = 0$ at $t=T$...  


\end{document}